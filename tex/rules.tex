\documentclass[11pt]{article}
\usepackage{fullpage}
\usepackage{latexsym}
\usepackage{verbatim}
\usepackage{amsthm}
\usepackage{amssymb}
\usepackage{amsmath}
\usepackage{stackengine}
\usepackage{scalerel}
\usepackage{code,proof,amsthm,amssymb, amsmath}
\usepackage{proof}
\usepackage{mathpartir}

\input{generic-defns}
\input{syn-defns}
\newcommand{\FunEvalsTo}[3]{{#1}\mathrel{\Downarrow'}\AbsABT{#2}{#3}}
\newcommand{\FunEvalsToHO}[3]{{#2}\vargen \fappabt{#1}{#2}\evalsto{#3}}

\newcommand{\arrtyabt}[2]{\OpABT{\cd{arr}}{#1;#2}}
\newcommand{\arrtycst}[2]{{#1}\mathrel{\to}{#2}}
\newcommand{\lamabt}[3]{\OpABTp{\cd{lam}}{#1}{\AbsABT{#2}{#3}}}
\newcommand{\lamcst}[3]{\mathop{\lambda}\cdparens{\TightIsOf{#2}{#1}}\,{#3}}
\newcommand{\appabt}[2]{\OpABT{\cd{ap}}{#1;#2}}
\newcommand{\appcst}[2]{{#1}\cdparens{#2}}

\newcommand{\letabt}[4]{\OpABTp{\cd{let}}{#1}{#2;\AbsABT{#3}{#4}}}
\newcommand{\letcst}[4]{\kwop{let}\TightIsOf{#3}{#1}\kwop{be}{#2}\kwop{in}{#4}}
\newcommand{\uletabt}[3]{\OpABT{\cd{let}}{#1;\AbsABT{#2}{#3}}}
\newcommand{\uletcst}[3]{\kwop{let}{#2}\kwop{be}{#1}\kwop{in}{#3}}

\newcommand{\fdefabt}[6]{\OpABTp{\cd{fun}}{#1;#2}{\AbsABT{#3}{#4};\AbsABT{#5}{#6}}}
\newcommand{\fdefcst}[6]{\kwop{fun}{#5}\TightIsOf{\cdparens{\TightIsOf{#3}{#1}}}{#2}\mathbin{\cd{=}}{#4}\kwop{in}{#6}}
\newcommand{\fappabt}[2]{\OpABTp{\cd{apply}}{#1}{#2}}
\newcommand{\fappcst}[2]{#1\cdparens{#2}}
\newcommand{\fsubst}[3]{\dblsqbracks{#1\mathord{/}#2}{#3}}


%%% Local Variables: 
%%% mode: latex
%%% TeX-master: "book"
%%% End: 

\newcommand{\LangPCF}{\textbsf{PCF}}

\newcommand{\genrecabt}[3]{\OpABTp{\cd{fix}}{#1}{\AbsABT{#2}{#3}}}
\newcommand{\genreccst}[3]{\kwop{fix}{#2}\cdcolon{#1}\kwop{is}{#3}}
\newcommand{\sgenreccst}[2]{\kwop{fix}{#1}\kwop{is}{#2}}

\newcommand{\bddrecabt}[4]{\OpABTp{\cd{fix}^{#1}}{#2}{\AbsABT{#3}{#4}}}
\newcommand{\bddreccst}[4]{\mathop{\cd{fix}^{#1}}{#3}\cdcolon{#2}\kwop{is}{#4}}

\newcommand{\parrtyabt}[2]{\OpABT{\cd{parr}}{#1;#2}}
\newcommand{\parrtycst}[2]{{#1}\mathbin{\rightharpoonup}{#2}}
\newcommand{\pappabt}[2]{\OpABT{\cd{ap}}{#1;#2}}
\newcommand{\pappcst}[2]{\appcst{#1}{#2}}

\newcommand{\funabt}[5]{\OpABTp{\cd{fun}}{#1;#2}{\AbsABT{#3}{\AbsABT{#4}{#5}}}}
\newcommand{\funcst}[5]{\kwop{fun}{#3}\cdparens{{#4}{\cdcolon}{#1}}{\cdcolon}{#2}\kwop{is}{#5}}

\newcommand{\empenv}{\bullet}
\newcommand{\extenv}[3]{{#1},{#2}{=}{#3}}

\newcommand{\expclo}[1]{\widehat{#1}}
\newcommand{\expclois}[2]{\expclo{#1}{=}{#2}}
\newcommand{\expenv}[2]{\widehat{#1}\parens{#2}}
\newcommand{\expenvis}[3]{\expenv{#1}{#2}{=}{#3}}

\newcommand{\cutoff}[2]{{#1}^{(#2)}}

%\newcommand{\Replace}[3]{\sqbracks{#2\mathbin{\leftarrow}#1}{#3}}

\newcommand{\lnattyabt}{\cd{lnat}}
\newcommand{\lnattycst}{\lnattyabt}
\newcommand{\lsuccabt}[2]{\OpABT{\cd{succ}}{\AbsABT{#1}{#2}}}
\newcommand{\lsucccst}[2]{{#1}\kwbin{is}\succcst{#2}}

%%% Local Variables: 
%%% mode: plain-tex
%%% TeX-master: "book"
%%% End: 

\newcommand{\unittyabt}{\kw{unit}}
\newcommand{\unittycst}{\kw{unit}}
\newcommand{\unittycstm}{\top}
\newcommand{\prodtyabt}[2]{\OpABT{\kw{prod}}{#1;#2}}
\newcommand{\prodtycst}[2]{{#1}\times{#2}}

\newcommand{\pprodtycst}[2]{{#1}\otimes{#2}}
\newcommand{\ppairexabt}[2]{\OpABT{\kw{fuse}}{#1;#2}}
\newcommand{\ppairexcst}[2]{{#1}\otimes{#2}}
\newcommand{\splitexabt}[4]{\OpABT{\kw{split}}{{#1};\AbsABT{#2,#3}{#4}}}
\newcommand{\splitexcst}[4]{\kw{split}\,{#1}\,\kw{as}\,\ppairexcst{#2}{#3}\,\kw{in}\,{#4}}
\newcommand{\checkexabt}[2]{\OpABT{\kw{check}}{#1;#2}}
\newcommand{\checkexcst}[2]{\kw{check}\,{#1}\,\kw{as}\,\unitexcst\,\kw{in}\,{#2}}

\newcommand{\unitexabt}{\kw{triv}}
\newcommand{\unitexcst}{\langle\rangle}
\newcommand{\pairexabt}[2]{\OpABT{\kw{pair}}{#1;#2}}
\newcommand{\pairexcst}[2]{\langle #1, #2\rangle}
\newcommand{\projexabt}[2]{\OpABT{\OpInst{\kw{pr}}{#2}}{#1}}
\newcommand{\fstexabt}[1]{\projexabt{#1}{\kw{l}}}
\newcommand{\sndexabt}[1]{\projexabt{#1}{\kw{r}}}
\newcommand{\projexcst}[2]{{#1}\mathbin\cdot{#2}}
\newcommand{\fstexcst}[1]{\projexcst{#1}{\kw{l}}}
\newcommand{\sndexcst}[1]{\projexcst{#1}{\kw{r}}}

\newcommand{\dgenprodcst}[1]{\brackets{#1}}
\newcommand{\dgentuplecst}[1]{\brackets{#1}}
\newcommand{\genprodabt}[3]{\OpABT{\kw{prod}}{\genff{#1}{#2}{#3}}}
\newcommand{\vargenprodcst}[3]{\prod_{{#2}\in{#1}}{#3}}
\newcommand{\genprodcst}[3]{\dgenprodcst{#3}_{#2\in #1}}
\newcommand{\gentupleabt}[3]{\OpABT{\kw{tpl}}{\genff{#1}{#2}{#3}}}
\newcommand{\gentuplecst}[3]{\dgentuplecst{#3}_{{#2}\in{#1}}}
\newcommand{\lgentuplescst}[3]{\gentuplecst{#1}{#2}{\fldexcst{#2}{#3}}}
\newcommand{\genprojabt}[3]{\projexabt{#3}{#2}}
\newcommand{\genprojcst}[3]{\projexcst{#3}{#2}}
\newcommand{\genprodcstgen}[3]{\dgenprodcst{\varexplff{#1}{#2}{#3}}}
\newcommand{\gentuplecstgen}[3]{\dgentuplecst{\varexplff{#1}{#2}{#3}}}

\newcommand{\rcdtycst}[1]{\dgenprodcst{#1}}
\newcommand{\rcdexcst}[1]{\dgentuplecst{#1}}
\newcommand{\fldtycst}[2]{\pairff{#1}{#2}}
\newcommand{\fldexcst}[2]{\pairff{#1}{#2}}
\newcommand{\fldselexcst}[2]{\projexcst{#2}{#1}}


%%% Local Variables: 
%%% mode: latex
%%% TeX-master: "book"
%%% End: 

\input{../pfpl/sum-defns}
\input{../pfpl/icoi-defns}
\newcommand{\LangT}{\textbsf{T}}

\newcommand{\nattyabt}{\cd{nat}}
\newcommand{\nattycst}{\nattyabt}

\newcommand{\zeroabt}{\cd{z}}
\newcommand{\zerocst}{\zeroabt}
\newcommand{\succabt}[1]{\OpABT{\cd{s}}{#1}}
\newcommand{\succcst}[1]{\succabt{#1}}

\newcommand{\predabt}[1]{\OpABT{\cd{p}}{#1}}
\newcommand{\predcst}[1]{\predabt{#1}}

\newcommand{\numeral}[1]{\overline{#1}}

\newcommand{\natrecabt}[6]{\OpABTp{\kw{rec}}{#3;\AbsABT{#4}{\AbsABT{#5}{#6}}}{#2}}
\newcommand{\natreccst}[6]{\kwop{rec}{#2}\,\cdbraces{\zerocst\casebrcst{#3}\casesepcst\succcst{#4}\kwop{with}{#5}\casebrcst{#6}}}

\newcommand{\natiterabt}[5]{\OpABTp{\cd{iter}}{#3;\AbsABT{#4}{#5}}{#2}}
\newcommand{\natitercst}[5]{\kwop{iter}{#2}\,\cdbraces{\zerocst\casebrcst{#3}\casesepcst\succcst{#4}\casebrcst{#5}}}

\newcommand{\natcaseabt}[4]{\OpABTp{\cd{ifz}}{#2;\AbsABT{#3}{#4}}{#1}}
\newcommand{\natcasecst}[4]{\kwop{ifz}{#1}\,\cdbraces{\zerocst\casebrcst{#2}\casesepcst\succcst{#3}\casebrcst{#4}}}

\newcommand{\nullstrabt}{\epsilon}
\newcommand{\consstrabt}[2]{{#1}\cdot{#2}}

%%% Local Variables: 
%%% mode: latex
%%% TeX-master: "book"
%%% End: 


% =========================================================================== %
%                                                                             %
%                          Using this LaTeX Template                          %
%                                                                             %
% - new tasks are on their own section (how Gradescope expects them)          %
% - use '\task' to start a new task                                           %
% - use 'begin{task} ... \end{task}' if you'd like to preface your answer     %
%   with the question itself (i.e., fill in the '...' with the question)      %
% - use '\nextgroup' to advance from, for example, Task 1.4 to Task 2.1       %
% - use '\skipaheadtask' to skip from, for example, Task 2.2 to Task 2.4      %
%                                                                             %
% You also have access to all the definitions from the handout. See defs.tex, %
% syn-defns.tex, and generic-defns.tex.                                       %
%                                                                             %
%               TODO: Fill in your personal information below!                %
%                                                                             %
% =========================================================================== %
\newcommand{\myname}{Andrew Carnegie}
\newcommand{\myandrewid}{andrew}
\newcommand{\hwnumber}{1}
% =========================================================================== %

\newcounter{group}
\setcounter{group}{1}
\newtheorem{theorem}{Task}[group]
% Remove '\newpage' below to preview your doc compactly.
% Remember to put it back before submitting to Gradescope.
\newcommand{\task}{\newpage\begin{theorem}\end{theorem}}
\newcommand{\nextgroup}{\stepcounter{group}}
\newcommand{\skipaheadtask}{\stepcounter{theorem}}
\newcommand{\ms}[1]{\ensuremath{\mathsf{#1}}}
\newcommand{\irl}[1]{\mathtt{#1}}
\newcounter{rule}
\setcounter{rule}{0}
\newcommand{\rn}
  {\addtocounter{rule}{1}(\arabic{rule})}

\newcounter{infercount}
\setcounter{infercount}{1}
\newcommand{\infern}[2]{\inferrule{#1}{#2}(\text{S}_{\arabic{infercount}}\stepcounter{infercount})}

\title{15-312 Assignment \hwnumber}
\author{\myname\\(\myandrewid)}
\date{\today}

\begin{document}
\maketitle

\section{Syntax}

\[
\begin{array}{r l l l l}
\ms{Type} & \tau \,\,\,\,\, ::= \\
	& \irl{nat}                	 			& \irl{nat}											& \text{naturals}\\
	& \unittyabt                	 			& \unittycst										& \text{unit}\\
  & \booltyabt                       & \booltycst                    & \text{boolean}\\
  & \prodtyabt{\tau_1}{\tau_2}       & \prodtycst{\tau_1}{\tau_2}    & \text{product}\\
	&\irl{arr}(\tau_1;\tau_2) 				& \arrtycst{\tau_1}{\tau_2} 									& \text{function}\\
  &\listtyabt{\tau}		& \listtycst{\tau}											& \text{list}\\
	 \\
\ms{Exp}
        & e   \,\,\,\,\, ::= \\
 	& x                                			& x 												& \text{variable}\\
  & \irl{nat}[n]							& \numeral{n}												& \text{number}\\
  & \irl{unit}							& ()												& \text{unit}\\
  & \irl{T}							& \irl{T}												& \text{true}\\
  & \irl{F}	   					& \irl{F}												& \text{false}\\
  & \ifexabt{e}{e_1}{e_2} & \ifexcst{e}{e_1}{e_2}  & \text{if}\\
  & \irl{lam}(x:\tau.e) 						&\lambda \; x : \tau. e 		& \text{abstraction}\\
  & \irl{ap}(e_1;e_2) 					& \appcst{e_1}{e_2} 										& \text{application}\\
  & \irl{tpl}(e_1;e_2)     	& \pairexcst{e_1}{e_2}                									& \text{tuple}\\
 	& \irl{fst}(e)					& \fstexcst{e}   										& \text{first projection}\\
 	& \irl{snd}(e)					& \sndexcst{e}   										& \text{second projection}\\
 	& \nilexabt					& []   										& \text{nil}\\
 	& \consexabt{e_1}{e_2}					& e_1::e_2   										& \text{cons}\\
 	& \listcaseexabt{e}{e_1}{x}{xs}{e_2}					& \listcaseexcst{e}{e_1}{x}{xs}{e_2}   	& \text{match list}\\
  \\
\ms{Val}
        & v   \,\,\,\,\, ::= \\
 	& \irl{val}[l](n)                                			& n^l 												& \text{numeric value}\\
 	& \irl{val}[l](\irl{T})                               			& \irl{T}^l 								  & \text{true value}\\
 	& \irl{val}[l](\irl{F})                                			& \irl{F}^l								  & \text{false value}\\
 	& \irl{val}[l](\irl{Null})                                  & \irl{Null}^l 								  & \text{null value}\\
 	& \irl{val}[l](\irl{cl}(V; x.e))                & (V, x.e)^l 					& \text{function value}\\
 	& \irl{val}[l_2](l_1)                                			& l_1^{l_2} 								  & \text{loc value}\\
 	& \irl{val}[l](\pairexabt{v_1}{v_2})                             & \pairexcst{v_1}{v_2}^l 								  & \text{pair value}\\
  \\
\ms{Loc}
        & l   \,\,\,\,\, ::= \\
 	& \irl{loc}(l)                                			& l 												& \text{location}\\
\end{array}
\]

\section{Heap semantics}

Model dynamics using judgement of the form:
\[
\fbox{$V,H,R \; \vdash e \Downarrow^s v, H'$}
\]

Where $V : VID \to Val$, $H : Loc \to Val$, and $R : \{Loc\}$. This can be read as: under stack $V$, heap $H$, and roots $R$,
the expression $e$ evaluates to $v$ using maximum heap space $s$, and engenders a new heap $H'$.\\

Roots represents the set of locations required to compute the continuation \emph{excluding} the current expression.
We can think of roots as the heap allocations necessary to compute the context with a hole that will be filled
by the current expression.\\

Below defines the size of reachable values and space for roots:

\begin{align*}
  &reach_{H}(n^l) = \{l\}\\
  &reach_{H}(\irl{T}^l) = \{l\}\\
  &reach_{H}(\irl{F}^l) = \{l\}\\
  &reach_{H}(\irl{Null}^l) = \{l\}\\
  &reach_{H}((V, x.e)^l) = \{l\} \cup (\bigcup\limits_{y \in FV(e) \setminus x} reach_H(V(y))) \\
  &reach_{H}(l_1^{l_2}) = \{l_2\} \cup loc_{H}(H(l_1))\\
  &reach_{H}(\pairexcst{v_1}{v_2}^l) = \{l\} \cup reach_{H}(v_1) \cup reach_{H}(v_2)\\\\
  &loc_{H}(l) = \{l\} \cup reach_{H}(H(l))\\\\
  &space_{H}(R) = |\bigcup\limits_{l \in R} loc_{H}(l)|\\\\
  &locs_{V,H}(e) = \bigcup\limits_{x \in FV(e)} reach_H(V(x))
\end{align*}

\begin{mathpar}

\infern
{ x \in dom(V)
}
{V,H,R \; \vdash x \Downarrow^{space_H(R \cup (reach_H(V(x))))} V(x),H}

\infern
{
  (l \; \text{fresh})\\
  H' = H[l \mapsto n^l]
}{
  V,H,R \; \vdash \numeral{n} \Downarrow^{space_{H'}(R \cup (\{l\})} n^l,H'
}

\infern{
  (l \; \text{fresh})\\
  H' = H[l \mapsto \irl{T}^l]
}{
  V,H,R \; \vdash \irl{T} \Downarrow^{space_{H'}(R \cup \{l\})} \irl{T}^l,H'
}

\infern{
  (l \; \text{fresh})\\
  H' = H[l \mapsto \irl{F}^l]
}{
  V,H,R \; \vdash \irl{F} \Downarrow^{space_{H'}(R \cup \{l\})} \irl{F}^l,H'
}

\infern{
  (l \; \text{fresh}) \\
  H' = H[l \mapsto \irl{Null}^l]
}{
  V,H,R \; \vdash () \Downarrow^{space_{H'}(R \cup \{l\})} \irl{Null}^l,H'
}

\infern{
  V,H,R \cup locs_{V,H}(e_1) \cup locs_{V,H}(e_2) \; \vdash e \Downarrow^s \irl{T}^l, H'\\
  V,H',R \; \vdash e_1 \Downarrow^{s_1} v_1, H_1
}{
  V,H,R \; \vdash \ifexabt{e}{e_1}{e_2} \Downarrow^{\max{(s,s_1)}} v_1, H_1
}

\infern{
  V,H,R \cup locs_{V,H}(e_1) \cup locs_{V,H}(e_2) \; \vdash e \Downarrow^s \irl{F}^l, H'\\
  V,H',R \; \vdash e_2 \Downarrow^{s_2} v_2, H_2
}{
  V,H,R \; \vdash \ifexabt{e}{e_1}{e_2} \Downarrow^{\max{(s,s_2)}} v_2, H_2
}

% function

\infern{
  (l \; \text{fresh})\\
  H' = H[l \mapsto (V,x.e)^l]
}{
  V,H,R \; \vdash \irl{lam}(x : \tau.e) \Downarrow^{space_{H'}(R \cup \{l\})} (V, x.e)^l,H'
}

\infern{
  V,H,R \cup locs_{V,H}(e_1) \; \vdash e_2 \Downarrow^{s_2} v_2, H_2 \\
  V,H_2,R \cup reach_{H_2}(v_2) \; \vdash e_1 \Downarrow^{s_1} v_1, H_1 \\
  v_1 = (V_1, x.e)^{l_1} \\
  V_1[x \mapsto v_2], H_1, R \; \vdash e \Downarrow^s v, H'
}{
  V,H,R \; \vdash \appcst{e_1}{e_2} \Downarrow^{\max{(s_2, s_1, s)}} v,H'
}

% tuples

\infern{
  V,H,R \cup locs_{V,H}(e_2) \; \vdash e_1 \Downarrow^{s_1} v_1, H_1 \\
  V,H_1,R \cup reach_{H_1}(v_1) \; \vdash e_2 \Downarrow^{s_2} v_2, H_2 \\
  (l \; \text{fresh})\\
  H' = H_2[l \mapsto \pairexcst{v_1}{v_2}^l]
}{
  V,H,R \; \vdash \pairexcst{e_1}{e_2} \Downarrow^{\max{(s_1, s_2)+1}} \pairexcst{v_1}{v_2}^l,H'
}

\infern{
  V,H,R \; \vdash e \Downarrow^s \pairexcst{v_1}{v_2}^l, H' \\
}{
  V,H,R \; \vdash \fstexcst{e} \Downarrow^s v_1,H'
}

\infern{
  V,H,R \; \vdash e \Downarrow^s \pairexcst{v_1}{v_2}^l, H' \\
}{
  V,H,R \; \vdash \sndexcst{e} \Downarrow^s v_2,H'
}

% lists

\infern{
  (l \; \text{fresh})\\
  H' = H[l \mapsto \irl{Null}^l]
}{
  V,H,R \; \vdash \nilexabt \Downarrow^{space_{H'}(R \cup \{l\})} \irl{Null}^l,H'
}

\infern{
  V,H,R \cup locs_{V,H}(e_2)\; \vdash e_1 \Downarrow^{s_1} v_1, H_1 \\
  V,H_1,R \cup reach_{H_1}(v_1) \; \vdash e_2 \Downarrow^{s_2} v_2, H_2 \\
  (l \; \text{fresh})\\
  H' = H_2[l \mapsto \pairexcst{v_1}{v_2}^l]
}{
  V,H,R \; \vdash \consexcst{e_1}{e_2} \Downarrow^{\max{(s_1,s_2)+1}} \pairexcst{v_1}{v_2}^l,H'
}

\infern{
  V,H,R \cup locs_{V,H}(e_1) \cup locs_{V,H}(e_2)\; \vdash e \Downarrow^{s} \irl{Null}^l, H' \\
  V,H',R \; \vdash e_1 \Downarrow^{s_1} v_1, H_1 \\
}{
  V,H,R \; \vdash \listcaseexcst{e}{e_1}{x}{xs}{e_2} \Downarrow^{\max{(s,s_1)}} v_1,H_1
}

\infern{
  V,H,R \cup locs_{V,H}(e_1) \cup locs_{V,H}(e_2)\; \vdash e \Downarrow^{s} \pairexcst{v_h}{v_t}^l, H' \\
  V[x \mapsto v_h, xs \mapsto v_t],H',R \; \vdash e_2 \Downarrow^{s_2} v_2, H_2 \\
}{
  V,H,R \; \vdash \listcaseexcst{e}{e_1}{x}{xs}{e_2} \Downarrow^{\max{(s,s_2)}} v_2,H_2
}
\end{mathpar}


\end{document}
