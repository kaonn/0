\documentclass[11pt]{article}
\usepackage{fullpage}
\usepackage{latexsym}
\usepackage{verbatim}
\usepackage{amsthm}
\usepackage{amssymb}
\usepackage{amsmath}
\usepackage{stackengine}
\usepackage{scalerel}
\usepackage{code,proof,amsthm,amssymb, amsmath}
\usepackage{proof}
\usepackage{mathpartir}
\usepackage{turnstile}
\usepackage{fancyvrb}
\usepackage[shortlabels]{enumerate}
\allowdisplaybreaks

% generic definitions

%\DeclareMathAccent{\colonaccent}{\mathpunct}{upright}{"3A}

% create a wavy division sign
\stackMath
\def\ccdot{\scalebox{1.15}{$\SavedStyle\cdot$}}
\def\genericform#1#2{\stackunder[#1]{\stackon[#2]{\SavedStyle\sim}{\ccdot}}{\ccdot}}
\def\altdiv{\mathrel{\ThisStyle{\mathchoice%
  {\genericform{-2.5pt}{-1.5pt}}%
  {\genericform{-2.5pt}{-1.5pt}}%
  {\genericform{-2.0pt}{-1.1pt}}%
  {\genericform{-1.4pt}{-0.8pt}}%
}}}

\newcommand{\optional}[1]{\lbrack #1\rbrack}

\newcommand{\VisibleSpace}{\mbox{\verb*/ /}}

\newcommand{\Domain}[1]{\mathop{\mathit{dom}}(#1)}
\newcommand{\Disjoint}[2]{{#1}\mathrel{\#}{#2}}
\newcommand{\NotInDom}[2]{#1\notin\Domain{#2}}
\newcommand{\NotIn}[2]{{#1}\notin{#2}}
\newcommand{\eqdef}{\mathrel{\triangleq}}
\newcommand{\isdef}{\eqdef}

\newcommand{\bnfdef}{\coloncolonequals}
\newcommand{\phantombnfdef}{\mathrel{\phantom{\bnfdef}}}
\newcommand{\bnfalt}{\mathrel{\mid}}

\newcommand{\pto}{\rightharpoonup}
\newcommand{\backsl}{\ensuremath{\backslash}}
\newcommand{\turnstile}{\ensuremath{\vdash}}
\newcommand{\twiddle}{\raisebox{-.1\height}{\texttildelow}}
%\newcommand{\twiddle}{\ensuremath{\tilde{\ }}}
\newcommand{\caret}{\ensuremath{\hat{\ }}}

\newcommand{\parens}[1]{({#1})}
\newcommand{\braces}[1]{\{{#1}\}}
\newcommand{\sqbracks}[1]{[{#1}]}
\newcommand{\ptbracks}[1]{\langle #1\rangle}
\newcommand{\brackets}[1]{\ptbracks{#1}}
\newcommand{\corners}[1]{\ulcorner #1\urcorner}
\newcommand{\dblptbracks}[1]{\langle\!\langle #1\rangle\!\rangle}
\newcommand{\dblsqbracks}[1]{[\![ #1]\!]}

\newcommand{\kw}[1]{\ensuremath{\mathtt{#1}}}
\newcommand{\kwop}[1]{\ensuremath{\mathop{\mathtt{#1}}}}
\newcommand{\kwbin}[1]{\ensuremath{\mathbin{\mathtt{#1}}}}

\newcommand{\cdcolon}{\kwbin{:}}
\newcommand{\cdcoloncolon}{\kwbin{::}}
\newcommand{\cdcomma}{\kw{,}}
\newcommand{\cdperiod}{\kw{.}}
\newcommand{\cddot}{\cdperiod}
\newcommand{\cddotdot}{\kw{{.}{.}}}
\newcommand{\cdsemi}{\kw{;}}
\newcommand{\cdbar}{\kwbin{|}} %|
\newcommand{\cdatsign}{\kw{@}}
\newcommand{\cdunderscore}{\kw{\_}}
\newcommand{\cdsharp}{\kw{\#}}
\newcommand{\cddollar}{\kw{\$}}
\newcommand{\cdtwiddle}{\kw{\~}}
\newcommand{\cdhat}{\kw{\^{ }}}
\newcommand{\cdstar}{\kw{*}}
\newcommand{\cdamper}{\kw{\&}}

\newcommand{\seq}[2]{{#2}_1,\dots,{#2}_{#1}}
\newcommand{\zeq}[2]{{#2}_0,\dots,{#2}_{#1-1}}
\newcommand{\seqsep}[3]{{#3}_1\mathbin{#1}\dots\mathbin{#1}{#3}_{#2}}
\newcommand{\zeqsep}[3]{{#3}_0\mathbin{#1}\dots\mathbin{#1}{#3}_{#2-1}}

\newcommand{\abs}[1]{\lvert#1\rvert}

\newcommand{\cdparens}[1]{\kw{(}{#1}\kw{)}}
\newcommand{\cdnoparens}[1]{#1}
\newcommand{\noparens}[1]{#1}
\newcommand{\cdcdc}{\kw{,}\dots\kw{,}}
\newcommand{\cdcds}{\kw{;}\dots\kw{;}}
\newcommand{\cdseq}[2]{#2_{1}\cdcdc #2_{#1}}
\newcommand{\cdzeq}[2]{#2_{0}\cdcdc #2_{#1-1}}
\newcommand{\cdsseq}[2]{#2_{1}\cdcds #2_{#1}}
\newcommand{\cdszeq}[2]{#2_{0}\cdcds #2_{#1-1}}
\newcommand{\cdpseq}[4]{{#3}_{1}{#2}{#4}_{1}\cdcdc{#3}_{#1}{#2}{#4}_{#1}}
\newcommand{\cdpzeq}[4]{{#3}_{0}{#2}{#4}_{0}\cdcdc{#3}_{#1-1}{#2}{#4}_{#1-1}}
\newcommand{\cdsqbracks}[1]{\kw{[}{#1}\kw{]}}
\newcommand{\cdptbracks}[1]{\kw{<}{#1}\kw{>}}
\newcommand{\cdrdbracks}[1]{\cdparens{#1}}
\newcommand{\cdbrackets}[1]{\cdptbracks{#1}}
\newcommand{\cdbraces}[1]{\kw{\{}{#1}\kw{\}}}
\newcommand{\cdcurlbracks}[1]{\cdbraces{#1}}

\newcommand{\IsType}[1]{\JPost{#1}{\textsf{type}}}
\newcommand{\IsTp}[1]{\IsType{#1}}
\newcommand{\IsKd}[1]{\JPost{#1}{\textsf{kind}}}
\newcommand{\EqTy}[2]{{#1}\equiv{#2}}
\newcommand{\EqTp}[2]{\EqTy{#1}{#2}}
\newcommand{\EqKd}[2]{{#1}\equiv{#2}}
\newcommand{\IsOf}[2]{{#1}\mathrel{:}{#2}}
\newcommand{\IsOfP}[2]{\IsOf{#1}{#2}}
\newcommand{\IsOfM}[2]{{#1}\mathrel{\altdiv}{#2}}
\newcommand{\IsOfMW}[3]{{#1}\mathrel{\altdiv}{#2}\mathrel{@}{#3}}
\newcommand{\IsOfW}[3]{{#1}\mathbin{:}{#2}\mathbin{@}{#3}}
\newcommand{\Has}[2]{{#1}\mathbin{\twiddle}{#2}}
\newcommand{\HasW}[3]{{#1}\mathbin{\twiddle}{#2}\mathbin{@}{#3}}
\newcommand{\IsOfKd}[2]{{#1}\mathrel{\coloncolon}{#2}}
\newcommand{\TightIsOfKd}[2]{{#1}\mathbin{\coloncolon}{#2}}
\newcommand{\TightIsOf}[2]{{#1}\mathbin{:}{#2}}
\newcommand{\NIsOfKd}[2]{{#1}\mathrel{\Uparrow}{#2}}
\newcommand{\CIsOfKd}[2]{{#1}\mathrel{\Downarrow}{#2}}
\newcommand{\EqOfKd}[3]{{#1}\mathrel{\equiv}{#2}\mathrel{\coloncolon}{#3}}
\newcommand{\IsVal}[1]{\JPost{#1}{\textsf{val}}}
\newcommand{\IsntVal}[1]{\JPost{#1}{\mathop{\neg}\textsf{val}}}
\newcommand{\LIsVal}[2]{\JPost{#2}{\textsf{val}_{#1}}}
\newcommand{\IsErr}[1]{\JPost{#1}{\textsf{err}}}
\newcommand{\IsOK}[1]{\JPost{#1}{\textsf{ok}}}
\newcommand{\LIsOK}[2]{\JPost{#2}{\textsf{ok}_{#1}}}

\newcommand{\IsOfSy}[2]{{#1}\uparrow{#2}}
\newcommand{\IsOfAn}[2]{{#1}\downarrow{#2}}

\newcommand{\IsOpnVal}[2]{{#1}\vdash\IsVal{#2}}

\newcommand{\IsMobile}[1]{\JPost{#1}{\textsf{mobile}}}

\newcommand{\DefEqv}[3]{\IsOf{{#1}\equiv{#2}}{#3}}
\newcommand{\IsDefEqvCl}[3]{\DefEqv{#1}{#2}{#3}}
\newcommand{\IsDefEqv}[4]{{#1}\vdash\DefEqv{#2}{#3}{#4}}
\newcommand{\IsDefEqvU}[3]{{#1}\vdash{#2}\equiv{#3}}
\newcommand{\DefEqvU}[2]{{#1}\equiv{#2}}

\newcommand{\JPre}[2]{{#2}\;{#1}}
\newcommand{\JPPre}[2]{{#2}\parens{#1}}
\newcommand{\JPost}[2]{{#1}\;{#2}}
\newcommand{\JPPost}[2]{\parens{#1}\;{#2}}
%\newcommand{\JPostTight}[2]{{#1}\,{#2}}
\newcommand{\JInfix}[3]{{#1}\mathrel{#2}{#3}}

% \newcommand{\imode}{{\forall}}
% \newcommand{\omode}{{\exists}}
% \newcommand{\oumode}{{\exists!}}
% \newcommand{\opmode}{{\exists^{\leq 1}}}

%\newcommand{\tytransto}[2]{{#1}\leadsto {#2}}
%\newcommand{\extransto}[2]{{#1}\leadsto {#2}}
% \newcommand{\extranstostacked}[2]{%
%   \begin{array}[c]{c}
%     {#1} \\
%     \leadsto \\
%     {#2}
%   \end{array}
% }

% finite functions
\newcommand{\ff}[1]{\braces{#1}}
\newcommand{\pairff}[2]{{#1}\mathbin{\hookrightarrow}{#2}}
\newcommand{\empff}{\emptyset}
\newcommand{\singff}[2]{\ff{\pairff{#1}{#2}}}
\newcommand{\combff}[2]{#1\mathbin{\otimes}#2}
\newcommand{\extff}[3]{\combff{#1}{\singff{#2}{#3}}}
\newcommand{\varextff}[3]{\combff{#1}{\pairff{#2}{#3}}}
\newcommand{\appff}[2]{{#1}\parens{#2}}
\newcommand{\explff}[3]{\ff{\pairff{{#2}_{0}}{{#3}_{0}},\dots,\pairff{{#2}_{{#1}-1}}{{#3}_{{#1}-1}}}}
\newcommand{\varexplff}[3]{\pairff{#2_{1}}{#3_{1}},\dots,\pairff{#2_{#1}}{#3_{#1}}}
\newcommand{\domff}[1]{\Domain{#1}}
\newcommand{\disjff}[2]{\domff{#1}\cap\domff{#2}=\emptyset}
\newcommand{\genff}[3]{\ff{{#2}\mathbin{\hookrightarrow}{#3}}_{{#2}\in{#1}}}

% families of judgments
\newcommand{\FamInst}[2]{{#1}\,[{#2}]}
\newcommand{\JFam}[3]{\FamInst{\JPPre{#3}{#1}}{#2}}

% language names
\newcommand{\textbsf}[1]{\textbf{\textsf{#1}}}
\newcommand{\Lang}[1]{\textbsf{#1}}

\newcommand{\calA}{\mathcal{A}}
\newcommand{\calB}{\mathcal{B}}
\newcommand{\calC}{\mathcal{C}}
\newcommand{\calE}{\mathcal{E}}
\newcommand{\calG}{\mathcal{G}}
\newcommand{\calI}{\mathcal{I}}
\newcommand{\calL}{\mathcal{L}}
\newcommand{\calO}{\mathcal{O}}
\newcommand{\calP}{\mathcal{P}}
\newcommand{\calQ}{\mathcal{Q}}
\newcommand{\calR}{\mathcal{R}}
\newcommand{\calS}{\mathcal{S}}
\newcommand{\calU}{\mathcal{U}}
\newcommand{\calV}{\mathcal{V}}
\newcommand{\calX}{\mathcal{X}}
\newcommand{\calY}{\mathcal{Y}}

% general judgment
\newcommand{\vargen}{\mathrel{\vert}}
\newcommand{\symgen}{\mathrel{\Vert}}

%%% Local Variables: 
%%% mode: latex
%%% TeX-master: "book"
%%% End: 

% generic symbols
\newcommand{\IsSymb}[1]{\JPost{#1}{\mathsf{sym}}}
\newcommand{\IsSym}[1]{\IsSymb{#1}}
\newcommand{\Diff}[2]{{#1}\not={#2}}
\newcommand{\DiffSymb}[2]{\IsSymb{\Diff{#1}{#2}}}

% strings
\newcommand{\IsChar}[1]{\JPost{#1}{\mathsf{char}}}
\newcommand{\IsString}[2]{\JPost{#2}{\mathsf{str}_{#1}}}
\newcommand{\IsStr}[1]{\JPost{#1}{\mathsf{str}}}
\newcommand{\IsCharStr}[1]{\IsString{\mathsf{char}}{#1}}
\newcommand{\NullString}{\epsilon}
\newcommand{\ConsString}[2]{{#1}\cdot{#2}}
\newcommand{\SingString}[1]{\ConsString{#1}{\NullString}}
\newcommand{\SingStringStar}[1]{#1}
\newcommand{\ConcString}[2]{{#1}\mathbin{\hat{\ }}{#2}}
\newcommand{\ConcStringStar}[2]{{#1}\,{#2}}
\newcommand{\IsConc}[3]{\JPPre{#1;#2;#3}{\mathsf{conc}}}
\newcommand{\ConcIs}[3]{\ConcString{#1}{#2}={#3}}
\newcommand{\IsEqStr}[2]{{#1}={#2}}
\newcommand{\LenIs}[2]{|{#1}|={#2}}

% permutations
\newcommand{\IdPerm}{\mathop{\mathit{id}}}
\newcommand{\SwapPerm}[2]{\sqbracks{{#1}\leftrightarrow{#2}}}
\newcommand{\CompPerm}[2]{{#1}\circ{#2}}

% sorts
\newcommand{\Sort}[1]{\textsf{#1}}
\newcommand{\Op}[1]{\kw{#1}}
\newcommand{\Ar}[2]{\parens{#1}{#2}}
\newcommand{\Vl}[2]{\noparens{#1}\cddot{#2}}
\newcommand{\OpInst}[2]{{#1}\cdsqbracks{#2}}

\newcommand{\ExprSort}{\Sort{Exp}}
\newcommand{\TypeSort}{\Sort{Typ}}
\newcommand{\RefSort}{\Sort{Ref}}

% asts
\newcommand{\IsOpn}[1]{\JPost{#1}{\mathsf{opn}}}

\newcommand{\OpAST}[2]{{#1}\cdparens{#2}}
\newcommand{\AbsAST}[2]{{#1}\cddot{#2}}

\newcommand{\BvarABG}[1]{\OpInst{\kw{bv}}{#1}}
\newcommand{\AbsABG}[1]{\cddot{#1}}

\newcommand{\OpABT}[2]{{#1}\cdparens{#2}}
\newcommand{\OpABTp}[3]{{#1}\cdbraces{#2}\cdparens{#3}}
\newcommand{\OpABTpp}[4]{{#1}\cdbraces{#2}\cdbraces{#3}\cdparens{#4}}
\newcommand{\OpABTn}[2]{{#1}\cdbraces{#2}}
\newcommand{\AbsABT}[2]{{#1}\cddot{#2}}
\newcommand{\SymABT}[2]{\AbsABT{#1}{#2}}

\newcommand{\ListABT}[2]{{#1}\cdsemi\dots\cdsemi{#2}}
\newcommand{\SeqABT}[2]{\ListABT{{#2}_1}{{#2}_{#1}}}
\newcommand{\ZeqABT}[2]{\ListABT{{#2}_0}{{#2}_{#1}}}

% size

\newcommand{\SzABTn}[3]{\mathop{\mathsf{sz}}(\IsABTn{#1}{#2})=#3}

% apartness, occurrence
\newcommand{\IsFree}[2]{{#1}\in{#2}}
\newcommand{\IsntFree}[2]{{#1}\notin{#2}}
\newcommand{\IsApart}[2]{{#1}\notin{#2}}
\newcommand{\Apart}[2]{\IsApart{#1}{#2}}

\newcommand{\Swap}[3]{\mathop{\SwapPerm{#1}{#2}}{#3}}
\newcommand{\SwapIs}[4]{\Swap{#1}{#2}{#3}={#4}}

%\newcommand{\PermAct}[2]{{#1}\mathbin{\cdot}{#2}}
\newcommand{\PermAct}[2]{\widehat{#1}\parens{#2}}
\newcommand{\PermActIs}[3]{\PermAct{#1}{#2}={#3}}
\newcommand{\PermActIsAST}[3]{\IsAST{\PermActIs{#1}{#2}{#3}}}

% \newcommand{\IsVar}[1]{\JPost{#1}{\mathsf{var}}}

\newcommand{\Subst}[3]{\sqbracks{{#1}\mathord{/}{#2}}{#3}}
\newcommand{\SubstIs}[4]{\Subst{#1}{#2}{#3}\mathbin{=}{#4}}
\newcommand{\SubstIsAST}[4]{\IsAST{\SubstIs{#1}{#2}{#3}{#4}}}
\newcommand{\SubstIsABTn}[5]{\IsABTn{\SubstIs{#1}{#2}{#3}{#4}}{#5}}
\newcommand{\SubstIsABT}[4]{\IsABT{\SubstIsAbt{#1}{#2}{#3}{#4}}}

\newcommand{\AlphaEq}[2]{{#1}=_\alpha{#2}}
\newcommand{\AlphaEqABTn}[3]{\IsABTn{\AlphaEq{#1}{#2}}{#3}}
\newcommand{\AlphaEqABT}[2]{\IsABT{\AlphaEq{#1}{#2}}}
\newcommand{\NotAlphaEq}[2]{{#1}\not=_\alpha{#2}}

%%% Local Variables: 
%%% mode: latex
%%% TeX-master: "book"
%%% End: 

\newcommand{\FunEvalsTo}[3]{{#1}\mathrel{\Downarrow'}\AbsABT{#2}{#3}}
\newcommand{\FunEvalsToHO}[3]{{#2}\vargen \fappabt{#1}{#2}\evalsto{#3}}

\newcommand{\arrtyabt}[2]{\OpABT{\cd{arr}}{#1;#2}}
\newcommand{\arrtycst}[2]{{#1}\mathrel{\to}{#2}}
\newcommand{\lamabt}[3]{\OpABTp{\cd{lam}}{#1}{\AbsABT{#2}{#3}}}
\newcommand{\lamcst}[3]{\mathop{\lambda}\cdparens{\TightIsOf{#2}{#1}}\,{#3}}
\newcommand{\appabt}[2]{\OpABT{\cd{ap}}{#1;#2}}
\newcommand{\appcst}[2]{{#1}\cdparens{#2}}

\newcommand{\letabt}[4]{\OpABTp{\cd{let}}{#1}{#2;\AbsABT{#3}{#4}}}
\newcommand{\letcst}[4]{\kwop{let}\TightIsOf{#3}{#1}\kwop{be}{#2}\kwop{in}{#4}}
\newcommand{\uletabt}[3]{\kwop{let}{(#2;\AbsABT{#1}{#3})}}
\newcommand{\uletcst}[3]{\kwop{let}{#2}\kwop{be}{#1}\kwop{in}{#3}}

\newcommand{\fdefabt}[6]{\OpABTp{\cd{fun}}{#1;#2}{\AbsABT{#3}{#4};\AbsABT{#5}{#6}}}
\newcommand{\fdefcst}[6]{\kwop{fun}{#5}\TightIsOf{\cdparens{\TightIsOf{#3}{#1}}}{#2}\mathbin{\cd{=}}{#4}\kwop{in}{#6}}
\newcommand{\fappabt}[2]{\OpABTp{\cd{apply}}{#1}{#2}}
\newcommand{\fappcst}[2]{#1\cdparens{#2}}
\newcommand{\fsubst}[3]{\dblsqbracks{#1\mathord{/}#2}{#3}}


%%% Local Variables: 
%%% mode: latex
%%% TeX-master: "book"
%%% End: 

\newcommand{\LangPCF}{\textbsf{PCF}}

\newcommand{\genrecabt}[3]{\OpABTp{\cd{fix}}{#1}{\AbsABT{#2}{#3}}}
\newcommand{\genreccst}[3]{\kwop{fix}{#2}\cdcolon{#1}\kwop{is}{#3}}
\newcommand{\sgenreccst}[2]{\kwop{fix}{#1}\kwop{is}{#2}}

\newcommand{\bddrecabt}[4]{\OpABTp{\cd{fix}^{#1}}{#2}{\AbsABT{#3}{#4}}}
\newcommand{\bddreccst}[4]{\mathop{\cd{fix}^{#1}}{#3}\cdcolon{#2}\kwop{is}{#4}}

\newcommand{\parrtyabt}[2]{\OpABT{\cd{parr}}{#1;#2}}
\newcommand{\parrtycst}[2]{{#1}\mathbin{\rightharpoonup}{#2}}
\newcommand{\pappabt}[2]{\OpABT{\cd{ap}}{#1;#2}}
\newcommand{\pappcst}[2]{\appcst{#1}{#2}}

\newcommand{\funabt}[5]{\OpABTp{\cd{fun}}{#1;#2}{\AbsABT{#3}{\AbsABT{#4}{#5}}}}
\newcommand{\funcst}[5]{\kwop{fun}{#3}\cdparens{{#4}{\cdcolon}{#1}}{\cdcolon}{#2}\kwop{is}{#5}}

\newcommand{\empenv}{\bullet}
\newcommand{\extenv}[3]{{#1},{#2}{=}{#3}}

\newcommand{\expclo}[1]{\widehat{#1}}
\newcommand{\expclois}[2]{\expclo{#1}{=}{#2}}
\newcommand{\expenv}[2]{\widehat{#1}\parens{#2}}
\newcommand{\expenvis}[3]{\expenv{#1}{#2}{=}{#3}}

\newcommand{\cutoff}[2]{{#1}^{(#2)}}

%\newcommand{\Replace}[3]{\sqbracks{#2\mathbin{\leftarrow}#1}{#3}}

\newcommand{\lnattyabt}{\cd{lnat}}
\newcommand{\lnattycst}{\lnattyabt}
\newcommand{\lsuccabt}[2]{\OpABT{\cd{succ}}{\AbsABT{#1}{#2}}}
\newcommand{\lsucccst}[2]{{#1}\kwbin{is}\succcst{#2}}

%%% Local Variables: 
%%% mode: plain-tex
%%% TeX-master: "book"
%%% End: 

\newcommand{\unittyabt}{\kw{unit}}
\newcommand{\unittycst}{\kw{unit}}
\newcommand{\unittycstm}{\top}
\newcommand{\prodtyabt}[2]{\OpABT{\kw{prod}}{#1;#2}}
\newcommand{\prodtycst}[2]{{#1}\times{#2}}

\newcommand{\pprodtycst}[2]{{#1}\otimes{#2}}
\newcommand{\ppairexabt}[2]{\OpABT{\kw{fuse}}{#1;#2}}
\newcommand{\ppairexcst}[2]{{#1}\otimes{#2}}
\newcommand{\splitexabt}[4]{\OpABT{\kw{split}}{{#1};\AbsABT{#2,#3}{#4}}}
\newcommand{\splitexcst}[4]{\kw{split}\,{#1}\,\kw{as}\,\ppairexcst{#2}{#3}\,\kw{in}\,{#4}}
\newcommand{\checkexabt}[2]{\OpABT{\kw{check}}{#1;#2}}
\newcommand{\checkexcst}[2]{\kw{check}\,{#1}\,\kw{as}\,\unitexcst\,\kw{in}\,{#2}}

\newcommand{\unitexabt}{\kw{triv}}
\newcommand{\unitexcst}{\langle\rangle}
\newcommand{\pairexabt}[2]{\OpABT{\kw{pair}}{#1;#2}}
\newcommand{\pairexcst}[2]{\langle #1, #2\rangle}
\newcommand{\projexabt}[2]{\OpABT{\OpInst{\kw{pr}}{#2}}{#1}}
\newcommand{\fstexabt}[1]{\projexabt{#1}{\kw{l}}}
\newcommand{\sndexabt}[1]{\projexabt{#1}{\kw{r}}}
\newcommand{\projexcst}[2]{{#1}\mathbin\cdot{#2}}
\newcommand{\fstexcst}[1]{\projexcst{#1}{\kw{l}}}
\newcommand{\sndexcst}[1]{\projexcst{#1}{\kw{r}}}

\newcommand{\dgenprodcst}[1]{\brackets{#1}}
\newcommand{\dgentuplecst}[1]{\brackets{#1}}
\newcommand{\genprodabt}[3]{\OpABT{\kw{prod}}{\genff{#1}{#2}{#3}}}
\newcommand{\vargenprodcst}[3]{\prod_{{#2}\in{#1}}{#3}}
\newcommand{\genprodcst}[3]{\dgenprodcst{#3}_{#2\in #1}}
\newcommand{\gentupleabt}[3]{\OpABT{\kw{tpl}}{\genff{#1}{#2}{#3}}}
\newcommand{\gentuplecst}[3]{\dgentuplecst{#3}_{{#2}\in{#1}}}
\newcommand{\lgentuplescst}[3]{\gentuplecst{#1}{#2}{\fldexcst{#2}{#3}}}
\newcommand{\genprojabt}[3]{\projexabt{#3}{#2}}
\newcommand{\genprojcst}[3]{\projexcst{#3}{#2}}
\newcommand{\genprodcstgen}[3]{\dgenprodcst{\varexplff{#1}{#2}{#3}}}
\newcommand{\gentuplecstgen}[3]{\dgentuplecst{\varexplff{#1}{#2}{#3}}}

\newcommand{\rcdtycst}[1]{\dgenprodcst{#1}}
\newcommand{\rcdexcst}[1]{\dgentuplecst{#1}}
\newcommand{\fldtycst}[2]{\pairff{#1}{#2}}
\newcommand{\fldexcst}[2]{\pairff{#1}{#2}}
\newcommand{\fldselexcst}[2]{\projexcst{#2}{#1}}


%%% Local Variables: 
%%% mode: latex
%%% TeX-master: "book"
%%% End: 

\newcommand{\casesepcst}{\ensuremath{\mathbin{\cdbar}}}
\newcommand{\casebrcst}{\ensuremath{\mathbin{\hookrightarrow}}}


\newcommand{\voidtyabt}{\kw{void}}
\newcommand{\voidtycst}{\voidtyabt}
\newcommand{\voidtycstm}{\bot}
\newcommand{\abortexcst}[2]{\OpABT{\kw{abort}}{#2}}
\newcommand{\nullcaseexcst}[2]{\abortexcst{#1}{#2}}
\newcommand{\sumtyabt}[2]{\OpABT{\kw{sum}}{#1;#2}}
\newcommand{\sumtycst}[2]{{#1}\mathbin{+}{#2}}
\newcommand{\abortexabt}[2]{\OpABTp{\kw{abort}}{#1}{#2}}
\newcommand{\nullcaseexabt}[2]{\abortexabt{#1}{#2}}
\newcommand{\inexabt}[3]{\OpABTp{\OpInst{\kw{in}}{#2}}{#1}{#3}}
\newcommand{\inexcst}[3]{{#2}\mathbin{\cdot}{#3}}
\newcommand{\inlexabt}[3]{\inexabt{#1;#2}{\kw{l}}{#3}}
\newcommand{\inlexcst}[2]{\inexcst{}{\kw{l}}{#2}}
\newcommand{\varinlexcst}[3]{\inlexcst{}{#3}}
\newcommand{\inrexabt}[3]{\inexabt{#1;#2}{\kw{r}}{#3}}
\newcommand{\inrexcst}[2]{\inexcst{}{\kw{r}}{#2}}
\newcommand{\varinrexcst}[3]{\inrexcst{}{#3}}
\newcommand{\caseexabt}[7]{\OpABT{\kw{case}}{#1;\AbsABT{#2}{#4};\AbsABT{#5}{#7}}}
\newcommand{\caseexcst}[7]{\kwop{case}{#1}\,\cdbraces{\inlexcst{}{#2}\casebrcst{#4}\casesepcst\inrexcst{}{#5}\casebrcst{#7}}}

\newcommand{\dgensumcst}[1]{\cdsqbracks{#1}}
\newcommand{\gensumabt}[3]{\OpABT{\kw{sum}}{\genff{#1}{#2}{#3}}}
\newcommand{\vargensumcst}[3]{\sum_{{#2}\in{#1}}{#3}}
\newcommand{\gensumcst}[3]{\dgensumcst{#3}_{#2\in #1}}
\newcommand{\geninjabt}[3]{\inexabt{#1}{#2}{#3}}
\newcommand{\geninjcst}[3]{\inexcst{}{#2}{#3}}
\newcommand{\gencaseabt}[5]{\OpABT{\kw{case}}{#2;\genff{#1}{#3}{\AbsABT{#4}{#5}}}}
\newcommand{\gencasecst}[5]{\kwop{case}{#2}\,\cdbraces{\geninjcst{#1}{#3}{#4}\casebrcst{#5}}_{{#3}\in{#1}}}
\newcommand{\dgencasecst}[3]{\kwop{case}{#2}\,\cdbraces{#3}}
\newcommand{\gensumcstgen}[3]{\dgensumcst{\varexplff{#1}{#2}{#3}}}
\newcommand{\gencasecstgen}[5]{\dgencasecst{}{#2}{\geninjcst{}{#3_1}{#4_1}\casebrcst{#5_1}\casesepcst\dots\casesepcst\geninjcst{}{#3_{#1}}{#4_{#1}}\casebrcst{#5_{#1}}}}

\newcommand{\booltyabt}{\kw{bool}}
\newcommand{\booltycst}{\booltyabt}
\newcommand{\trexabt}{\kw{true}}
\newcommand{\trexcst}{\trexabt}
\newcommand{\faexabt}{\kw{false}}
\newcommand{\faexcst}{\faexabt}
\newcommand{\ifexabt}[3]{\OpABT{\kw{if}}{#1;#2;#3}}
\newcommand{\ifexcst}[3]{\kwop{if}{#1}\kwop{then}{#2}\kwop{else}{#3}}

\newcommand{\opttyabt}[1]{\OpABT{\kw{opt}}{#1}}
\newcommand{\opttycst}[1]{{#1}\,\kw{opt}}
\newcommand{\noneexabt}{\kw{null}}
\newcommand{\noneexcst}{\noneexabt}
\newcommand{\someexabt}[1]{\OpABT{\kw{just}}{#1}}
\newcommand{\someexcst}[1]{\someexabt{#1}}
\newcommand{\whichexabt}[5]{\OpABTpp{\kw{ifnull}}{#4}{#2;\AbsABT{#3}{#5}}{#1}}
\newcommand{\whichexcst}[5]{\kwop{which}{#1}\,\cdbraces{\noneexcst\casebrcst{#2}\casesepcst\someexcst{#3}\casebrcst{#5}}}

\newcommand{\vartycst}[1]{\dgensumcst{#1}}
\newcommand{\alttycst}[2]{\pairff{#1}{#2}}
\newcommand{\varinjexcst}[2]{\geninjcst{}{#1}{#2}}

\newcommand{\altcst}[4]{\geninjcst{}{#1}{#3}\casebrcst{#4}}
\newcommand{\varcaseexcst}[2]{\kwop{case}{#1}\,\cdbraces{#2}}

\newcommand{\suitty}{\kw{suit}}
\newcommand{\casesuit}[5]{\kwop{case}{#1}\,\cdbraces{\clubsuit\casebrcst{#2}\casesepcst\diamondsuit\casebrcst{#3}\casesepcst\heartsuit\casebrcst{#4}\casesepcst\spadesuit\casebrcst{#5}}}
\newcommand{\casesuitcst}[5]{\varcaseexcst{#1}{\altcst{\kw{clubs}}{#1}{\unittycst}{\_}{#2}}\casesepcst\altcst{\kw{diamonds}}{\unittycst}{\_}{#3}\casesepcst\altcst{\kw{hearts}}{\unittycst}{\_}{#4}\casesepcst\altcst{\kw{spades}}{\unittycst}{\_}{#5}}

%\newcommand{\chartycst}{\kw{char}}
\newcommand{\codech}[1]{\OpABT{\kw{codech}}{#1}}
\newcommand{\chcode}[1]{\OpABT{\kw{chcode}}{#1}}
\newcommand{\chcodetycst}{\kw{chcode}}

\newcommand{\sigtyabt}{\kw{signal}}
\newcommand{\sigtycst}{\kw{signal}}

%%% Local Variables: 
%%% mode: latex
%%% TeX-master: "book"
%%% End: 

\newcommand{\LangM}{\textbsf{M}}    % Mendler's M

\newcommand{\indrecnatcst}[4]{\kw{rec}_{\nattycst}\cdparens{\AbsABT{#2}{#3};{#4}}}
\newcommand{\indinnatcst}[1]{\kw{fold}_{\nattycst}\cdparens{#1}}

\newcommand{\choicecst}{\ensuremath{\mathbin{\cdamper}}}

\newcommand{\streamtycst}{\kw{stream}}
\newcommand{\strgencst}[4]{\kwop{strgen}{#2}\,\kw{is}\,{#1}\,\kw{in}\,\cdptbracks{\pairff{\kw{hd}}{#3}\cdcomma\pairff{\kw{tl}}{#4}}}
\newcommand{\shdcst}[1]{\OpABT{\kw{hd}}{#1}}
\newcommand{\stlcst}[1]{\OpABT{\kw{tl}}{#1}}
\newcommand{\stlncst}[2]{\OpABT{\kw{tl}^{(#1)}}{#2}}
\newcommand{\seqtycst}{\kw{seq}}

\newcommand{\itreetycst}{\kw{itree}}
\newcommand{\iviewcst}[1]{\OpABT{\kw{view}}{#1}}
\newcommand{\itreegencst}[3]{\kwop{itgen}{#2}\,\kw{is}\,{#1}\,\kw{in}\,{#3}}

\newcommand{\coigenstrcst}[3]{\kw{gen}_{\streamtycst}\cdparens{\AbsABT{#1}{#2};{#3}}}
\newcommand{\coioutstrcst}[1]{\kw{unfold}_{\streamtycst}\cdparens{#1}}

\newcommand{\indtyabt}[2]{\OpABT{\kw{ind}}{\AbsABT{#1}{#2}}}
\newcommand{\indtycstg}[1]{\OpABT{\mu}{#1}}
\newcommand{\indtycst}[2]{\indtycstg{\AbsABT{#1}{#2}}}

\newcommand{\indinabt}[3]{\OpABTp{\kw{fold}}{\AbsABT{#1}{#2}}{#3}}
\newcommand{\indincstg}[2]{\OpABT{\kw{fold}_{#1}}{#2}}
\newcommand{\indincst}[3]{\indincstg{\AbsABT{#1}{#2}}{#3}}
\newcommand{\sindincst}[1]{\indincstg{}{#1}} % short form
\newcommand{\indrecabt}[5]{\OpABTp{\kw{rec}}{\AbsABT{#1}{#2}}{\AbsABT{#3}{#4};{#5}}}
\newcommand{\indreccstg}[3]{\OpABT{\kw{rec}_{#1}}{#2;#3}}
\newcommand{\indreccst}[5]{\indreccstg{}{\AbsABT{#3}{#4}}{#5}}

\newcommand{\coityabt}[2]{\OpABT{\kw{coi}}{\AbsABT{#1}{#2}}}
\newcommand{\coitycstg}[1]{\OpABT{\nu}{#1}}
\newcommand{\coitycst}[2]{\coitycstg{\AbsABT{#1}{#2}}}

\newcommand{\coioutabt}[3]{\OpABTp{\kw{unfold}}{\AbsABT{#1}{#2}}{#3}}
\newcommand{\coioutcstg}[2]{\OpABT{\kw{unfold}_{#1}}{#2}}
\newcommand{\coioutcst}[3]{\coioutcstg{\AbsABT{#1}{#2}}{#3}}
\newcommand{\scoioutcst}[1]{\coioutcstg{}{#1}} % short form
\newcommand{\coigenabt}[5]{\OpABTp{\kw{gen}}{\AbsABT{#1}{#2}}{\AbsABT{#3}{#4};{#5}}}
\newcommand{\coigencstg}[3]{\OpABT{\kw{gen}_{#1}}{#2;#3}}
\newcommand{\coigencst}[5]{\coigencstg{}{\AbsABT{#3}{#4}}{#5}}

\newcommand{\conattycst}{\kw{conat}}
\newcommand{\omegaexcst}{\omega}

\newcommand{\naticst}{\nattycst}
\newcommand{\natfcst}{\conattycst}

\newcommand{\listtyabt}[1]{\OpABT{\kw{list}}{#1}}
\newcommand{\listtycst}[1]{{#1}\,\kw{list}}
\newcommand{\natlisttyabt}{\kw{natlist}}
\newcommand{\natlisttycst}{\kw{natlist}}
\newcommand{\nilexabt}{\kw{nil}}
\newcommand{\nilexcst}{\nilexabt}
\newcommand{\consexabt}[2]{\OpABT{\kw{cons}}{#1;#2}}
\newcommand{\consexcst}[2]{\consexabt{#1}{#2}}
\newcommand{\listcaseexabt}[5]{\OpABTp{\kw{case}}{#1}{#2;\AbsABT{#3,#4}{#5}}}
\newcommand{\listcaseexcst}[5]{\kwop{case}{#1}\,\cdbraces{\nilexcst\casebrcst{#2}\casesepcst\consexcst{#3}{#4}\casebrcst{#5}}}
\newcommand{\listrecexabt}[6]{\OpABTp{\kw{rec}}{#1}{#2;#3;\AbsABT{#4;#5}{#6}}}
\newcommand{\listrecexcst}[6]{\kwop{rec}{#2}\,\cdbraces{\nilexcst\casebrcst{#3}\casesepcst\consexcst{#4}{#5}\casebrcst{#6}}}

\newcommand{\sigtwotycst}{\kw{2signal}}

\newcommand{\cozcst}{\tilde{\cd{z}}}
\newcommand{\coscst}{\tilde{\cd{s}}}

%%% Local Variables: 
%%% mode: latex
%%% TeX-master: "book"
%%% End: 

\newcommand{\LangT}{\textbsf{T}}

\newcommand{\nattyabt}{\cd{nat}}
\newcommand{\nattycst}{\nattyabt}

\newcommand{\zeroabt}{\cd{z}}
\newcommand{\zerocst}{\zeroabt}
\newcommand{\succabt}[1]{\OpABT{\cd{s}}{#1}}
\newcommand{\succcst}[1]{\succabt{#1}}

\newcommand{\predabt}[1]{\OpABT{\cd{p}}{#1}}
\newcommand{\predcst}[1]{\predabt{#1}}

\newcommand{\numeral}[1]{\overline{#1}}

\newcommand{\natrecabt}[6]{\OpABTp{\kw{rec}}{#3;\AbsABT{#4}{\AbsABT{#5}{#6}}}{#2}}
\newcommand{\natreccst}[6]{\kwop{rec}{#2}\,\cdbraces{\zerocst\casebrcst{#3}\casesepcst\succcst{#4}\kwop{with}{#5}\casebrcst{#6}}}

\newcommand{\natiterabt}[5]{\OpABTp{\cd{iter}}{#3;\AbsABT{#4}{#5}}{#2}}
\newcommand{\natitercst}[5]{\kwop{iter}{#2}\,\cdbraces{\zerocst\casebrcst{#3}\casesepcst\succcst{#4}\casebrcst{#5}}}

\newcommand{\natcaseabt}[4]{\OpABTp{\cd{ifz}}{#2;\AbsABT{#3}{#4}}{#1}}
\newcommand{\natcasecst}[4]{\kwop{ifz}{#1}\,\cdbraces{\zerocst\casebrcst{#2}\casesepcst\succcst{#3}\casebrcst{#4}}}

\newcommand{\nullstrabt}{\epsilon}
\newcommand{\consstrabt}[2]{{#1}\cdot{#2}}

%%% Local Variables: 
%%% mode: latex
%%% TeX-master: "book"
%%% End: 


% =========================================================================== %
%                                                                             %
%                          Using this LaTeX Template                          %
%                                                                             %
% - new tasks are on their own section (how Gradescope expects them)          %
% - use '\task' to start a new task                                           %
% - use 'begin{task} ... \end{task}' if you'd like to preface your answer     %
%   with the question itself (i.e., fill in the '...' with the question)      %
% - use '\nextgroup' to advance from, for example, Task 1.4 to Task 2.1       %
% - use '\skipaheadtask' to skip from, for example, Task 2.2 to Task 2.4      %
%                                                                             %
% You also have access to all the definitions from the handout. See defs.tex, %
% syn-defns.tex, and generic-defns.tex.                                       %
%                                                                             %
%               TODO: Fill in your personal information below!                %
%                                                                             %
% =========================================================================== %
\newcommand{\myname}{Andrew Carnegie}
\newcommand{\myandrewid}{andrew}
\newcommand{\hwnumber}{1}
% =========================================================================== %

\newcounter{group}
\setcounter{group}{1}
\newtheorem{theorem}{Task}[group]
% Remove '\newpage' below to preview your doc compactly.
% Remember to put it back before submitting to Gradescope.
\newcommand{\task}{\newpage\begin{theorem}\end{theorem}}
\newcommand{\nextgroup}{\stepcounter{group}}
\newcommand{\skipaheadtask}{\stepcounter{theorem}}
\newcommand{\ms}[1]{\ensuremath{\mathsf{#1}}}
\newcommand{\irl}[1]{\mathtt{#1}}
\newcounter{rule}
\setcounter{rule}{0}
\newcommand{\rn}
  {\addtocounter{rule}{1}(\arabic{rule})}

\newcounter{infercount}
\setcounter{infercount}{1}
\newcommand{\infern}[2]{\inferrule{#1}{#2}(\text{S}_{\arabic{infercount}}\stepcounter{infercount})}
\newcommand*\ts[2]{%
  \,\scalebox{1}[0.5]{$\sststile[ss]{\textstyle#1}{\textstyle#2}$}\,
}
\newcommand{\inferr}[2]{\inferrule{#2}{#1}}
\newcommand{\inferrr}[3]{\inferrule[#1]{#2}{#3}}
\newcommand{\paircaseabt}[4]{\irl{case}(#2,#3.#4)}
\newcommand{\paircasecst}[4]{\irl{case} \; #1\; \{(#2;#3) \hookrightarrow #4\}}
\newcommand{\na}[1]{\mathsf{no\_alias}(#1)}
\newtheorem{lemma}[theorem]{Lemma}

\title{15-312 Assignment \hwnumber}
\author{\myname\\(\myandrewid)}
\date{\today}

\begin{document}
\maketitle

\[
\begin{array}{r l l l l}
\ms{Type} & \tau \,\,\,\,\, ::= \\
	& \irl{nat}                	 			& \irl{nat}											& \text{naturals}\\
	& \unittyabt                	 			& \unittycst										& \text{unit}\\
  & \booltyabt                       & \booltycst                    & \text{boolean}\\
  & \prodtyabt{\tau_1}{\tau_2}       & \prodtycst{\tau_1}{\tau_2}    & \text{product}\\
	&\irl{arr}(\tau_1;\tau_2) 				& \arrtycst{\tau_1}{\tau_2} 									& \text{function}\\
  &\listtyabt{\tau}		& \listtycst{\tau}											& \text{list}\\
	 \\
\ms{Exp}
        & e   \,\,\,\,\, ::= \\
 	& x                                			& x 												& \text{variable}\\
  & \irl{nat}[n]							& \numeral{n}												& \text{number}\\
  & \irl{unit}							& ()												& \text{unit}\\
  & \irl{T}							& \irl{T}												& \text{true}\\
  & \irl{F}	   					& \irl{F}												& \text{false}\\
  & \ifexabt{x}{e_1}{e_2} & \ifexcst{x}{e_1}{e_2}  & \text{if}\\
  & \irl{lam}(x:\tau.e) 						&\lambda \; x : \tau. e 		& \text{abstraction}\\
  & \irl{ap}(f;x) 					& \appcst{f}{x} 										& \text{application}\\
  & \irl{tpl}(x_1;x_2)     	& \pairexcst{x_1}{x_2}                									& \text{pair}\\
 	& \paircaseabt{p}{x_1}{x_2}{e_1}					& \paircasecst{p}{x_1}{x_2}{e_1}   	& \text{match pair}\\
 	& \nilexabt					& []   										& \text{nil}\\
 	& \consexabt{x_1}{x_2}					& x_1::x_2   										& \text{cons}\\
 	& \listcaseexabt{l}{e_1}{x}{xs}{e_2}					& \listcaseexcst{l}{e_1}{x}{xs}{e_2}   	& \text{match list}\\
  & \irl{let}(e_1; x : \tau.e_2)			& \irl{let}\; x = e_1 \; \irl{in}\; e_2   	& \text{let}\\
  \\
\ms{Val}
        & v   \,\,\,\,\, ::= \\
 	& \irl{val}(n)                                			& n 												& \text{numeric value}\\
 	& \irl{val}(\irl{T})                               			& \irl{T} 								  & \text{true value}\\
 	& \irl{val}(\irl{F})                                			& \irl{F}								  & \text{false value}\\
 	& \irl{val}(\irl{Null})                                  & \irl{Null} 								  & \text{null value}\\
 	& \irl{val}(\irl{cl}(V; x.e))                & (V, x.e) 					& \text{function value}\\
 	& \irl{val}(l)                                			& l 								  & \text{loc value}\\
 	& \irl{val}(\pairexabt{v_1}{v_2})                             & \pairexcst{v_1}{v_2} 								  & \text{pair value}\\
  \\
\ms{State} & s   \,\,\,\,\, ::= \\
 	& \irl{alive}                                			& \irl{alive} 												& \text{live value}\\
 	& \irl{dead}                                			& \irl{dead} 												& \text{dead value}\\\\
\ms{Loc} & l   \,\,\,\,\, ::= \\
 	& \irl{loc}(l)                                			& l 												& \text{location}\\\\
\ms{Var} & l   \,\,\,\,\, ::= \\
 	& \irl{var}(x)                                			& x 												& \text{variable}\\
\end{array}
\]

\section{Garbage collection semantics}

Model dynamics using judgement of the form:
\[
\fbox{$V,H,R,F \; \vdash e \Downarrow v, H', F'$}
\]

\noindent
Where $V : \ms{Var} \to \ms{Val} \times \ms{State}$, $H : \ms{Loc} \to \ms{Val}$, $R \subseteq \ms{Loc}$, and $F \subseteq \ms{Loc}$. This can be read as: under stack $V$, heap $H$, roots $R$,
freelist $F$, the expression $e$ evaluates to $v$, and engenders a new heap $H'$ and freelist $F'$.\\

\noindent
Note that the stack maps each variable to a value $v$ \emph{and} a state $s$. If $s$ is \irl{alive}, then $v$ can still be used, while $\irl{dead}$ indicates that $v$ is already used and cannot be used again. We write $\overline V = \{x \in V \mid V(x) = (\_,\irl{alive}) \}$ for the variables in $V$ that are alive.\\ 

\noindent
Roots represents the set of locations required to compute the continuation \emph{excluding} the current expression.
We can think of roots as the heap allocations necessary to compute the context with a hole that will be filled
by the current expression.\\


\noindent
Below defines the size of reachable values and space for roots:

\begin{align*}
%   &reach_{H}((V, x.e)) = \bigcup\limits_{y \in FV(e) \setminus x} reach_H(V(y)) \\
%   &reach_{H}(l) = \{l\} \cup reach_{H}(H(l))\\
%   &reach_{H}(\pairexcst{v_1}{v_2}) = reach_{H}(v_1) \cup reach_{H}(v_2)\\
%   &reach_{H}(\_) = \emptyset\\\\
%   &locs_{V,H}(e) = \bigcup\limits_{x \in FV(e)} reach_H(V(x))\\\\
  &locs_{V,H}(e) = \bigcup\limits_{x \in FV(e)} \{l \in H \mid \exists l' \in root(x). H \vDash p : l' \leadsto l\}\\\\
  &size(\pairexcst{v_1}{v_2}) = size(v_1) + size(v_2)\\
  &size(\_) = 1\\\\
  &copy(H,L,\pairexcst{v_1}{v_2}) =\\
  &\quad\mathsf{let }\; L_1 \subseteq L \mathsf{ with }\; |L_1| = size(v_1) \;\mathsf{ in}\\
  &\quad\mathsf{let }\; H_1,\_ = copy(H,L_1,v_1) \;\mathsf{ in} \\
  &\quad copy(H_1,L \setminus L_1, v_2)\\
  &copy(H,l,v) = H[l \mapsto v], l\\
\end{align*}

\begin{mathpar}

\infern
{ x \in dom(V)
}
{V,H,R,F \; \vdash x \Downarrow V(x),H,F}

\infern
{
}{
  V,H,R,F \; \vdash \numeral{n} \Downarrow \irl{val}(n),H,F
}

\infern{
}{
  V,H,R,F \; \vdash \irl{T} \Downarrow \irl{val(T)},H,F
}

\infern{
}{
  V,H,R,F \; \vdash \irl{F} \Downarrow \irl{val(F)},H ,F
}

\infern{
}{
  V,H,R,F \; \vdash () \Downarrow \irl{val(Null)},H ,F
}

\infern{
  V(x) = \irl{T}\\
  g = \{l \in H \mid l \notin F \cup R \cup locs_{V,H}(e_1)\}\\
  V,H,R,F \cup g\; \vdash e_1 \Downarrow v, H',F'
}{
  V,H,R,F \; \vdash \ifexabt{x}{e_1}{e_2} \Downarrow v, H',F'
}

\infern{
  V(x) = \irl{F}\\
  g = \{l \in H \mid l \notin F \cup R \cup locs_{V,H}(e_2)\}\\
  V,H,R,F \cup g \; \vdash e_2 \Downarrow v, H',F'
}{
  V,H,R,F \; \vdash \ifexabt{x}{e_1}{e_2} \Downarrow v, H' ,F'
}

% function

\infern{
  l \in F\\
  F' = F \setminus \{l\}\\
  H' = H[l \mapsto (V,x.e)]
}{
  V,H,R,F \; \vdash \irl{lam}(x : \tau.e) \Downarrow l,H' ,F'
}

\infern{
  V(f) = (V_1, x.e) \\
  V(x) = v_1 \\
  V_1[x \mapsto v_1], H, R \; \vdash e \Downarrow v, H',F'
}{
  V,H,R,F \; \vdash \appcst{f}{x} \Downarrow v,H',F'
}

% tuples

\infern{
  V(x_1) = v_1 \\
  V(x_2) = v_2 \\
}{
  V,H,R,F \; \vdash \pairexcst{x_1}{x_2} \Downarrow \pairexcst{v_1}{v_2},H ,F
}

\infern{
  V(x) = \pairexcst{v_1}{v_2}\\
  g = \{l \in H \mid l \notin F \cup R \cup locs_{V,H}(e)\}\\
  V[x_1 \mapsto v_1, x_2 \mapsto v_2],H,R,F \cup g \; \vdash e \Downarrow v,H',F'
}{
  V,H,R,F \; \vdash \paircasecst{x}{x_1}{x_2}{e} \Downarrow v, H' ,F'
}

% lists

\infern{
}{
  V,H,R,F \; \vdash \nilexabt \Downarrow \irl{val(Null)},H,F
}

\infern{
  v = \pairexcst{V(x_1)}{V(x_2)}\\
  L \subseteq F\\
  |L| = size_H(v)\\
  F' = F \setminus L\\
  H',l = copy(H,L,v)\\
}{
  V,H,R,F \; \vdash \consexcst{x_1}{x_2} \Downarrow l,H' ,F'
}

\infern{
  V(x) = \irl{Null}\\
  g = \{l \in H \mid l \notin F \cup R \cup locs_{V',H}(e_1)\}\\
  V,H,R,F \cup g \; \vdash e_1 \Downarrow v, H',F' \\
}{
  V,H,R,F \; \vdash \listcaseexcst{x}{e_1}{x_h}{x_t}{e_2} \Downarrow v,H',F'
}

\infern{
  V(x) = (l,\irl{alive}) \\
  H(l) = \pairexcst{v_h}{v_t} \\
  V' = V\{x \mapsto (l,\irl{dead})\}\\
  V'' = V'[x_h \mapsto (v_h,\irl{alive}), x_t \mapsto (v_t,\irl{alive})]\\
  g = \{l \in H \mid l \notin F \cup R \cup locs_{V'',H}(e_2)\}\\
  V'',H,R,F \cup g \; \vdash e_2 \Downarrow v, H',F' \\
}{
  V,H,R,F \; \vdash \listcaseexcst{x}{e_1}{x_h}{x_t}{e_2} \Downarrow v,H',F'
}

\infern{
  R' = R \cup locs_{V,H}(\irl{lam}(x : \tau.e_2))\\
  V,H,R',F \vdash e_1 \Downarrow v_1,H_1,F_1\\
  V' = V[x \mapsto v_1]\\
  R'' = R \cup locs_{V',H_1}(e_2)\\
  g = \{ l \in H_1 \mid l \notin R'' \cup F_1 \}\\
  V',H_1,R, F_1 \cup g \vdash e_2 \Downarrow v_2,H_2,F_2 \\
}{
  V,H,R,F \; \vdash \irl{let}(e_1; x : \tau.e_2) \Downarrow v_2,H_2,F_2
}
\end{mathpar}
\section{Operational semantics}
In order to prove the soundess of the type system, we also define a simplified operational semantics that does not account for garbage collection. 

\[
\fbox{$V,H \vdash e \Downarrow v, H'$}
\]

This can be read as: under stack $V$, heap $H$ the expression $e$ evaluates to $v$, and engenders a new heap $H'$. We write the representative rules.

\begin{mathpar}
\infern{
  v = \pairexcst{V(x_1)}{V(x_2)}\\
  H',l = copy(H,L,v)\\
}{
  V,H \; \vdash \consexcst{x_1}{x_2} \Downarrow l,H'
}

\infern{
  V(x) = (l,\irl{alive}) \\
  H(l) = \pairexcst{v_h}{v_t} \\
  V' = V\{x \mapsto (l,\irl{dead})\}\\
  V'' = V'[x_h \mapsto (v_h,\irl{alive}), x_t \mapsto (v_t,\irl{alive})]\\
  V'',H \; \vdash e_2 \Downarrow v, H' \\
}{
  V,H \; \vdash \listcaseexcst{x}{e_1}{x_h}{x_t}{e_2} \Downarrow v,H'
}

\infern{
  V,H \vdash e_1 \Downarrow v_1,H_1\\
  V' = V[x \mapsto v_1]\\
  V',H_1\vdash e_2 \Downarrow v_2,H_2\\
}{
  V,H \; \vdash \irl{let}(e_1; x : \tau.e_2) \Downarrow v_2,H_2
}
\end{mathpar}

\section{Type rules}

The type system takes into account of garbaged collected cells by returning potential locally in a match construct. Since we are interested in the number of heap cells, all constants are assumed to be nonnegative.

\begin{mathpar}
\inferr{
  \Sigma; \emptyset \sststile{q}{q} n : \irl{nat}
}{
  n \in \mathbb{Z}
}(\text{L:ConstI})

\inferr{
  \Sigma; \emptyset \sststile{q}{q} () : \irl{unit}
}{
}(\text{L:ConstU})

\inferr{
  \Sigma; \emptyset \sststile{q}{q} \irl{T} : \irl{bool}
}{
}(\text{L:ConstT})

\inferr{
  \Sigma; \emptyset \sststile{q}{q} \irl{F} : \irl{bool}
}{
}(\text{L:ConstF})

\inferr{
  \Sigma; x : B \sststile{q}{q} x : B
}{
}(\text{L:Var})

\inferr{
  \Sigma; \Gamma, x : \irl{bool} \sststile{q'}{q} \ifexcst{x}{e_t}{e_f} : B
}{
  \Sigma; \Gamma \sststile{q'}{q} e_t : B \\
  \Sigma; \Gamma \sststile{q'}{q} e_f : B
}(\text{L:Cond})

\inferr{
  \Sigma; x_1 : A_1, x_2 : A_2 \sststile{q}{q} \pairexcst{x_1}{x_2} : \prodtycst{A_1}{A_2}
}{
}(\text{L:Pair})

\inferr{
  \Sigma; \Gamma, x : (A_1,A_2) \sststile{q'}{q} \paircasecst{x}{x_1}{x_2}{e} : B
}{
  \Sigma; \Gamma, x_1 : A_1, x_2 : A_2 \sststile{q'}{q} e : B
}(\text{L:MatP})

\inferr{
  \Sigma; \emptyset \sststile{q}{q} \irl{nil} : L^p(A)
}{
}(\text{L:Nil})

\inferr{
  \Sigma; \Gamma, x_h : A, x_t : L^p(A) \sststile{q}{q+p+1} \consexcst{x_h}{x_t} : L^p(A)
}{
}(\text{L:Cons})

\inferr{
  \Sigma; \Gamma, x : L^p(A) \sststile{q'}{q} \listcaseexcst{x}{e_1}{x_h}{x_t}{e_2} : B
}{
  \Sigma; \Gamma \sststile{q'}{q} e_1 : B \\
  \Sigma; \Gamma, x_h : A, x_t : L^p(A) \sststile{q'}{q + p + 1} e_2 : B
}(\text{L:MatL})

\inferr{
  \Sigma; \Gamma_1, \Gamma_2 \sststile{q'}{q} \irl{let}(e_1; x : \tau.e_2) : B
}{
  \Sigma; \Gamma_1 \sststile{p}{q} e_1 : A \\
  \Sigma; \Gamma_2, x : A \sststile{q'}{p} e_2 : B
}(\text{L:Let})
\end{mathpar}

Now if we take $\dagger :  L^p(A) \mapsto L(A)$ as the map that erases resource annotations, 
 we obtain a simpler typing judgement $\fbox{\Sigma^{\dagger}; \Gamma^{\dagger} \vdash e : B^{\dagger}}$.

\section{Paths and aliasing}

In order prove soundness of the type system, we need some auxiliary judgements to defining properties of a heap. Below we define $root : Val \to \{\{Loc\}\}$ that maps stack values its the root \emph{multiset}, the multiset of locations that's already on the stack. 
\begin{align*}
&root(\pairexcst{v_1}{v_2}) = root(v_1) \uplus root(v_2)\\
&root(l) = \{l\}\\
&root(\_) = \emptyset
\end{align*}
For a multiset $S$, we write $\mu : S \to \mathbb{N}^+$ for the multiplicity function of $S$, which maps each element to the count of its occurence. If $\forall s \in S. \mu(s) = 1$, then $S$ is a property set, and we denote it by $\ms{set}(S)$.

Next, we define the judgements \fbox{H \vDash p : l \leadsto l'} for path formuation and \fbox{H \vDash p = p' : l \leadsto l'} for path equality. A path can be thought of as a sequence of locations that is traversable by following pointers in the heap.\\

\fbox{H \vDash p : l \leadsto l'}
\begin{mathpar}
\inferr{
  H \vDash id_l : l \leadsto l
}{
  l \in H
}(\text{Id})

\inferr{
  H \vDash (l,l') : l \leadsto l'
}{
  H(l) = v \\
  l' \in root(v)\\
  l' \in H
}(\text{Edge})

\inferr{
  H \vDash q \circ p : l \leadsto l''
}{
  H \vDash p : l \leadsto l'\\
  H \vDash q : l' \leadsto l''
}(\text{Comp})
\end{mathpar}

\fbox{H \vDash p = p' : l \leadsto l'}
\begin{mathpar}
\inferr{
  H \vDash p  \circ id_l \equiv p : l \leadsto l'
}{
  H \vDash p : l \leadsto l'
}(\text{LeftID})

\inferr{
  H \vDash id_{l'} \circ p \equiv p : l \leadsto l'
}{
  H \vDash p : l \leadsto l'
}(\text{RightID})

\inferr{
  H \vDash p \circ (l,l') \equiv q \circ (l,l') : l \leadsto l''
}{
  H(l) = v\\
  l' \in root(v) \\
  l' \in H
  H \vDash p \equiv q : l' \leadsto l''
}(\text{Eq})
\end{mathpar}
Note that it is \emph{not} the case that $id_l \equiv (l,l) : l \leadsto l$, since the former is an actual identity, while the latter is an infinite loop in the heap: $H(l) = l$.  

Next, we define the predicates $\ms{forest}(H)$ and $\ms{no\_alias}$: 
\begin{description}
\item [$\ms{forest}(H)$: ]\\
$\forall l,l_1,l_2 \in H$, if $H \vDash p : l_1 \leadsto l$ and $H \vDash q : l_2 \leadsto l$, then $l_1 = l_2$ and $H \vDash p \equiv q : l_1 \leadsto l$.
\item[$\na{V}$: ] \\
$\forall x,y \in \overline V$, x \ne y. \text{ Let } $r_x = root(\overline V(x))$, $r_y = root(\overline V(y))$. Then:
\begin{enumerate}[(1)]
\item $\ms{set}(r_x), \ms{set}(r_y)$
\item $r_x \cap r_y = \emptyset$
\end{enumerate}
\end{description}
If the induced graph of heap $H$ is a forest, then it is a disjoint union of arborescences (directed trees), and there is at most one path from one loaction in $H$ to another by following the pointers.
\section{Soundness for garbage collection semantics}

We simplify the soundness proof of the type system for the general metric to one with monotonic resource.
(No function types for now)

\begin{lemma}
\label{a} If $\Sigma; \Gamma \sststile{q'}{q} e : B$, then $\Sigma^{\dagger}; \Gamma^{\dagger} \sststile{q'}{q} e : B^{\dagger}$.
\end{lemma}
\begin{lemma}
For all stacks $V$ and heaps $H$, if $\na{V}$, $\ms{forest}(H)$, $\Sigma^{\dagger}; \Gamma^{\dagger} \sststile{q'}{q} e : B^{\dagger}$, $F \cap R = \emptyset$, $H \vDash V : \Gamma$, and $V,H,R,F \; \vdash e \Downarrow v, H', F'$, then $F' \cap R = \emptyset$ and $\ms{forest}(H')$.
\end{lemma}
\begin{theorem}[Soundness]
\label{a} let $H \vDash V : \Gamma$, $\Sigma; \Gamma \sststile{q'}{q} e : B$,
and $V,H \; \vdash e \Downarrow v, H'$. Then $\forall C \in \mathbb{Q}^{+}$ and $\forall F \subseteq \ms{Loc}$ with $|F| \ge \Phi_{V,H}(\Gamma) + q + C$,  if $\na{V}$, $R \cap locs_{V,H}(e) = \emptyset$, \text{ and } $F \cap locs_{V,H}(e) = \emptyset$, then there exists $F' \subseteq \ms{Loc}$ s.t.
\begin{enumerate}
  \item $V,H,R,F \vdash e \Downarrow v, H', F'$
  \item $|F'| \ge \Phi_{H'}(v:B) + q' + C$
\end{enumerate}
\end{theorem}

\begin{proof}
Induction on the evaluation judgement.\\
\begin{description}
  \item[Case 1: E:Var]
  \begin{align}
  &V,H,R,F \; \vdash x \Downarrow V(x),H,F \tag{admissibility}\\
  &\Sigma; x : B \sststile{q}{q} x : B \tag{admissibility}\\
  &|F| - |F'|\\
  &\quad = |F| - |F| \tag{ad.}\\
  &\quad = 0\\
  &\Phi_{V,H}(\Gamma) + q - (\Phi_{H'}(v:B) + q')\\
  &\quad = \Phi_{V,H}(x : B) + q - (\Phi_{H}(V(x) : B) + q) \tag{ad.}\\
  &\quad = \Phi_{H}(V(x) : B) + q  - (\Phi_{H}(V(x) : B) + q) \tag{def. of $\Phi_{V,H}$}\\
  &\quad = 0\\
  &|F| - |F'| \le \Phi_{V,H}(\Gamma) +q - (\Phi_{H'}(v:B) + q') \tag{(3),(5)}
  \end{align}
  \item[Case 2: E:Const*]
  Due to similarity, we show only for E:ConstI
  \begin{align*}
  &|F| - |F'| = |F| - |F| \tag{ad.}\\
  &\quad = 0\\
  &\Phi_{V,H}(\Gamma) +q - (\Phi_{H'}(v:B) + q') = \Phi_{V,H}(\emptyset) +q - (\Phi_{H}(v:int) + q) \tag{ad.}\\
  &\quad = 0 \tag{def of $\Phi_{V,H}$}\\
  &|F| - |F'| \le \Phi_{V,H}(\Gamma) +q - (\Phi_{H'}(v:B) + q')
  \end{align*}
  \item[Case 4: E:App]
  \item[Case 5: E:CondT]
  \begin{align*}
  &\Gamma = \Gamma', x : \irl{bool} \tag{ad.}\\
  &H \vDash V : \Gamma' \tag{def of W.F.E}\\
  &\Sigma; \Gamma' \sststile{q'}{q} e_t : B \tag{ad.}\\
  &V,H,R,F \cup g\; \vdash e_t \Downarrow v, H',F' \tag{ad.}\\
  &|F \cup g| - |F'| \le \Phi_{V,H}(\Gamma) +q - (\Phi_{H'}(v:B) + q') \tag{IH}\\
  &|F| - |F'| \le \Phi_{V,H}(\Gamma) +q - (\Phi_{H'}(v:B) + q') \\
  \end{align*}
  \item[Case 6: E:CondF] 
  Similar to E:CondT
  \item[Case 7: E:Let]
  \begin{align*}
  &V,H,R',F \vdash e_1 \Downarrow v_1,H_1,F_1 \tag{ad.}\\
  &\Sigma; \Gamma_1 \sststile{p}{q} e_1 : A \tag{ad.}\\
  &H \vDash V : \Gamma_1 \tag{$\Gamma_1 \subseteq \Gamma$}\\
  &|F| - |F_1| \le \Phi_{V,H}(\Gamma_1) +q - (\Phi_{H_1}(v_1:A) + p) \tag{IH}\\
  &V',H_1,R, F_1 \cup g \vdash e_2 \Downarrow v_2,H_2,F_2 \tag{ad.}\\
  &\Sigma; \Gamma_2, x : A \sststile{q'}{p} e_2 : B \tag{ad.}\\
  &H_1 \vDash v_1 : A \text{ and } \tag{Theorem 3.3.4}\\ 
  &H_1 \vDash V : \Gamma_2 \tag{???}\\
  &H_1 \vDash V' : \Gamma_2, x : A \tag{def of $\vDash$}\\
  &|F_1 \cup g| - |F_2| \le  \Phi_{V',H_1}(\Gamma_2, x:A) +p - (\Phi_{H_2}(v_2:B) + q') \tag{IH}\\
  &|F_1| - |F_2| \le \Phi_{V',H_1}(\Gamma_2,x:A) + p - (\Phi_{H_2}(v_2:B) + q')\\
  &\text{summing the inequalities:}\\
  &|F| - |F_1| + |F_1| - |F_2| \le \Phi_{V,H}(\Gamma_1) +q - (\Phi_{H_1}(v_1:A) + p) + \Phi_{V',H_1}(\Gamma_2,x:A) + p - (\Phi_{H_2}(v_2:B) + q')\\
  &|F| - |F_2| \le \Phi_{V,H}(\Gamma_1) +q - \Phi_{H_1}(v_1:A) + \Phi_{V',H_1}(\Gamma_2,x:A) - (\Phi_{H_2}(v_2:B) + q')\\
  &\quad = \Phi_{V,H}(\Gamma_1) + \Phi_{V',H_1}(\Gamma_2) + q - \Phi_{H_1}(v_1:A) + \Phi_{V',H_1}(x:A) - (\Phi_{H_2}(v_2:B) + q') \tag{def of $\Phi_{V,H}$}\\nn 
  &\quad = \Phi_{V,H}(\Gamma_1) + \Phi_{V,H}(\Gamma_2) + q - \Phi_{H_1}(v_1:A) + \Phi_{V',H_1}(x:A) - (\Phi_{H_2}(v_2:B) + q') \tag{Lemma 4.3.3}\\
  &\quad = \Phi_{V,H}(\Gamma) + q - \Phi_{H_1}(v_1:A) + \Phi_{H_1}(v_1:A) - (\Phi_{H_2}(v_2:B) + q') \tag{def of $\Phi_{V,H}$}\\
  &\quad = \Phi_{V,H}(\Gamma) + q - (\Phi_{H_2}(v_2:B) + q') \\
  \end{align*}
  \item[Case 8: E:Pair]
  Similar to E:Const*
  \item[Case 9: E:MatP]
  Similar to E:MatCons
  \item[Case 10: E:Nil]
  Similar to E:Const*
  \item[Case 11: E:Cons]
  \begin{align*}
  &|F| - |F'|\\
  &\quad = |F| - |F \setminus \{l\}| \tag{ad.}\\
  &\quad = 1\\
  &\Phi_{V,H}(\Gamma) +q - (\Phi_{H'}(v:B) + q')\\
  &\quad = \Phi_{V,H}(x_h : A,x_t:L^p(A)) + q + p + 1 - (\Phi_{H'}(v:L^p(A)) + q) \tag{ad.}\\
  &\quad = \Phi_{V,H}(x_h : A,x_t:L^p(A)) + p + 1 - \Phi_{H'}(v:L^p(A))) \\
  &\quad = \Phi_{H}(V(x_h):A) + \Phi_{H}(V(x_t):L^p(A)) + p + 1 - \Phi_{H'}(v:L^p(A))) \tag{def of $\Phi_{V,H}$}\\
  &\quad = \Phi_{H}(v_h:A) + \Phi_{H}(v_t:L^p(A)) + p + 1 - \Phi_{H'}(v:L^p(A))) \tag{ad.}\\
  &\quad = \Phi_{H}(v_h:A) + \Phi_{H}(v_t:L^p(A)) + p + 1 - (p + \Phi_{H'}(v_h:A) + \Phi_{H'}(v_t:L^p(A))) \tag{Lemma 4.1.1}\\
  &\quad = \Phi_{H}(v_h:A) + \Phi_{H}(v_t:L^p(A)) + p + 1 - (p + \Phi_{H}(v_h:A) + \Phi_{H}(v_t:L^p(A))) \tag{Lemma 4.3.3}\\
  &\quad = 1\\
  &\text{Hence,}\\
  &|F| - |F'| \le \Phi_{V,H}(\Gamma) +q - (\Phi_{H'}(v:B) + q')
  \end{align*}
  \item[Case 12: E:MatNil]
  Similar to E:Cond*
  \item[Case 13: E:MatCons]
  \begin{align*}
  &V(x) = (l,\irl{alive}) \tag{ad.}\\
  &H(l) = \pairexcst{v_h}{v_t} \tag{ad.}\\
  &\Gamma = \Gamma', x : L^p(A) \tag{ad.}\\
  &\Sigma; \Gamma', x_h : A, x_t : L^p(A) \sststile{q'}{q + p + 1} e_2 : B \tag{ad.}\\
  &V'',H,R,F \cup g \; \vdash e_2 \Downarrow v_2, H_2, F' \tag{ad.} \\
  &H \vDash V(x) : L^p(A) \tag{def of W.D.E}\\
  &H''\vDash v_h : A,\; H'' \vDash v_t : L^p(A) \tag{ad.}\\
  &H\vDash v_h : A,\; H \vDash v_t : L^p(A) \tag{???}\\
  &H \vDash V'' : \Gamma', x_h : A, x_t : L^p(A) \tag{def of W.D.E}\\
  &\text{Suppose }  \na{V}{H}, R \cap locs_{V,H}(e) = \emptyset, \text{ and } F \cap locs_{V,H}(e) = \emptyset\\
  &\text{NTS }  |F| - |F'| \le \Phi_{V,H}(\Gamma) +q - (\Phi_{H'}(v:B) + q') \text{ and } \na{V}{H'}\\
  &\text{WTS } \na{V''}{H}\\
  &\text{let } l \in H \text{ arbitrary }, y, z \in \overline V'' \text{ arbitrary },  r_y = root(\overline V''(y)), r_z = root(\overline V''(z))\\
  &\quad\textbf{case: } y \notin \{x_h,x_t\}, z \notin \{x_h,x_t\}\\
  &\quad y,z \in \overline V \tag{def of $V''$}\\
  &\quad (1) - (3) \text{ holds } \tag{Sp.}\\
  &\quad\textbf{case: } y = x_h, z \notin \{x_h,x_t\}\\
  &\quad \ms{set}(root(\pairexcst{v_h}{v_t})) \tag{Sp.}\\
  &\quad \ms{set}(root(v_h)) \tag{def of $\ms{set}$}\\
  &\quad \ms{set}(r_y) \tag{def of $V''$}\\
  &\quad z \in \overline V \tag{def of $V''$}\\
  &\quad \ms{set}(r_z) \tag{Sp.}\\
  &\quad \text{hence we have (1)}\\
  &\quad \text{Suppose } l' \in r_y \cap r_z\\
  &\quad l' \in H \tag{$H \vDash V'' :  \Gamma', x_h : A, x_t : L^p(A)$}\\
  &\quad H \vDash id_{l'} : l' \leadsto l' \tag{Id}\\
  &\quad H \vDash (l,l') : l \leadsto l' \tag{Edge}\\
  &\quad H \vDash id_{l'} \equiv (l,l') : l' \leadsto l' \tag{$\ms{linear}_H(r_x,r_z)$}\\
  &\quad \text{contradiction, hence } r_y \cap r_z = \emptyset,  \tag{hence we have (2)}\\
  &\quad \text{let } l' \in H \text{ arbitrary }, l_1,l_2 \in r_y \tag{ arbitrary }\\
  &\quad \text{suppose } H \vDash p : l_1 \leadsto l', H \vDash q : l_2 \leadsto l'\\
  &\quad H \vDash (l,l_1) : l \leadsto l_1 \text{ and }  H \vDash (l,l_2) : l \leadsto l_2 \tag{Edge}\\
  &\quad H \vDash p \circ (l,l_1) : l \leadsto l' \text{ and }  H \vDash q \circ (l,l_2) : l \leadsto l' \tag{Comp}\\
  &\quad H \vDash p \circ (l,l_1) \equiv q \circ (l,l_2) : l \leadsto l' \tag{$\ms{linear}_H(r_x,r_x)$}\\
  &\quad H \vDash p \equiv q : l_1 \leadsto l' \tag{inversion on Eq}\\
  &\quad \text{hence we have } \ms{linear}_H(r_y,r_y)\\
  &\quad \ms{linear}_H(r_z,r_z) \tag{Sp.}\\
  &\quad \text{let } l' \in H \text{ arbitrary }, l_1 \in r_y, l_2 \in r_z \tag{ arbitrary }\\
  &\quad \text{suppose } H \vDash p : l_1 \leadsto l', H \vDash q : l_2 \leadsto l'\\
  &\quad H \vDash (l,l_1) : l \leadsto l_1 \tag{Edge}\\
  &\quad H \vDash p \circ (l,l_1) : l \leadsto l' \tag{Comp}\\
  &\quad l = l_2 \tag{$\ms{linear}_H(r_x,r_z)$}\\
  &\quad \text{contradiction since } r_x \cap r_z = \emptyset\\
  &\quad \text{hence we have } \ms{linear}_H(r_y,r_z)\\
  &\quad \text{hence we have (3)}\\
  &\quad\textbf{case: } y = x_t, z \notin \{x_h,x_t\}\\
  &\quad\textbf{case: } y = \notin \{x_h,x_t\}, z = x_h\\
  &\quad\textbf{case: } y = \notin \{x_h,x_t\}, z = x_t\\
  &\quad \text{all symmetric to previous case}\\
  &\quad\textbf{case: } y = x_h, z = x_t\\
  &\quad \text{we get (1) the same way as the previous case}\\
  &\quad \ms{set}(root(\pairexcst{v_h}{v_t})) \tag{(1)}\\
  &\quad \ms{set}(root(v_h) \uplus root(v_t)) \tag{def of $root$}\\
  &\quad root(v_h) \cap root(v_t) = \emptyset \tag{def of $\ms{set}$}\\
  &\quad r_y \cap r_z = \emptyset \tag{def of $r_y,r_z$}\\
  &\quad \text{we get (3) the same way as the previous case}\\
  &\quad \text{hence we have } \na{V''}{H}\\
  &\text{let } l' \in locs_{V'',H}(e_2) \text{ arbitrary }\\
  &\quad \exists! x' \in \overline V''. \exists! l'' \in root(\overline V''(x')). H \vDash p : l'' \leadsto l' \tag{def of $locs_{V,H}$}\\
  &\quad \textbf{case: } x' \notin \{x_h,x_t\}\\
  &\quad x \in \overline V \tag{def of $V''$}\\
  &\quad l' \in locs_{V,H}(e) \tag{def of $locs_{V,H}$}\\
  &\quad \textbf{case: } x' = x_h\\
  &\quad H \vDash (l,l'') : l \leadsto l'' \tag{Edge}\\
  &\quad H \vDash p \circ (l,l'') : l \leadsto l' \tag{Comp}\\
  &\quad l' \in locs_{V,H}(e) \tag{def of $locs_{V,H}$}\\
  &\text{thus we have } locs_{V''H}(e_2) \subseteq locs_{V,H}(e)\\
  &F \cap locs_{V'',H}(e_2) = \emptyset \tag{Sp.}\\
  &g \cap locs_{V'',H}(e_2) = \emptyset \tag{def. of $g$}\\
  &(F \cup g) \cap locs_{V'',H}(e_2) = \emptyset\\
  &|F \cup g| - |F'| \le  \Phi_{V,H}(\Gamma',x_h:A,x_t:L^p(A)) + q + p + 1 - (\Phi_{H'}(v:B) + q') \tag{IH}\\
  &\quad = \Phi_{V,H}(\Gamma') + \Phi_H(v_h:A) + \Phi_H(v_t:L^p(A)) + p + q + 1 - (\Phi_{H'}(v:B) + q') \tag{def of $\Phi_{V,H}$}\\
  &\quad = \Phi_{V,H}(\Gamma') + \Phi_H(\pairexcst{v_h}{v_t}^L : L^p(A)) + q + 1 - (\Phi_{H'}(v:B) + q') \tag{Lemma 4.1.1}\\
  &\quad = \Phi_{V,H}(\Gamma', z : L^p(A)) + q + 1 - (\Phi_{H'}(v:B) + q') \tag{def of $\Phi_{V,H}$}\\
  &\quad = \Phi_{V,H}(\Gamma) + q + 1 - (\Phi_{H'}(v:B) + q') \tag{Lemma 4.1.1}\\
  &\text{suppose } l \in locs_{V',H}(e_2) \\
  &\exists x' \in FV(e_2) \cap \overline V'', l' \in root(\overline V''(x')). x \ne x', H \vDash p : l' \leadsto l \tag{def. of $locs_{V,H}$}\\
  &\quad\textbf{case: } x' \notin \{x_h,x_t\}\\
  &\quad \text{contradiction by} \na{V}{H}\\ 
  &\quad\textbf{case: } x' = x_h\\
  &\quad H \vDash p \circ (l,l') : l \leadsto l\\
  &\quad H \vDash id_l : l \leadsto l\\
  &\quad \text{contradiction since } \ms{linear}_H(r_x,r_x)\\
  &\text{hence we have } l \notin locs_{V'',H}(e_2)\\
  &l \in g \tag{def of $g$}\\
  &|g| \ge 1\\
  &|F \cup g| - |F'|\\
  &\quad = |F| + |g| - |F'| \tag{$F,g$ disjoint}\\
  &\text{Hence,}\\
  &|F| + |g| - |F'| \le  \Phi_{V,H}(\Gamma) + q + 1 - (\Phi_{H'}(v:B) + q')\\
  &|F| - |F'| \le \Phi_{V,H}(\Gamma) + q + 1 - |g| - (\Phi_{H'}(v:B) + q')\\
  &\quad \le \Phi_{V,H}(\Gamma) + q - (\Phi_{H'}(v:B) + q') \tag{$|g| \ge 1$}\\
  \end{align*}
\end{proof}


\end{document}
