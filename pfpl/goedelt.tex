%\chapter{\foreignlanguage{german}{G"odel's} \LangT{}}
\chapter{System \LangT{} of Higher-Order Recursion}
\chaplabel{goedelt}

System \LangT{}, well-known as \emph{G\"{o}del's T}, is the
combination of function types with the type of natural numbers.  In
contrast to $\LangExpr$, which equips the naturals with some
arbitrarily chosen arithmetic operations, the language \LangT{}
provides a general mechanism, called \emph{primitive recursion}, from
which these primitives may be defined.  Primitive recursion captures
the essential inductive character of the natural numbers, and hence
may be seen as an intrinsic termination proof for each program in the
language.  Consequently, we may only define \emph{total} functions in
the language, those that always return a value for each argument.  In
essence every program in \LangT{} ``comes equipped'' with a proof of
its termination.  Although this may seem like a shield against
infinite loops, it is also a weapon that can be used to show that some
programs cannot be written in \LangT{}.  To do so would demand a
master termination proof for every possible program in the language,
something that we shall prove does not exist.

\section{Statics}
\seclabel{goedelt}{statics}

\index{G\"{o}del's \LangT{}}
\index{\LangT{}|see{G\"{o}del's \LangT}}
The syntax of \LangT{} is given by the following grammar:
\begin{synchart}{syn}
  \TypeSort & \tau & \bnfdef & \nattyabt & \nattycst & \text{naturals} \\
              &      & \bnfalt & \arrtyabt{\tau_1}{\tau_2} &
              \arrtycst{\tau_1}{\tau_2} & \text{function} \\
  \ExprSort & e    & \bnfdef & x & x & \text{variable} \\
              &      & \bnfalt & \zeroabt & \zerocst & \text{zero} \\
              &      & \bnfalt & \succabt{e} & \succcst{e} & \text{successor} \\
              &      & \bnfalt & \natrecabt{\tau}{e}{e_0}{x}{y}{e_1} &
              \multicolumn{2}{l}{\natreccst{\tau}{e}{e_0}{x}{y}{e_1}} \\
              &      &             &            &  & \text{recursion} \\
              &      & \bnfalt & \lamabt{\tau}{x}{e} & \lamcst{\tau}{x}{e} &
              \text{abstraction} \\
              &      & \bnfalt & \appabt{e_1}{e_2} & \appcst{e_1}{e_2} & \text{application}
\end{synchart}
\index{G\"{o}del's \LangT{}!recursor}
We write $\numeral{n}$ for the expression
$\succabt{\dots \succabt{\zeroabt}}$, in which the successor is applied $n\geq
0$ times to zero.  The expression $\natrecabt{\tau}{e}{e_0}{x}{y}{e_1}$ is
called the \emph{recursor}.  It represents the $e$-fold iteration of the
transformation $\AbsABT{x}{\AbsABT{y}{e_1}}$ starting from $e_0$.  The bound
variable $x$ represents the predecessor and the bound variable $y$ represents
the result of the $x$-fold iteration.  The ``\textcd{with}'' clause in the
concrete syntax for the recursor binds the variable $y$ to the result of the
recursive call, as will become clear shortly.

\index{G\"{o}del's \LangT{}!iterator}
Sometimes the \emph{iterator},
$\natiterabt{\tau}{e}{e_0}{y}{e_1}$, is considered as an alternative to the
recursor.  It has essentially the same meaning as the recursor, except that only
the result of the recursive call is bound to $y$ in $e_1$, and no binding is
made for the predecessor.  Clearly the iterator is a special case of the
recursor, because we can always ignore the predecessor binding.  Conversely, the
recursor is definable from the iterator, provided that we have product types
(Chapter~\chapref{prod}) at our disposal.  To define the recursor from the
iterator, we simultaneously compute the predecessor while iterating the
specified computation.

\smallskip

\index{G\"{o}del's \LangT{}!statics}
The statics of \LangT{} is given by the following typing rules:
\begin{infrules}{ss}
  \begin{infrule}{var}
    \infer
    {\Gamma,\IsOf{x}{\tau}\vdash\IsOf{x}{\tau}}
    {\strut}
  \end{infrule}
  \begin{infrule}{zero}
    \infer
    {\Gamma\vdash \IsOf{\zeroabt}{\nattyabt}}
    {\strut}
  \end{infrule}
  \begin{infrule}{succ}
    \infer{\Gamma \vdash \IsOf{\succabt{e}}{\nattyabt}}
    {\Gamma\vdash \IsOf{e}{\nattyabt}}
  \end{infrule}
  \begin{infrule}{rec}
    \infer{\Gamma\vdash \IsOf{\natrecabt{\tau}{e}{e_0}{x}{y}{e_1}}{\tau}}
    {\Gamma\vdash \IsOf{e}{\nattyabt} &
      \Gamma\vdash \IsOf{e_0}{\tau} &
      \Gamma, \IsOf{x}{\nattyabt}, \IsOf{y}{\tau} \vdash \IsOf{e_1}{\tau}}
  \end{infrule}
  \begin{infrule}{lam}
    \infer
    {\Gamma\vdash\IsOf{\lamabt{\tau_1}{x}{e}}{\arrtyabt{\tau_1}{\tau_2}}}
    {\Gamma,\IsOf{x}{\tau_1}\vdash \IsOf{e}{\tau_2}}
  \end{infrule}
  \begin{infrule}{app}
    \infer
    {\Gamma\vdash\IsOf{\appabt{e_1}{e_2}}{\tau}}
    {\Gamma\vdash\IsOf{e_1}{\arrtyabt{\tau_2}{\tau}} &
      \Gamma\vdash\IsOf{e_2}{\tau_2}}
  \end{infrule}
\end{infrules}

As usual, admissibility of the structural rule of substitution is crucially
important.
\begin{lemma}
  \lemlabel{goedelt}{subst}
  If $\Gamma\vdash \IsOf{e}{\tau}$ and $\Gamma,\IsOf{x}{\tau}\vdash
  \IsOf{e'}{\tau'}$, then $\Gamma\vdash \IsOf{\Subst{e}{x}{e'}}{\tau'}$.
\end{lemma}

\section{Dynamics}
\seclabel{goedelt}{dynamics}

\index{G\"{o}del's \LangT{}!dynamics}
The closed values of \LangT{} are defined by the following rules:
\begin{infrules}{ds-val}
  \begin{infrule}{zero}
    \infer
    {\IsVal{\zeroabt}}
    {\strut}
  \end{infrule}
  \begin{infrule}{succ}
    \infer
    {\IsVal{\succabt{e}}}
    {\optional{\IsVal{e}}}
  \end{infrule}
  \begin{infrule}{lam}
    \infer
    {\IsVal{\lamabt{\tau}{x}{e}}}
    {\strut}
  \end{infrule}
\end{infrules}
The premise of rule~\infruleref{goedelt}{ds-val}{succ} is included for an
\emph{eager} interpretation of successor, and excluded for a \emph{lazy}
interpretation.

The transition rules for the dynamics of \LangT{} are as follows:
\begin{infrules}{ds}
   \begin{optinfrule}{succ-step}
     \infer
     {{\succabt{e}}\StepsTo{\succabt{e'}}}
     {{e}\StepsTo{e'}}
   \end{optinfrule}
  \begin{infrule}{app-fun}
    \infer
    {\appabt{e_1}{e_2}\StepsTo\appabt{e_1'}{e_2}}
    {e_1\StepsTo e_1'}
  \end{infrule}
  \begin{optinfrule}{app-arg}
    \infer
    {\appabt{e_1}{e_2}\StepsTo\appabt{e_1}{e_2'}}
    {\IsVal{e_1} & e_2\StepsTo e_2'}
  \end{optinfrule}
  \begin{infrule}{app-beta}
    \infer
    {\appabt{\lamabt{\tau}{x}{e}}{e_2}\StepsTo\Subst{e_2}{x}{e}}
    {\optional{\IsVal{e_2}}}
  \end{infrule}
  \begin{infrule}{rec-step}
    \infer
    {{\natrecabt{\tau}{e}{e_0}{x}{y}{e_1}}\StepsTo{\natrecabt{\tau}{e'}{e_0}{x}{y}{e_1}}}
    {{e}\StepsTo{e'}}
  \end{infrule}
  \begin{infrule}{rec-zero}
    \infer
    {{\natrecabt{\tau}{\zeroabt}{e_0}{x}{y}{e_1}}\StepsTo{e_0}}
    {\strut}
  \end{infrule}
  \begin{infrule}{rec-succ}
    \infer
    {{\natrecabt{\tau}{\succabt{e}}{e_0}{x}{y}{e_1}}
      \StepsTo
      {\Subst{e,\natrecabt{\tau}{e}{e_0}{x}{y}{e_1}}{x,y}{e_1}}}
    {\IsVal{\succabt{e}}}
  \end{infrule}
\end{infrules}
The bracketed rules and premises are included for an eager successor
and call-by-value application, and omitted for a lazy successor and
call-by-name application.  Rules~\infruleref{goedelt}{ds}{rec-zero}
and ~\infruleref{goedelt}{ds}{rec-succ} specify the behavior of the
recursor on $\zeroabt$ and $\succabt{e}$.  In the former case the
recursor reduces to $e_0$, and in the latter case the variable $x$ is
bound to the predecessor $e$ and $y$ is bound to the (unevaluated)
recursion on $e$.  If the value of $y$ is not required in the rest of
the computation, the recursive call is not evaluated.

% \begin{lemma}
%   If $\IsOf{e}{\nattycst}$ and $\IsVal{e}$, then $e=\numeral{n}$ for some $n\geq
%   0$.  That is, either $e=\zerocst$ or $e=\succcst{e'}$ for some $e'$ such that
%   $\IsNat{e'}$ and $\IsVal{e'}$.
% \end{lemma}

\begin{comment}
If, instead, we prefer to insist that the recursive call always be evaluated we
may replace rule~\infruleref{goedelt}{dynamic}{rec-succ} with the following
variant:
\begin{infrule*}{rec-succ}
  \infer[.]
  {
    \begin{gathered}
    \natrecabt{\tau}{\succabt{e}}{e_0}{x}{y}{e_1} \\[-1ex]
    \StepsTo  \\[-1ex]
    \appabt{\appabt{\lamabt{\nattyabt}{x}{\lamabt{\tau}{y}{e_1}}}{e}}{\natrecabt{\tau}{e}{e_0}{x}{y}{e_1}}
    \end{gathered}
  }
  {\IsVal{\succabt{e}}}
\end{infrule*}
Rule~\infrulerefstar{goedelt}{rec-succ} forces evaluation of the recursive call prior to evaluation of $e_1$.
\end{comment}

\begin{lemma}[Canonical Forms]
  \lemlabel{goedelt}{cfl} If $\IsOf{e}{\tau}$ and $\IsVal{e}$, then
  \begin{enumerate}
  \item If $\tau=\nattycst$, then $e=\succcst{\succcst{\dots\zerocst}}$ for some
    number $n\geq 0$ occurrences of the successor starting with zero.
  \item If $\tau=\arrtycst{\tau_1}{\tau_2}$, then $e=\lamcst{\tau_1}{x}{e_2}$
    for some $e_2$.
  \end{enumerate}
\end{lemma}

\index{G\"{o}del's \LangT{}!safety}

\begin{theorem}[Safety]
  \thmlabel{goedelt}{safety}
  \begin{enumerate}
  \item If $\IsOf{e}{\tau}$ and $e\StepsTo e'$, then $\IsOf{e'}{\tau}$.
  \item If $\IsOf{e}{\tau}$, then either $\IsVal{e}$ or $e\StepsTo e'$ for some
    $e'$.
  \end{enumerate}
\end{theorem}

\section{Definability}
\seclabel{goedelt}{definability}

\index{G\"{o}del's \LangT{}!definability}
A mathematical function $f:\mathbb{N}\to\mathbb{N}$ on the natural numbers is
\emph{definable} in \LangT{} iff there exists an expression $e_f$ of type
$\arrtycst{\nattycst}{\nattycst}$ such that for every $n\in\mathbb{N}$,
\begin{displayed}{definability}
  \IsDefEqvCl{\appcst{e_f}{\numeral{n}}}{\numeral{f(n)}}{\nattycst}.
\end{displayed}
That is, the numeric function $f:\mathbb{N}\to\mathbb{N}$ is definable iff there
is an expression $e_f$ of type $\arrtycst{\nattycst}{\nattycst}$ such that, when
applied to the numeral representing the argument $n\in\mathbb{N}$, the
application is definitionally equal to the numeral corresponding to
$f(n)\in\mathbb{N}$.

\index{G\"{o}del's \LangT{}!definitional equality}
Definitional equality for \LangT{}, written
$\IsDefEqv{\Gamma}{e}{e'}{\tau}$, is the strongest congruence containing these
axioms:
\begin{infrules}{eq}
  \begin{infrule}{beta}
    \infer
    {\IsDefEqv{\Gamma}{\appabt{\lamabt{\tau_1}{x}{e_2}}{e_1}}{\Subst{e_1}{x}{e_2}}{\tau_2}}
    {\Gamma,\IsOf{x}{\tau_1}\vdash\IsOf{e_2}{\tau_2} & \Gamma\vdash\IsOf{e_1}{\tau_1}}
  \end{infrule}
  \begin{infrule}{rec-zero}
    \infer
    {\IsDefEqv{\Gamma}{\natrecabt{\tau}{\zeroabt}{e_0}{x}{y}{e_1}}{e_0}{\tau}}
    {\Gamma\vdash\IsOf{e_0}{\tau} & \Gamma,\IsOf{x}{\tau}\vdash \IsOf{e_1}{\tau}}
  \end{infrule}
  \begin{infrule}{rec-succ}
    \infer
    {\IsDefEqv{\Gamma}{\natrecabt{\tau}{\succabt{e}}{e_0}{x}{y}{e_1}}{\Subst{e,\natrecabt{\tau}{e}{e_0}{x}{y}{e_1}}{x,y}{e_1}}{\tau}}
    {\Gamma\vdash\IsOf{e_0}{\tau} & \Gamma,\IsOf{x}{\tau}\vdash \IsOf{e_1}{\tau}}
  \end{infrule}
\end{infrules}
%It may be proved that if $\IsOf{e}{\nattycst}$, then
%$\IsDefEqvCl{e}{\numeral{n}}{\nattycst}$ iff $e\MultiStepsTo\numeral{n}$.

\smallskip

For example, the doubling function, $d(n)=2\times n$, is definable in
\LangT{} by the expression $\IsOf{e_d}{\arrtycst{\nattycst}{\nattycst}}$
given by
\begin{displayed*}
  \lamcst{\nattycst}{x}{\natreccst{\nattycst}{x}{\zerocst}{u}{v}{\succcst{\succcst{v}}}}.
\end{displayed*}
To check that this defines the doubling function, we proceed by induction on
$n\in\mathbb{N}$.  For the basis, it is easy to check that
\begin{displayed*}
  \IsDefEqvCl{\appcst{e_d}{\numeral{0}}}{\numeral{0}}{\nattycst}.
\end{displayed*}
For the induction, assume that
\begin{displayed*}
  \IsDefEqvCl{\appcst{e_d}{\numeral{n}}}{\numeral{d(n)}}{\nattycst}.
\end{displayed*}
Then calculate using the rules of definitional equality:
\begin{align*}
  \appcst{e_d}{\numeral{n+1}}
  & \equiv \succcst{\succcst{\appcst{e_d}{\numeral{n}}}} \\
  & \equiv \succcst{\succcst{\numeral{2\times n}}} \\
  & = \numeral{2\times (n+1)} \\
  & = \numeral{d(n+1)}.
\end{align*}

%\section{Ackermann's Function}
%\seclabel{goedelt}{ackermann}
\smallskip

As another example, consider the following function, called \emph{Ackermann's
  function}, defined by the following equations:
\begin{align*}
A(0,n)     &= n+1 \\
A(m+1,0)   &= A(m,1) \\
A(m+1,n+1) &= A(m, A(m+1,n)).
\end{align*}
The Ackermann function grows very quickly.  For example, $A(4,2) \approx 2^{65,536}$,
which is often cited as being larger than the number of atoms in the
universe!  Yet we can show that the Ackermann function is total by a
lexicographic induction on the pair of arguments $(m,n)$.  On each recursive
call, either $m$ decreases, or else $m$ remains the same, and $n$ decreases, so
inductively the recursive calls are well-defined, and hence so is $A(m,n)$.

A \emph{first-order primitive recursive function} is a function of type
$\arrtycst{\nattycst}{\nattycst}$ that is defined using the recursor, but
without using any higher order functions.  Ackermann's function is defined so
that it is not first-order primitive recursive, but is higher-order primitive
recursive.  The key to showing that it is definable in \LangT{} is to
note that $A(m+1,n)$ iterates $n$ times the function $A(m,-)$, starting with
$A(m,1)$.  As an auxiliary, let us define the higher-order function
\begin{displaymath}
\IsOf{\cd{it}}{\arrtycst{\cdparens{\arrtycst{\nattycst}{\nattycst}}}{\arrtycst{\nattycst}{\arrtycst{\nattycst}{\nattycst}}}}
\end{displaymath}
to be the $\lambda$-abstraction
\begin{displaymath}
  {\lamcst
    {\arrtycst{\nattycst}{\nattycst}}
    {f}
    {\lamcst
      {\nattycst}
      {n}
      {\natreccst
        {\nattycst}
        {n}
        {\cd{id}}
        {\_}{g}{f\circ g}}}},
\end{displaymath}
where $\cd{id}=\lamcst{\nattycst}{x}{x}$ is the identity, and $f\circ g =
\lamcst{\nattycst}{x}{\appcst{f}{\appcst{g}{x}}}$ is the composition of $f$ and
$g$.  It is easy to check that
\begin{displaymath}
  \IsDefEqvCl
  {\appcst{\appcst{\appcst{\cd{it}}{f}}{\numeral{n}}}{\numeral{m}}}
  {f^{\parens{n}}(\numeral{m})}
  {\nattycst},
\end{displaymath}
where the latter expression is the $n$-fold composition of $f$ starting with
$\numeral{m}$.  We may then define the Ackermann function
\begin{displaymath}
  \IsOf{e_a}{\arrtycst{\nattycst}{\arrtycst{\nattycst}{\nattycst}}}
\end{displaymath}
to be the expression
\begin{displaymath}
  \lamcst
  {\nattycst}
  {m}
  {\natreccst
    {\arrtycst{\nattycst}{\nattycst}}
    {m}
    {\cd{s}}
    {\_}
    {f}
    {\lamcst
      {\nattycst}
      {n}
      {\appcst{\appcst{\appcst{\cd{it}}{f}}{n}}{\appcst{f}{\numeral{1}}}}}}.
\end{displaymath}

It is instructive to check that the following equivalences are valid:
\begin{align}
  \appcst{\appcst{e_a}{\numeral{0}}}{\numeral{n}}  &\equiv{} \succcst{\numeral{n}} \\
  \appcst{\appcst{e_a}{\numeral{m+1}}}{\numeral{0}} & \equiv{}
  \appcst{\appcst{e_a}{\numeral{m}}}{\numeral{1}} \\
  \appcst{\appcst{e_a}{\numeral{m+1}}}{\numeral{n+1}} &\equiv{}
  \appcst{\appcst{e_a}{\numeral{m}}}{\appcst{\appcst{e_a}{\succcst{\numeral{m}}}}{\numeral{n}}}.
\end{align}
That is, the Ackermann function is definable in \LangT{}.

\section{Undefinability}
\seclabel{goedelt}{nondef}

\index{G\"{o}del's \LangT{}!undefinability}
It is impossible to define an infinite loop in \LangT{}.
\begin{theorem}
  \thmlabel{goedelt}{termination}
  If $\IsOf{e}{\tau}$, then there exists $\IsVal{v}$ such that $\IsDefEqvCl{e}{v}{\tau}$.
\end{theorem}
\begin{proof}
  See Corollary~\corref{equiv-t}{termination}.
\end{proof}

Consequently, values of function type in \LangT{} behave like mathematical
functions: if $\IsOf{e}{\arrtycst{\tau_1}{\tau_2}}$ and $\IsOf{e_1}{\tau_1}$, then
$\appcst{e}{e_1}$ evaluates to a value of type $\tau_2$.  Moreover, if
$\IsOf{e}{\nattycst}$, then there exists a natural number $n$ such that
$\IsDefEqvCl{e}{\numeral{n}}{\nattycst}$.

Using this, we can show, using a technique called \emph{diagonalization}, that
there are functions on the natural numbers that are not definable in
\LangT{}.  We make use of a technique, called \emph{G\"{o}del-numbering},
that assigns a unique natural number to each closed expression of \LangT{}.
By assigning a unique number to each expression, we may manipulate
expressions as data values in \LangT{} so that \LangT{} is able to
compute with its own programs.\footnote{The same
  technique lies at the heart of the proof of G\"{o}del's celebrated
  incompleteness theorem.  The undefinability of certain functions on the
  natural numbers within \LangT{} may be seen as a form of incompleteness
  like that considered by G\"{o}del.}

The essence of G\"{o}del-numbering is captured by the following simple
construction on abstract syntax trees.  (The generalization to abstract binding
trees is slightly more difficult, the main complication being to ensure that all
$\alpha$-equivalent expressions are assigned the same G\"{o}del number.)  Recall
that a general ast $a$ has the form $\OpAST{o}{a_1,\dots,a_k}$, where $o$ is
an operator of arity $k$.  Enumerate the operators so that every
operator has an index $i\in\mathbb{N}$, and let $m$ be the index of $o$ in this
enumeration.  Define the \emph{G\"{o}del number} $\corners{a}$ of $a$ to be the
number
\begin{displayed*}
  2^m\,3^{n_1}\,5^{n_2}\,\dots\,p_k^{n_k},
\end{displayed*}
where $p_k$ is the $k$th prime number (so that $p_0=2$, $p_1=3$, and so on), and
$n_1,\dots,n_k$ are the G\"{o}del numbers of $a_1,\dots,a_k$, respectively.
This procedure assigns a natural number to each ast.  Conversely, given a natural number,
$n$, we may apply the prime factorization theorem to ``parse'' $n$ as a unique
abstract syntax tree.  (If the factorization is not of the right form,
which can only be because the arity of the operator does not match the number of
factors, then $n$ does not code any ast.)

Now, using this representation, we may define a (mathematical) function
$f_{\textit{univ}}:\mathbb{N}\to\mathbb{N}\to\mathbb{N}$ such that, for any
$\IsOf{e}{\arrtycst{\nattycst}{\nattycst}}$,
$f_{\textit{univ}}(\corners{e})(m)=n$ iff
$\IsDefEqvCl{\appcst{e}{\numeral{m}}}{\numeral{n}}{\nattycst}$.\footnote{The
  value of $f_{\textit{univ}}(k)(m)$ may be chosen arbitrarily to be zero when
  $k$ is not the code of any expression $e$.}  The determinacy of the dynamics,
together with Theorem~\thmref{goedelt}{termination}, ensure that
$f_{\textit{univ}}$ is a well-defined function.  It is called the
\emph{universal function} for \LangT{} because it specifies the behavior of
any expression $e$ of type $\arrtycst{\nattycst}{\nattycst}$.  Using the
universal function, let us define an auxiliary mathematical function, called the
\emph{diagonal function} $\delta:\mathbb{N}\to\mathbb{N}$, by the equation
$\delta(m)=f_{\textit{univ}}(m)(m)$.  The $\delta$ function is chosen so that
$\delta(\corners{e})=n$ iff
$\IsDefEqvCl{\appcst{e}{\numeral{\corners{e}}}}{\numeral{n}}{\nattycst}$.  (The
motivation for its definition will become clear in a moment.)

The function $f_{\textit{univ}}$ is not definable in \LangT{}.  Suppose
that it were definable by the expression $e_{\textit{univ}}$, then the diagonal
function $\delta$ would be definable by the expression 
\begin{displayed*}
e_{\delta}=\lamcst{\nattycst}{m}{\appcst{\appcst{e_{\textit{univ}}}{m}}{m}}.
\end{displayed*}
But in that case we would have the equations
\begin{align*}
  \appcst{e_{\delta}}{\numeral{\corners{e}}} & \equiv
  \appcst{\appcst{e_{\textit{univ}}}{\numeral{\corners{e}}}}{\numeral{\corners{e}}}
  \\
  & \equiv \appcst{e}{\numeral{\corners{e}}}.
\end{align*}
Now let $e_{\Delta}$ be the function expression
\begin{displayed*}
\lamcst{\nattycst}{x}{\succcst{\appcst{e_{\delta}}{x}}},
\end{displayed*}
so that we may deduce
\begin{align*}
  \appcst{e_{\Delta}}{\numeral{\corners{e_{\Delta}}}}
  & \equiv \succcst{\appcst{e_{\delta}}{\numeral{\corners{e_{\Delta}}}}} \\
  & \equiv \succcst{\appcst{e_{\Delta}}{\numeral{\corners{e_{\Delta}}}}}.
\end{align*}
But the termination theorem implies that there exists $n$ such that
$\DefEqvU{\appcst{e_{\Delta}}{\numeral{\corners{e_{\Delta}}}}}{\numeral{n}}$, and hence we
have $\DefEqvU{\numeral{n}}{\succcst{\numeral{n}}}$, which is impossible.

We say that a language $\mathcal{L}$ is \emph{universal} if it is possible to
write an interpreter for $\mathcal{L}$ in $\mathcal{L}$ itself.  It is
intuitively clear that $f_{\textit{univ}}$ is computable in the sense that we
can define it in \emph{some} sufficiently powerful programming language.  But
the preceding argument shows that \LangT{} is not up to the task; it is not
a universal programming language.  Examination of the foregoing proof reveals an
inescapable tradeoff: by insisting that all expressions terminate, we
necessarily lose universality---there are computable functions that are not
definable in the language.

\section{Notes}
\seclabel{goedelt}{notes}

System \LangT{} was introduced by G\"{o}del in his study of the
consistency of arithmetic \citep{goedel:t}.  He showed how to
``compile'' proofs in arithmetic into well-typed terms of system
\LangT{}, and to reduce the consistency problem for arithmetic to the
termination of programs in \LangT{}.  It was perhaps the first
programming language whose design was directly influenced by the
verification (of termination) of its programs.

\begin{exercises}
  
  \begin{exercise}{cfl}
    \index{G\"{o}del's \LangT{}!canonical forms}
%
    Prove Lemma~\lemref{goedelt}{cfl}.
  \end{exercise}

  \begin{exercise}{safety}
    \index{G\"{o}del's \LangT{}!safety}
%
    Prove Theorem~\thmref{goedelt}{safety}.
  \end{exercise}

  \begin{exercise}{nat-term}
    \index{G\"{o}del's \LangT{}!termination}
%
    Attempt to prove that if $e:\nattycst$ is closed, then there exists $n$ such that
    $e\MultiStepsTo\numeral{n}$.  Where does the proof break down?
  \end{exercise}

  \begin{exercise}{direct-term}
    \index{G\"{o}del's \LangT{}!termination}
%
    Attempt to prove termination for all well-typed closed terms: if $\IsOf{e}{\tau}$,
    then there exists $\IsVal{e'}$ such that $e\MultiStepsTo e'$.  You are free
    to appeal to Lemma~\lemref{goedelt}{cfl} and
    Theorem~\thmref{goedelt}{safety} as necessary.  Where does the attempt break
    down?  Can you think of a stronger inductive hypothesis that might evade the
    difficulty?
  \end{exercise}

  \begin{exercise}{hered-term}
    \index{G\"{o}del's \LangT{}!hereditary termination}
    Define a closed term $e$ of type $\tau$ in \LangT{} to be \emph{hereditarily
      terminating at type $\tau$} by induction on the structure of $\tau$ as
    follows:
    \begin{enumerate}
    \item If $\tau=\nattycst$, then $e$ is hereditarily
      terminating at type $\tau$ iff $e$ is terminating (that is, iff
      $e\MultiStepsTo\numeral{n}$ for some $n$.)
    \item If $\tau=\arrtycst{\tau_1}{\tau_2}$, then $e$ is 
      hereditarily terminating iff when $e_1$ is hereditarily terminating at
      type $\tau_1$, then $\appcst{e}{e_1}$ is hereditarily terminating at type
      $\tau_2$.
    \end{enumerate}
    Attempt to prove hereditary termination for well-typed terms: if $e:\tau$,
    then $e$ is hereditarily terminating at type $\tau$.  The stronger inductive
    hypothesis bypasses the difficulty that arose in
    Exercise~\exref{goedelt}{direct-term}, but introduces another obstacle.  What
    is the complication?  Can you think of an even stronger inductive hypothesis
    that would suffice for the proof?
  \end{exercise}

  \begin{exercise}{head-exp}
    Show that if $e$ is hereditarily terminating at type $\tau$,
    $\IsOf{e'}{\tau}$, and $e'\StepsTo e$, then $e$ is also hereditarily
    terminating at type $\tau$.  (The need for this result will become
    clear in the solution to Exercise~\exref{goedelt}{hered-term}.)
  \end{exercise}

  \begin{exercise}{gen-term}
    Define an open, well-typed term
    \begin{displayed*}
      \IsOf{x_1}{\tau_1},\dots,\IsOf{x_n}{\tau_n}\vdash\IsOf{e}{\tau}
    \end{displayed*}
    to be \emph{open hereditarily terminating} iff every substitution instance
    \begin{displayed*}
      \Subst{e_1,\dots,e_n}{x_1,\dots,x_n}{e}
    \end{displayed*}
    is closed hereditarily terminating at type $\tau$ when each $e_i$ is
    closed hereditarily terminating at type $\tau_i$ for each $1\leq i\leq n$.
    Derive Exercise~\exref{goedelt}{nat-term} from this result.
  \end{exercise}

\end{exercises}

%%% Local Variables: 
%%% mode: latex
%%% TeX-master: "book"
%%% End: 
