% Copyright 2004 by Till Tantau <tantau@users.sourceforge.net>.
%
% In principle, this file can be redistributed and/or modified under
% the terms of the GNU Public License, version 2.
%
% However, this file is supposed to be a template to be modified
% for your own needs. For this reason, if you use this file as a
% template and not specifically distribute it as part of a another
% package/program, I grant the extra permission to freely copy and
% modify this file as you see fit and even to delete this copyright
% notice. 

\documentclass{beamer}

\usepackage{adjustbox}
\usepackage{booktabs}
\usepackage{fullpage}
\usepackage{latexsym}
\usepackage{verbatim}
\usepackage{amsthm}
\usepackage{amssymb}
\usepackage{amsmath}
\usepackage{stackengine}
\usepackage{scalerel}
\usepackage{code,proof,amsthm,amssymb, amsmath, mathbbol}
\usepackage{proof}
\usepackage{mathpartir}
\usepackage{turnstile}
\usepackage{fancyvrb}
\usepackage{stmaryrd}
\usepackage{mathtools}
\usepackage{graphicx}
\usepackage{hyperref}
\usepackage{tikz}
\usepackage[cache=false]{minted}
\graphicspath{ {./} }
\usetikzlibrary{shapes.geometric, arrows}
\newcommand{\ms}[1]{\ensuremath{\mathsf{#1}}}
\newcommand{\irl}[1]{\mathtt{#1}}

\newcounter{rule}
\setcounter{rule}{0}
\newcommand{\rn}
  {\addtocounter{rule}{1}(\arabic{rule})}	

\newcounter{infercount}
\setcounter{infercount}{1}
\newcommand{\infern}[2]{\inferrule{#1}{#2}(\text{S}_{\arabic{infercount}}\stepcounter{infercount})}
\newcommand*\ts[2]{%
  \,\scalebox{1}[0.5]{$\sststile[ss]{\textstyle#1}{\textstyle#2}$}\,
}
\newcommand{\inferr}[2]{\inferrule{#2}{#1}}
\newcommand{\inferrr}[3]{\inferrule[#1]{#2}{#3}}
\newcommand{\paircaseabt}[4]{\irl{match_P}(#2,#3.#4)}
\newcommand{\paircasecst}[4]{\irl{match} \; #1\; \{(#2;#3) \hookrightarrow #4\}}
\newcommand{\na}[1]{\mathsf{linearCtxt}(#1)}
\newcommand{\nr}[1]{\mathsf{no\_ref}(#1)}
\newcommand{\stable}[1]{\mathsf{stable}(#1)}
\newcommand{\set}[1]{\mathsf{set}(#1)}
\newcommand{\safe}[1]{\mathsf{safe}(#1)}
\newcommand{\dist}[1]{\mathsf{disjoint}(#1)}
\newcommand{\stack}[1]{\irl{stack}(#1)}
\newcommand{\denote}[1]{\llbracket#1\rrbracket}
\newcommand{\nil}{[]}
\newcommand{\cons}[2]{\pi(#1,#2)}
\newcommand{\sharecst}[4]{\irl{share}\;#1\;\irl{as}\;#2,#3\;\irl{in}\;#4}
\newcommand{\sharecpcst}[4]{\irl{share}\;#1\;\irl{as}\;#2,#3\;\irl{in}\;#4}
\newcommand{\shareabt}[4]{\irl{share}(#1;#2,#3.#4)}
\newcommand{\ssize}[2]{\left\Vert #2 \right\Vert_{#1}}
\newcommand{\card}[1]{card(#1)}
\newcommand{\val}[1]{\irl{val}(#1)}
\newcommand{\gc}[3]{\mathsf{gc}(#1,#2,#3)}
\newcommand{\wfc}[5]{\mathsf{linearComp}(#1,#2,#3,#4,#5)}
\newcommand{\veq}[4]{#3 \sim^{#1}_{#2} #4}
\newcommand{\ctxeq}[2]{(#1) \sim (#2)}
\newcommand{\oh}[1]{\oslash(#1)}
\newcommand{\fogc}{\ms{FO}^{gc}}
\newcommand{\jan}[1]{{\color{red} [\emph{Jan: #1}]}}
\newcommand{\yue}[1]{{\color{blue} [\emph{Yue: #1}]}}
\newcommand{\gcSem}{\ensuremath{\mathcal{E}_{\ms{gc}}}}
\newcommand{\copySem}{\ensuremath{\mathcal{E}_{\ms{copy}}}}
\newtheorem{attempt}{Attempt}
\theoremstyle{definition}

\input{./generic-defns}
\input{./syn-defns}
\newcommand{\FunEvalsTo}[3]{{#1}\mathrel{\Downarrow'}\AbsABT{#2}{#3}}
\newcommand{\FunEvalsToHO}[3]{{#2}\vargen \fappabt{#1}{#2}\evalsto{#3}}

\newcommand{\arrtyabt}[2]{\OpABT{\cd{arr}}{#1;#2}}
\newcommand{\arrtycst}[2]{{#1}\mathrel{\to}{#2}}
\newcommand{\lamabt}[3]{\OpABTp{\cd{lam}}{#1}{\AbsABT{#2}{#3}}}
\newcommand{\lamcst}[3]{\mathop{\lambda}\cdparens{\TightIsOf{#2}{#1}}\,{#3}}
\newcommand{\appabt}[2]{\OpABT{\cd{ap}}{#1;#2}}
\newcommand{\appcst}[2]{{#1}\cdparens{#2}}

\newcommand{\letabt}[4]{\OpABTp{\cd{let}}{#1}{#2;\AbsABT{#3}{#4}}}
\newcommand{\letcst}[4]{\kwop{let}\TightIsOf{#3}{#1}\kwop{be}{#2}\kwop{in}{#4}}
\newcommand{\uletabt}[3]{\OpABT{\cd{let}}{#1;\AbsABT{#2}{#3}}}
\newcommand{\uletcst}[3]{\kwop{let}{#2}\kwop{be}{#1}\kwop{in}{#3}}

\newcommand{\fdefabt}[6]{\OpABTp{\cd{fun}}{#1;#2}{\AbsABT{#3}{#4};\AbsABT{#5}{#6}}}
\newcommand{\fdefcst}[6]{\kwop{fun}{#5}\TightIsOf{\cdparens{\TightIsOf{#3}{#1}}}{#2}\mathbin{\cd{=}}{#4}\kwop{in}{#6}}
\newcommand{\fappabt}[2]{\OpABTp{\cd{apply}}{#1}{#2}}
\newcommand{\fappcst}[2]{#1\cdparens{#2}}
\newcommand{\fsubst}[3]{\dblsqbracks{#1\mathord{/}#2}{#3}}


%%% Local Variables: 
%%% mode: latex
%%% TeX-master: "book"
%%% End: 

\newcommand{\LangPCF}{\textbsf{PCF}}

\newcommand{\genrecabt}[3]{\OpABTp{\cd{fix}}{#1}{\AbsABT{#2}{#3}}}
\newcommand{\genreccst}[3]{\kwop{fix}{#2}\cdcolon{#1}\kwop{is}{#3}}
\newcommand{\sgenreccst}[2]{\kwop{fix}{#1}\kwop{is}{#2}}

\newcommand{\bddrecabt}[4]{\OpABTp{\cd{fix}^{#1}}{#2}{\AbsABT{#3}{#4}}}
\newcommand{\bddreccst}[4]{\mathop{\cd{fix}^{#1}}{#3}\cdcolon{#2}\kwop{is}{#4}}

\newcommand{\parrtyabt}[2]{\OpABT{\cd{parr}}{#1;#2}}
\newcommand{\parrtycst}[2]{{#1}\mathbin{\rightharpoonup}{#2}}
\newcommand{\pappabt}[2]{\OpABT{\cd{ap}}{#1;#2}}
\newcommand{\pappcst}[2]{\appcst{#1}{#2}}

\newcommand{\funabt}[5]{\OpABTp{\cd{fun}}{#1;#2}{\AbsABT{#3}{\AbsABT{#4}{#5}}}}
\newcommand{\funcst}[5]{\kwop{fun}{#3}\cdparens{{#4}{\cdcolon}{#1}}{\cdcolon}{#2}\kwop{is}{#5}}

\newcommand{\empenv}{\bullet}
\newcommand{\extenv}[3]{{#1},{#2}{=}{#3}}

\newcommand{\expclo}[1]{\widehat{#1}}
\newcommand{\expclois}[2]{\expclo{#1}{=}{#2}}
\newcommand{\expenv}[2]{\widehat{#1}\parens{#2}}
\newcommand{\expenvis}[3]{\expenv{#1}{#2}{=}{#3}}

\newcommand{\cutoff}[2]{{#1}^{(#2)}}

%\newcommand{\Replace}[3]{\sqbracks{#2\mathbin{\leftarrow}#1}{#3}}

\newcommand{\lnattyabt}{\cd{lnat}}
\newcommand{\lnattycst}{\lnattyabt}
\newcommand{\lsuccabt}[2]{\OpABT{\cd{succ}}{\AbsABT{#1}{#2}}}
\newcommand{\lsucccst}[2]{{#1}\kwbin{is}\succcst{#2}}

%%% Local Variables: 
%%% mode: plain-tex
%%% TeX-master: "book"
%%% End: 

\newcommand{\unittyabt}{\kw{unit}}
\newcommand{\unittycst}{\kw{unit}}
\newcommand{\unittycstm}{\top}
\newcommand{\prodtyabt}[2]{\OpABT{\kw{prod}}{#1;#2}}
\newcommand{\prodtycst}[2]{{#1}\times{#2}}

\newcommand{\pprodtycst}[2]{{#1}\otimes{#2}}
\newcommand{\ppairexabt}[2]{\OpABT{\kw{fuse}}{#1;#2}}
\newcommand{\ppairexcst}[2]{{#1}\otimes{#2}}
\newcommand{\splitexabt}[4]{\OpABT{\kw{split}}{{#1};\AbsABT{#2,#3}{#4}}}
\newcommand{\splitexcst}[4]{\kw{split}\,{#1}\,\kw{as}\,\ppairexcst{#2}{#3}\,\kw{in}\,{#4}}
\newcommand{\checkexabt}[2]{\OpABT{\kw{check}}{#1;#2}}
\newcommand{\checkexcst}[2]{\kw{check}\,{#1}\,\kw{as}\,\unitexcst\,\kw{in}\,{#2}}

\newcommand{\unitexabt}{\kw{triv}}
\newcommand{\unitexcst}{\langle\rangle}
\newcommand{\pairexabt}[2]{\OpABT{\kw{pair}}{#1;#2}}
\newcommand{\pairexcst}[2]{\langle #1, #2\rangle}
\newcommand{\projexabt}[2]{\OpABT{\OpInst{\kw{pr}}{#2}}{#1}}
\newcommand{\fstexabt}[1]{\projexabt{#1}{\kw{l}}}
\newcommand{\sndexabt}[1]{\projexabt{#1}{\kw{r}}}
\newcommand{\projexcst}[2]{{#1}\mathbin\cdot{#2}}
\newcommand{\fstexcst}[1]{\projexcst{#1}{\kw{l}}}
\newcommand{\sndexcst}[1]{\projexcst{#1}{\kw{r}}}

\newcommand{\dgenprodcst}[1]{\brackets{#1}}
\newcommand{\dgentuplecst}[1]{\brackets{#1}}
\newcommand{\genprodabt}[3]{\OpABT{\kw{prod}}{\genff{#1}{#2}{#3}}}
\newcommand{\vargenprodcst}[3]{\prod_{{#2}\in{#1}}{#3}}
\newcommand{\genprodcst}[3]{\dgenprodcst{#3}_{#2\in #1}}
\newcommand{\gentupleabt}[3]{\OpABT{\kw{tpl}}{\genff{#1}{#2}{#3}}}
\newcommand{\gentuplecst}[3]{\dgentuplecst{#3}_{{#2}\in{#1}}}
\newcommand{\lgentuplescst}[3]{\gentuplecst{#1}{#2}{\fldexcst{#2}{#3}}}
\newcommand{\genprojabt}[3]{\projexabt{#3}{#2}}
\newcommand{\genprojcst}[3]{\projexcst{#3}{#2}}
\newcommand{\genprodcstgen}[3]{\dgenprodcst{\varexplff{#1}{#2}{#3}}}
\newcommand{\gentuplecstgen}[3]{\dgentuplecst{\varexplff{#1}{#2}{#3}}}

\newcommand{\rcdtycst}[1]{\dgenprodcst{#1}}
\newcommand{\rcdexcst}[1]{\dgentuplecst{#1}}
\newcommand{\fldtycst}[2]{\pairff{#1}{#2}}
\newcommand{\fldexcst}[2]{\pairff{#1}{#2}}
\newcommand{\fldselexcst}[2]{\projexcst{#2}{#1}}


%%% Local Variables: 
%%% mode: latex
%%% TeX-master: "book"
%%% End: 

\input{../pfpl/sum-defns}
\input{../pfpl/icoi-defns}
\newcommand{\LangT}{\textbsf{T}}

\newcommand{\nattyabt}{\cd{nat}}
\newcommand{\nattycst}{\nattyabt}

\newcommand{\zeroabt}{\cd{z}}
\newcommand{\zerocst}{\zeroabt}
\newcommand{\succabt}[1]{\OpABT{\cd{s}}{#1}}
\newcommand{\succcst}[1]{\succabt{#1}}

\newcommand{\predabt}[1]{\OpABT{\cd{p}}{#1}}
\newcommand{\predcst}[1]{\predabt{#1}}

\newcommand{\numeral}[1]{\overline{#1}}

\newcommand{\natrecabt}[6]{\OpABTp{\kw{rec}}{#3;\AbsABT{#4}{\AbsABT{#5}{#6}}}{#2}}
\newcommand{\natreccst}[6]{\kwop{rec}{#2}\,\cdbraces{\zerocst\casebrcst{#3}\casesepcst\succcst{#4}\kwop{with}{#5}\casebrcst{#6}}}

\newcommand{\natiterabt}[5]{\OpABTp{\cd{iter}}{#3;\AbsABT{#4}{#5}}{#2}}
\newcommand{\natitercst}[5]{\kwop{iter}{#2}\,\cdbraces{\zerocst\casebrcst{#3}\casesepcst\succcst{#4}\casebrcst{#5}}}

\newcommand{\natcaseabt}[4]{\OpABTp{\cd{ifz}}{#2;\AbsABT{#3}{#4}}{#1}}
\newcommand{\natcasecst}[4]{\kwop{ifz}{#1}\,\cdbraces{\zerocst\casebrcst{#2}\casesepcst\succcst{#3}\casebrcst{#4}}}

\newcommand{\nullstrabt}{\epsilon}
\newcommand{\consstrabt}[2]{{#1}\cdot{#2}}

%%% Local Variables: 
%%% mode: latex
%%% TeX-master: "book"
%%% End: 



% There are many different themes available for Beamer. A comprehensive
% list with examples is given here:
% http://deic.uab.es/~iblanes/beamer_gallery/index_by_theme.html
% You can uncomment the themes below if you would like to use a different
% one:
%\usetheme{AnnArbor}
%\usetheme{Antibes}
%\usetheme{Bergen}
%\usetheme{Berkeley}
%\usetheme{Berlin}
%\usetheme{Boadilla}
%\usetheme{boxes}
%\usetheme{CambridgeUS}
%\usetheme{Copenhagen}
%\usetheme{Darmstadt}
%\usetheme{default}
%\usetheme{Frankfurt}
%\usetheme{Goettingen}
%\usetheme{Hannover}
%\usetheme{Ilmenau}
%\usetheme{JuanLesPins}
%\usetheme{Luebeck}
\usetheme{Madrid}
%\usetheme{Malmoe}
%\usetheme{Marburg}
%\usetheme{Montpellier}
%\usetheme{PaloAlto}
%\usetheme{Pittsburgh}
%\usetheme{Rochester}
%\usetheme{Singapore}
%\usetheme{Szeged}
%\usetheme{Warsaw}
\title{Automated Space Bounds for Programs with Garbage Collection}

% A subtitle is optional and this may be deleted
\subtitle{}

\author{Yue Niu \and Jan Hoffmann}
% - Give the names in the same order as the appear in the paper.
% - Use the \inst{?} command only if the authors have different
%   affiliation.

\institute[Carnegie Mellon University] % (optional, but mostly needed)
{
  %
  Department of Computer Science\\
	Carnegie Mellon University
}
% - Use the \inst command only if there are several affiliations.
% - Keep it simple, no one is interested in your street address.

\date{Sep 2018}
% - Either use conference name or its abbreviation.
% - Not really informative to the audience, more for people (including
%   yourself) who are reading the slides online

\subject{Theoretical Computer Science}
% This is only inserted into the PDF information catalog. Can be left
% out. 

% If you have a file called "university-logo-filename.xxx", where xxx
% is a graphic format that can be processed by latex or pdflatex,
% resp., then you can add a logo as follows:

% \pgfdeclareimage[height=0.5cm]{university-logo}{university-logo-filename}
% \logo{\pgfuseimage{university-logo}}

% Delete this, if you do not want the table of contents to pop up at
% the beginning of each subsection:
\AtBeginSubsection[]
{
  \begin{frame}<beamer>{Outline}
    \tableofcontents[currentsection,currentsubsection]
  \end{frame}
}

% Let's get started
\begin{document}

\begin{frame}
  \titlepage
\end{frame}

\begin{frame}{Outline}
  \tableofcontents
  % You might wish to add the option [pausesections]
\end{frame}

% Section and subsections will appear in the presentation overview
% and table of contents.
\section{Cost Semantics}

\begin{frame}{Resource Aware ML}
  Extension of RaML to deal with garbage collected programs. 
\end{frame}

\begin{frame}{Cost Semantics}
  \begin{itemize}
  \item Natural semantics
  \item Environment + heap + freelist
  \item Reachability
    \end{itemize}

    \begin{center}
      \includegraphics[scale=0.5]{memory}
      \end{center}
\end{frame}

\begin{frame}{Natural Semantics}
  \begin{itemize}
  \item Variables\; $\ms{Var} ::= x_1,x_2,...$
    \item Locations\; $\ms{Loc} ::= l_1,l_2,...$
    \item Values\; $\ms{Val} ::= \irl{T} ,\irl{F}, \mathbb{N}, \ms{Loc}, \pairexcst{v_1}{v_2} $
    \item Stacks: $\ms{Var} \to \ms{Val}$
    \item Heaps: $\ms{Loc} \to \ms{Val}$
    \end{itemize}

    \[V,H \vdash e \Downarrow v,H'\] Under stack $V$ heap $H$, $e$ evaluates to
    value $v$ and heap $H'$
\end{frame}
  
\begin{frame}{Reachability}
 

  \begin{mathpar}
\inferrule{
	A = reach_H(v_1)\\
	B = reach_H(v_2)
}{
	A \uplus B = reach_H(\pairexcst{v_1}{v_2}) 
} 

\inferrule{
	A = reach_H(H(l))\\
}{
	\{l\} \uplus A = reach_H(l)
} 

\inferrule{
	v \in \mathbb{N} \cup \{\irl{T},\irl{F},\irl{Null}\}
}{
	\emptyset = reach_H(v)
} 
\end{mathpar}

For a program state $(V, H,e)$, associate to it the \emph{reachable set}:
\begin{align*}
  &locs_{V,H}(e) = \biguplus\limits_{x \in FV(e)} reach_H(V(x))
\end{align*}
Cost of a program could be defined as the \emph{maximum} size of
the reachable set for any $(V,H,e)$ in execution. 
\end{frame}

\begin{frame}{Previous Work}
  \begin{itemize}
  \item Spoonhower, Blelloch, Harper: cost semantics for parallel programs
    \item Minamide: cost semantics for cbv lambda calculus
    \end{itemize}
  \end{frame}

\begin{frame}{Leaf Saturation +  Aggregation}
  Reachable at the leaf:
        \[
	\inferrule{
		V(x) = v
		}{
			V,H,R \vdash x \Downarrow^{|dom(R \uplus reach_H(v))|} v,H
			}
         \]
Aggregate:
\[
	\inferrule{
		V,H,R \uplus locs_{V,H}(x.e_2) \vdash e_1 \Downarrow^{s_1} v_1,H_1\\
		V[x \mapsto v_1],H,R \vdash e_2 \Downarrow^{s_2} v_2,H_2\\
	}{
		V,H,R \vdash \irl{let}(e_1; x : \tau.e_2) \Downarrow^{\max{s_1,s_2}} v_2,H_2
	}
      \]
      
  \end{frame}

\begin{frame}{Freelist}
  \begin{itemize}
  \item Problem: cost accounted for on use site instead of creation
  \item Solution: add persistency through freelists
  \end{itemize}
The garbage collection cost semantics \gcSem{} is defined by a collection of judgement of the form
\[
V,H,R,F \; \vdash e \Downarrow v, H', F'
\]

Under stack $V$, heap $H$, continuation set $R$,
freelist $F$,  the $e$ evaluates to  $v$, 
new heap $H'$ and freelist $F'$.
\end{frame}

\begin{frame}{Judgments}
\begin{mathpar}
\inferrule
{ V(x) = v\\
}
{V,H,R,F \; \vdash x \Downarrow v,H,F}(\text{Var})

\inferrule{
	V_1 = V\restriction_{FV(e_1)}\\
  R' = R \cup locs_{V,H}(\lambda x : \tau.e_2)\\
  V_1,H,R',F \vdash e_1 \Downarrow v_1,H_1,F_1\\
	V_2 = (V[x \mapsto v_1])\restriction_{FV(e_2)}\\
  g = \{ l \in H_1 \mid l \notin F_1 \cup R \cup locs_{V_2,H_1}(e_2) \}\\
  V_2,H_1,R, F_1 \cup g \vdash e_2 \Downarrow v_2,H_2,F_2 \\
}{
  V,H,R,F \; \vdash \irl{let}(e_1; x : \tau.e_2) \Downarrow v_2,H_2,F_2
}(\text{Let})
\end{mathpar}

\end{frame}

\section{Type System}

\begin{frame}{First Order Types}
  \[
\begin{array}{r l l l c r l l l}
  \ms{BTypes} & A \,\,\,\,\, ::= &\hspace{8em}     &  &\\
	& \irl{nat}                	 			& \irl{nat}										\\
	& \unittyabt                	 			& \unittycst								\\
  & \booltyabt                       & \booltycst                \\
              & \prodtyabt{A_1}{A_2}       & \prodtycst{A_1}{A_2}\\
  	&\irl{list}^p(A)		& L^p(A)\\

 \ms{FTypes} & \rho \,\,\,\,\, ::= \\
              &\irl{arr}(A_1;A_2;p;q) 				& A_1 \xrightarrow{p/q} A_2\\ 	
\end{array}
\]

\end{frame}

\begin{frame}{Structural Values}

\begin{align*}
	\denote{\unittyabt} &= \{\val{\irl{Null}}\}\\
	\denote{\booltyabt} &= \{\val{\irl{T}}, \val{\irl{F}}\}\\
	\denote{\irl{nat}} &= \mathbb{N}\\
\pairexcst{a_1}{a_2} &\in \denote{\prodtycst{A_1}{A_2}} 
	\text{ if } a_1 \in \denote{A_1} \text{ and } a_2 \in \denote{A_2}\\
\nilexcst &\in \denote{L(A)}\\
\consexcst{a}{l} &\in \denote{L(A)} \text{ if } a \in \denote{A} \text{ and } l \in \denote{L(A)}\\
\end{align*}
\end{frame}

\begin{frame}{Linear Potential}
\begin{align*}
	&\Phi_H(v : A) = 0 \text{ if } A \in \{\unittycst, \booltyabt, \irl{nat}\}\\
&\Phi_H(\pairexcst{v_1}{v_2} : \prodtycst{A_1}{A_2}) = \Phi_H(v_1 : A_1) + \Phi_H(v_2 : A_2)\\
	&\Phi_H(l : L^p(A)) = p + \Phi_H(v_h : A) + \Phi_H(v_T : L^p(A)) \text{ if } 
		H(l) = \pairexcst{v_h}{v_h}
	%p\cdot n + \sum_{1 \le i \le n} \Phi_H(a_i : A)  
\end{align*}
%
Write $\Phi_{V,H}(\Gamma)$ for $\Sigma_{x \in dom(V)} \Phi_H(V(x) : \Gamma(x))$.\\
Now define $A \curlyvee A_1,A_2,n$ as the sharing relation for resource-annotated types:
\begin{align*}
	&L^p(A) \curlyvee^n L^q(A_1),L^r(A_2) & \text{if } p = q + r + n \;\text{and}\; 
			A \curlyvee^n A_1,A_2\\
	&\prodtycst{A}{B} \curlyvee^n \prodtycst{A_1}{B_1}, \prodtycst{A_2}{B_2}
		&\text{ if } A \curlyvee^n A_1,A_2 \text{ and } B \curlyvee^n B_1,B_2\\
	&A \curlyvee^n  A,A& \text{ if } A \in \{\unittycst, \booltycst, \irl{nat}\}
\end{align*}
\end{frame}

\begin{frame}{Type Rules}
  \begin{mathpar}
\inferr{
  \Sigma; \Gamma, x : L^p(A) \sststile{q'}{q} \listcaseexcst{x}{e_1}{x_h}{x_t}{e_2} : B
}{
  \Sigma; \Gamma \sststile{q'}{q} e_1 : B \\
  \Sigma; \Gamma, x_h : A, x_t : L^p(A) \sststile{q'}{q + p + {\color{red}{1}}} e_2 : B
}(\text{L:MatL})

\inferr{
  \Sigma; \Gamma, x : A \sststile{q'}{q} \sharecpcst{x}{x_1}{x_2}{e} : B
}{
  A \;\curlyvee^{\color{red}{1}} \;A_1, A_2\\
  \Sigma; \Gamma, x_1 : A_1, x_2 : A_2 \sststile{q'}{q} e : B
}(\text{L:Share})

  \end{mathpar}
\end{frame}

\begin{frame}[fragile]{Example}
  \begin{verbatim}
let rec append (l1, l2) =
  match l1 with
  | [] -> l2
  | x::xs -> x::(append (xs, l2))

let appTwice l = 
    share l as l1,l2 in
    let l1' = append (l1, []) in 
    let l2' = append (l2, []) in 
    (l1',l2')
  \end{verbatim}
\end{frame}


\begin{frame}{Old bounds}
\begin{align*}
	&\texttt{append} : \prodtycst{L^p(\irl{int})}{L^q(\irl{int})} \xrightarrow{r/r} L^s(\irl{int}), \text{ 
	where } p \ge s + 1 \text{ and } q \ge s\\
	&\texttt{appTwice} : L^p(\irl{int}) \xrightarrow{q/q} \prodtycst{L^r(\irl{int})}{L^s(\irl{int})},
	\text{ where } p \ge r + s + 2
\end{align*}

\end{frame}

\begin{frame}{New Bounds}
  \begin{align*}
	&\texttt{append} : \prodtycst{L^p(\irl{int})}{L^q(\irl{int})} \xrightarrow{r/r} L^s(\irl{int}),
	\text{ where } {\color{green}{p \ge s}} \text{ and } q \ge s\\
	&\texttt{appTwice} : L^p(\irl{int}) \xrightarrow{q/q} \prodtycst{L^r(\irl{int})}{L^s(\irl{int})},
	\text{ where } p + q \ge r + s + {\color{green}{1}}
\end{align*}
\end{frame}

\begin{frame}[fragile]{Results}
\begin{table}[]
	\begin{adjustbox}{width=1\textwidth, center=\textwidth}
  \begin{tabular}{@{}llllllllll@{}}
    \toprule
		 & \multicolumn{3}{l}{heap metric} & \multicolumn{3}{l}{gc metric} &  \\ \midrule
		 \textit{\textbf{function}} & computed bound & constraints & time & computed bound &  constraints & time & optimal \\
		 quicksort & $1.00 + 3.50M + 1.50M^2$ & 8515 & 0.52 & 0 & 8519 & 0.48 & 0 \\
		 mergesort & $1.00 - 4.67M + 6.33M^2$ & 9572 & 0.64 & $-0.50M + 0.50M^2$ & 9578 & 0.58 & $\lfloor \log(M) \rfloor$ \\
		 ocamlsort & $7.50 + 5.50M + 1.00M^2$ & 8565 & 0.51 & $1.00 + 1.00M$ & 8573 & 0.50 & $M+1$ \\
		 selection sort & $2.00 + 3.00M + 1.00M^2$ & 639 & 0.06 & 0 & 642 & 0.05 & 0 \\
		 eratosthenes & $1.00 + 1.50M + 0.50M^2$ & 515 & 0.06 & 0 & 517 & 0.04 & 0 \\
		 dfs & $3.00 + 2.00M$ & 5481 & 0.90 & 2 & 5483 & 0.36 & 2 \\
		 bfs & $5.00 + 10.00M$ & 24737 & 4.15 & 4 & 24742 & 1.62 & 4 \\
		 transpose & $1.00 + 3.50LM + 0.50LM^2$ & 10680 & 0.50 & $1.00 + 2.00LM$ & 10684 & 0.50 & $\max{(0,2L-1)}$ \\
		 map\_it & $2.00 + 2.00LM + 4.00M$ & 30699 & 1.58 & $1.00M + 1.00$ & 30703 & 1.57 & M + 1 \\
		 pairs & $1.00 + 1.00M^2$ & 10214 & 0.60 & $0.50M + 0.50M^2$ & 10217 & 0.64 & $0.5M^2-1.5M+2$ \\ \midrule
	\end{tabular}
	\end{adjustbox}
	\caption{Automatic Bound Analysis with RaML}
\label{table:bounds}
\end{table}
\end{frame}


\section{Soundness}
\begin{frame}{Soundness}
\begin{theorem}[Soundness]
\label{itm:soundness} Let $H \vDash V {:} \Gamma$, $\;\; \Sigma; \Gamma \sststile{q'}{q} e : B$,
and $V,H \vdash e \Downarrow v, H'$.
Then for all configurations $W,Y,F,R$:
If $V,H \sim W,Y$ and $|F| \ge \Phi_{V,H}(\Gamma) + q$,
then there exists a value $w$, and a freelist $F'$ such that
$$
\begin{array}{ccc}
	W,Y,R,F \vdash^{\gcSem} e \Downarrow w, Y', F'  & \text{ and } & \veq{H'}{Y'}{v}{w} \; .
\end{array}
$$
\end{theorem}
The theorem states that,
given a terminating expression
and a freelist that is sufficiently large (as predicated by the type derivation), 
a run with \gcSem will normalize to an equivalent value.
\end{frame}

\begin{frame}{Proof Sketch}
  To facilitate the proof, we define an intermediate semantics
\copySem{} which is semantically linear. The proof has two stages:
First, we show \copySem{} over-approximates \gcSem, meaning that any
computation that succeeds with \gcSem{} will succeed with an
equally-sized or smaller freelist with \gcSem{}. Then we show $\fogc$
is sound with respect to \copySem{}, and thus by the previous step sound with respect to
\gcSem{}.
The chart below shows the plan for proving the soundess of the type 
system w.r.t \gcSem:

\tikzstyle{process} = [rectangle, text centered, draw=black, fill=orange!30]
\tikzstyle{arrow} = [thick,->,>=stealth]

\begin{center}
\begin{tikzpicture}[node distance=2cm]
\node (type) [process] {Type Rules};
\node (copy) [process, right of=type, xshift=3cm] {\copySem};
\node (free) [process, below of=copy] {\gcSem};

\draw [arrow] (type) -- node[anchor=south] {sound w.r.t (\ref{itm:soundness})} (copy);
\draw [arrow] (copy) -- node[anchor=west] {Over-approximates (\ref{itm:frugal})} (free);
\draw [dashed,->] (type) -- node[anchor=south] {} (free);

\end{tikzpicture}
\end{center}
\end{frame}

\begin{frame}{Linear Semantics}
The mediating semantics \copySem{} is \emph{semantically linear}; that is, no location can be 
used twice in a program. Variable sharing is achieved via \emph{copying}: 
the shared value is created by allocating a fresh set of locations from the freelist and
copying the value:

{ \small \begin{mathpar}
	\inferr{
		H,v = copy(H,L,v)	
	}{
		v \in \{n,\irl{T},\irl{F},\irl{Null}\}
	}

	\inferr{
		H'\{l' \mapsto v\},l' = copy(H,L,l) 
	}{
		l' \in L\\
		H',v = copy(H,L \setminus \{l'\},H(l)) 
	}

\inferr{
	H_2,\pairexcst{v_1'}{v_2'} = copy(H,L,\pairexcst{v_1}{v_2})
}{
	L_1 \sqcup L_2 \subseteq L\\
	|L_1| = |dom(reach_H(v_1)|\\
	|L_2| = |dom(reach_H(v_2)|\\
	H_1,v_1' = copy(H,L_1,v_1)\\
	H_2,v_2' = copy(H_1,L_2,v_2)\\
}
\end{mathpar}}
\end{frame}

\begin{frame}{Copy Sharing}
  \small \[
\inferrule{
	V(x) = v'\\
  L \subseteq F\\
  |L| = |dom(reach_H(v'))|\\
  H',v'' = copy(H,L,v')\\
	V' = (V[x_1 \mapsto v',x_2 \mapsto v'']) \restriction_{FV(e)}\\
	F' =	F \setminus L\\
	g = \{l \in H \mid l \notin F' \cup R \cup locs_{V',H}(e)\}\\
 	V',H',R,F' \sqcup g \; \vdash e \Downarrow v,H'',F''
}{
  V,H,R,F \; \vdash \sharecpcst{x}{x_1}{x_2}{e} \Downarrow v,H'',F''
}
\]
\end{frame}

\begin{frame}{Proof Attempt}
  \begin{attempt}
	Let $\mathcal{C}_2 = (V,H,R,F)$ be a configuration and $(\mathcal{C}_2, e)$ 
	be a linear computation. Given that 
	$\mathcal{C}_2 \vdash^{\mathsf{copy}} e \Downarrow v,H',F'$, 
	for all configurations $\mathcal{C}_1 = (W,Y,R,M)$ such that $W,Y \sim V,H$ and $|M| = |F|$,
there is exists a triple
$(w,Y',M') \in \ms{Val} \times \ms{Heap} \times \ms{Loc}$ such that
	\begin{center}
			$\mathcal{C}_1 \vdash^{\mathsf{free}} e \Downarrow w,Y',M'$
		and 	 $\veq{H'}{Y'}{v}{w}$
		and 	 $|M'| \ge |F'|$ .
	\end{center}
\end{attempt}
\end{frame}

\begin{frame}{Contention}
We proceed with induction on the derivation of the judgment in \copySem. 
Almost every case goes through, save for E:Let. 
First, we get $W_1,Y \sim V_1,H$ and We have the following from induction on the first premise:
%
\begin{center}
	 $W_1,Y,R',M \vdash^{\mathsf{free}} e \Downarrow w_1,Y_1,M_1$ and, 	 $\veq{H_1}{Y_1}{v_1}{w_1}$
         and,  $|M_1| \ge |F_1|$
\end{center}
%
To instantiate the induction hypothesis on the second premise, we need to show that, among 
other things, $|M_1 \cup j| \ge |F_1 \cup g|$, where $j$ is the set of collected locations in 
the \gcSem{} judgment. We cannot show this precisely because $g$ might contain more cells 
then $j$ due to the linearity of \copySem{}, thus preventing a piecewise comparison. 
But of course $|j|$ is always less than $|g|$, since \gcSem{} doesn't copy to share 
values! This shows that there is a mismatch between the induction hypothesis and the relationship
between the sizes of the respective freelists and the garbage sets. 

\end{frame}

\begin{frame}{\copySem{} over-approximates \gcSem}
  Idea: maintain a mapping $\gamma$ from the active heap used in the GC semantics
  $H_{gc}\setminus F_{gc}$ to \emph{subsets} of
the heap used in the copying semantics $H_{copy} \setminus F_{copy}$ such that the image of 
$H_{gc}\setminus F_{gc}$ under $\gamma$ forms a partition on $H_{copy} \setminus F_{copy}$.
The intuition is that given a cell $l \in H_{gc}$, there must be multiple cells
$\gamma(l)$ in  $H_{copy}\setminus F_{copy}$  
that are were allocated during sharing, and thus ``morally the same'' as $l$.

\begin{center}
\includegraphics[scale=0.4]{relation}
\end{center}
\end{frame}

\begin{frame}{Well-formed configuration}
  \begin{definition} A configuration $(V,H,R,F)$ is well-formed if 
	\begin{enumerate}
		\item $dom(H) \subseteq reach_H(V) \cup R \cup F$
		\item $reach_H(V) \cup R \subseteq dom(H) \setminus F$
		\item $collect(R,reach_H(V),H,F) = \emptyset$
	\end{enumerate}
\end{definition}
\end{frame}

\begin{frame}{Subvalues}
  \begin{definition}
	Let \ms{dir} be the set \{\ms{L},\ms{R},\ms{N}\}, denoting left, right, and next 
	respectively. We can index values via directions:
        $$
	\begin{array}{lclclcl}
		get_H(Just(\pairexcst{v_1}{v_2},\ms{L})) &=& Just(v_1) & \hspace{4em} &
		get_H(Just(\pairexcst{v_1}{v_2},\ms{R})) &=& Just(v_2)\\
		get_H(Just(\pairexcst{v_1}{v_2}),_) &=& None &&
		get_H(Just(l),\ms{N}) &=& Just(H(l)) \\
		get_H(Just(l), \_) &=& None &&
		get_H(r,\_) &=& r
	\end{array}
        $$
	Let $P$ be a sequence of directions. Extend $get$ to sequence of directions:
	\begin{align*}
		find_H(v,D::P) &= find_H(get_H(v,D),P)\\
		find_H(v,[]) &= v
	\end{align*}
	Call $P$ valid w.r.t a value $v$ if $find_H(v,P) = Just (v')$ for some $v'$.
	Write $V_H(x;P)$ for $fromJust(find_H(V(x),P))$ given a valid sequence $P$ w.r.t $V(x)$,
	and $reach_H(V(x;P))$ for $reach_H(V_H(x;P))$.
\end{definition}
\end{frame}


\begin{frame}{Copy Extension}
\begin{definition}
A well-formed configuration $\mathcal{C}_2 = (V_2,H_2,R_2,F_2)$ is a \emph{copy extension} of another well-formed configuration
$\mathcal{C}_1 = (V_1,H_1,R_1,F_1)$ iff
\begin{enumerate}
\item $V_1,H_1 \sim V_2,H_2$
\item There is a proper partition $\gamma : dom(H_1) \setminus F_1 \to \mathcal{P}(dom(H_2) \setminus F_2)$ 
such that for all $l \in dom(\gamma)$, $|\gamma(l)| = reach_{H_1}(V_1)(l) + R_1(l)$
\item For all $l \in dom(\gamma)$, $x \in dom(V_1)$, valid sequence of directions $P$ w.r.t $V_1(x)$,
	$|reach_{H_2}(V_2(x;P)) \cap \gamma(l)| = reach_{H_1}(V_1(x;P))(l)$.
\item	For all $l \in dom(\gamma)$, $|\gamma(l) \cap R_2| = R_1(l)$
\item $|F_1| = |F_2| + |\oh{\gamma}|$, where 
	$\oh{\gamma} = \bigcup_{P \in ec(\gamma)} P \setminus (rep(P))$
\end{enumerate}
Write this as $\mathcal{C}_1 \preceq \mathcal{C}_2$.
\end{definition} 

The intention is that $\mathcal{C}_2$ is a configuration for initiating an evaluation using \copySem
, and $\mathcal{C}_1$ a configuration for \gcSem. 
\end{frame}

\begin{frame}{Extension Lemma}
  \begin{lemma}\label{itm:frugal}
	Let $(\mathcal{C}_2,e)$ be a linear computation. Given that 
	$\mathcal{C}_2 \vdash^{\mathsf{copy}} e \Downarrow v,H',F'$,
	for all well-formed configurations $\mathcal{C}_1$ such that $\mathcal{C}_1 \preceq \mathcal{C}_2$,
there is exists a triple
$(w,Y',M') \in \ms{Val} \times \ms{Heap} \times \ms{Loc}$ and 
	$\gamma' : dom(Y') \setminus M' \to \mathcal{P}(dom(H') \setminus F')$ s.t.
	\begin{enumerate}
			\item $\mathcal{C}_1 \vdash^{\mathsf{free}} e \Downarrow w,Y',M'$
			\item $\veq{H'}{Y'}{v}{w}$
			\item $\gamma'$ is a proper partition, such that for all $l \in dom(\gamma')$, 
				$|\gamma'(l)| = |reach_{Y_1}(w_1)(l)| + S(l)$
			\item For all $P$, $|reach_{H'}(find_{H'}(v;P)) \cap \gamma'(l)| = 
				reach_{Y'}(find_{Y'}(w;P))(l)$
			\item For all $l \in dom(\gamma')$, $\gamma'(l) \cap R = \gamma(l) \cap R$
			\item $|M'| = |F'| + |\oh{\gamma'}|$
	\end{enumerate}
\end{lemma}
\end{frame}

\begin{frame}{Operational Semantics}
  To state the soundness theorem, we need an operational semantics to witness terminating expressions.
  Just take $\mathcal{E}_{\ms{oper}}$ the normal natural semantics with
  rules of the form $V,H \vdash e \Downarrow v,H$.\\
  
  In words, the theorem states that given a terminating expression (as shown by a run with
  $\mathcal{E}_{\ms{oper}}$),
and given a freelist that is sufficiently large (as predicated by the type derivation), 
a run with \copySem will normalize to an equivalent value, and the resulting freelist 
will be sufficiently large (as predicated by the type derivation).
\end{frame}

\begin{frame}{Linear Context}
First, we characterize semantically linear contexts: 
\begin{definition}[Linear context]
Given a context $(V,H)$, let
$x,y \in dom(V)$, $x \ne y$, and $r_x = reach_H(V(x))$, $r_y = reach_H(V(y))$.
	It is \emph{linear} given that  $\ms{set}(r_x)$, $\ms{set}(r_y)$, and $r_x \cap r_y = \emptyset$.
      \end{definition}
      
Where $\ms{set}(S)$ means $S$ a proper set ($S(x) \le 0 \forall x$). 
Denote this by $\na{V,H}$. Whenever $\na{V,H}$ holds,
there is at 
most one path from a variable on the stack $V$ to any location in $H$. 
\end{frame}

\begin{frame}{Linearity of \copySem}
Now we can 
formalize our intuition for linear computations: 
\begin{definition}[Linear computation]
Given a configuration $\mathcal{C} = (V,H,R,F)$ and an expression $e$, 
we say the 5-tuple $(\mathcal{C},e)$ is a \emph{computation}; it is a \emph{linear computation} 
given the that  $dom(V) = FV(e)$, $\na{V,H}$, and $\dist{\{R,F,locs_{V,H}(e)\}}$.
And we write $\wfc{V}{H}{R}{F}{e}$ to denote this fact.
\end{definition}

% main lemma
Given a semantically linear computation, the resulting value is linear: 
\begin{lemma}[Linearity of \copySem]\label{itm:na}
For all stacks $V$ and heaps $H$, let  $V,H,R,F \; \vdash e \Downarrow v, H', F'$ 
and $\Sigma; \Gamma \vdash e : B$. Then given that $\wfc{V}{H}{R}{F}{e}$, we have that $\ms{set}(reach_{H'}(v))$ and $\dist{\{R,F',reach_{H'}(v)\}}$.
\end{lemma}
\end{frame}

\begin{frame}{Soundness of \copySem}
\begin{theorem}[Soundness]
\label{b} let $H_o \vDash V_o : \Gamma$, $\Sigma; \Gamma \sststile{q'}{q} e : B$,
$V_o,H_o \; \vdash e \Downarrow v_o, H_o'$.
Then $\forall C \in \mathbb{Q}^{+}$ and configuration $V,H,R,F$ s.t.
\begin{enumerate} 
\item $V_o,H_o \sim V,H$
\item $\wfc{V}{H}{R}{F}{e}$
\item $|F| \ge \Phi_{V,H}(\Gamma) + q + C$ 
\end{enumerate}
then there exists a triple $(v,H',F')$, and a freelist $F'$ s.t.
\begin{enumerate}
  \item $V,H,R,F \vdash e \Downarrow v, H', F'$
	\item $\veq{H_o'}{H'}{v_o}{v}$
  \item $|F'| \ge \Phi_{H'}(v:B) + q' + C$
\end{enumerate}
\end{theorem}
\end{frame}

  
% Placing a * after \section means it will not show in the
% outline or table of contents.
\section*{Summary}

\begin{frame}{Summary}
  \begin{itemize}
  \item New cost semantics for garbage collection of first-order functional programs
  \item Extension of type based amortized analysis to derive better space bounds
  \end{itemize}
  
  \begin{itemize}
  \item Questions:
    \begin{itemize}
    \item Extension to higher-order functions
    \item Type rules for accounting for non-local garbage collection;
      variable sharing vs context sharing
    \end{itemize}
  \end{itemize}
\end{frame}



% All of the following is optional and typically not needed. 

\end{document}


