\documentclass[11pt]{article}
\usepackage{fullpage}
\usepackage{latexsym}
\usepackage{verbatim}
\usepackage{amsthm}
\usepackage{amssymb}
\usepackage{amsmath}
\usepackage{stackengine}
\usepackage{scalerel}
\usepackage{code,proof,amsthm,amssymb, amsmath}
\usepackage{proof}
\usepackage{mathpartir}
\usepackage{turnstile}

\input{generic-defns}
\input{syn-defns}
\newcommand{\FunEvalsTo}[3]{{#1}\mathrel{\Downarrow'}\AbsABT{#2}{#3}}
\newcommand{\FunEvalsToHO}[3]{{#2}\vargen \fappabt{#1}{#2}\evalsto{#3}}

\newcommand{\arrtyabt}[2]{\OpABT{\cd{arr}}{#1;#2}}
\newcommand{\arrtycst}[2]{{#1}\mathrel{\to}{#2}}
\newcommand{\lamabt}[3]{\OpABTp{\cd{lam}}{#1}{\AbsABT{#2}{#3}}}
\newcommand{\lamcst}[3]{\mathop{\lambda}\cdparens{\TightIsOf{#2}{#1}}\,{#3}}
\newcommand{\appabt}[2]{\OpABT{\cd{ap}}{#1;#2}}
\newcommand{\appcst}[2]{{#1}\cdparens{#2}}

\newcommand{\letabt}[4]{\OpABTp{\cd{let}}{#1}{#2;\AbsABT{#3}{#4}}}
\newcommand{\letcst}[4]{\kwop{let}\TightIsOf{#3}{#1}\kwop{be}{#2}\kwop{in}{#4}}
\newcommand{\uletabt}[3]{\OpABT{\cd{let}}{#1;\AbsABT{#2}{#3}}}
\newcommand{\uletcst}[3]{\kwop{let}{#2}\kwop{be}{#1}\kwop{in}{#3}}

\newcommand{\fdefabt}[6]{\OpABTp{\cd{fun}}{#1;#2}{\AbsABT{#3}{#4};\AbsABT{#5}{#6}}}
\newcommand{\fdefcst}[6]{\kwop{fun}{#5}\TightIsOf{\cdparens{\TightIsOf{#3}{#1}}}{#2}\mathbin{\cd{=}}{#4}\kwop{in}{#6}}
\newcommand{\fappabt}[2]{\OpABTp{\cd{apply}}{#1}{#2}}
\newcommand{\fappcst}[2]{#1\cdparens{#2}}
\newcommand{\fsubst}[3]{\dblsqbracks{#1\mathord{/}#2}{#3}}


%%% Local Variables: 
%%% mode: latex
%%% TeX-master: "book"
%%% End: 

\newcommand{\LangPCF}{\textbsf{PCF}}

\newcommand{\genrecabt}[3]{\OpABTp{\cd{fix}}{#1}{\AbsABT{#2}{#3}}}
\newcommand{\genreccst}[3]{\kwop{fix}{#2}\cdcolon{#1}\kwop{is}{#3}}
\newcommand{\sgenreccst}[2]{\kwop{fix}{#1}\kwop{is}{#2}}

\newcommand{\bddrecabt}[4]{\OpABTp{\cd{fix}^{#1}}{#2}{\AbsABT{#3}{#4}}}
\newcommand{\bddreccst}[4]{\mathop{\cd{fix}^{#1}}{#3}\cdcolon{#2}\kwop{is}{#4}}

\newcommand{\parrtyabt}[2]{\OpABT{\cd{parr}}{#1;#2}}
\newcommand{\parrtycst}[2]{{#1}\mathbin{\rightharpoonup}{#2}}
\newcommand{\pappabt}[2]{\OpABT{\cd{ap}}{#1;#2}}
\newcommand{\pappcst}[2]{\appcst{#1}{#2}}

\newcommand{\funabt}[5]{\OpABTp{\cd{fun}}{#1;#2}{\AbsABT{#3}{\AbsABT{#4}{#5}}}}
\newcommand{\funcst}[5]{\kwop{fun}{#3}\cdparens{{#4}{\cdcolon}{#1}}{\cdcolon}{#2}\kwop{is}{#5}}

\newcommand{\empenv}{\bullet}
\newcommand{\extenv}[3]{{#1},{#2}{=}{#3}}

\newcommand{\expclo}[1]{\widehat{#1}}
\newcommand{\expclois}[2]{\expclo{#1}{=}{#2}}
\newcommand{\expenv}[2]{\widehat{#1}\parens{#2}}
\newcommand{\expenvis}[3]{\expenv{#1}{#2}{=}{#3}}

\newcommand{\cutoff}[2]{{#1}^{(#2)}}

%\newcommand{\Replace}[3]{\sqbracks{#2\mathbin{\leftarrow}#1}{#3}}

\newcommand{\lnattyabt}{\cd{lnat}}
\newcommand{\lnattycst}{\lnattyabt}
\newcommand{\lsuccabt}[2]{\OpABT{\cd{succ}}{\AbsABT{#1}{#2}}}
\newcommand{\lsucccst}[2]{{#1}\kwbin{is}\succcst{#2}}

%%% Local Variables: 
%%% mode: plain-tex
%%% TeX-master: "book"
%%% End: 

\newcommand{\unittyabt}{\kw{unit}}
\newcommand{\unittycst}{\kw{unit}}
\newcommand{\unittycstm}{\top}
\newcommand{\prodtyabt}[2]{\OpABT{\kw{prod}}{#1;#2}}
\newcommand{\prodtycst}[2]{{#1}\times{#2}}

\newcommand{\pprodtycst}[2]{{#1}\otimes{#2}}
\newcommand{\ppairexabt}[2]{\OpABT{\kw{fuse}}{#1;#2}}
\newcommand{\ppairexcst}[2]{{#1}\otimes{#2}}
\newcommand{\splitexabt}[4]{\OpABT{\kw{split}}{{#1};\AbsABT{#2,#3}{#4}}}
\newcommand{\splitexcst}[4]{\kw{split}\,{#1}\,\kw{as}\,\ppairexcst{#2}{#3}\,\kw{in}\,{#4}}
\newcommand{\checkexabt}[2]{\OpABT{\kw{check}}{#1;#2}}
\newcommand{\checkexcst}[2]{\kw{check}\,{#1}\,\kw{as}\,\unitexcst\,\kw{in}\,{#2}}

\newcommand{\unitexabt}{\kw{triv}}
\newcommand{\unitexcst}{\langle\rangle}
\newcommand{\pairexabt}[2]{\OpABT{\kw{pair}}{#1;#2}}
\newcommand{\pairexcst}[2]{\langle #1, #2\rangle}
\newcommand{\projexabt}[2]{\OpABT{\OpInst{\kw{pr}}{#2}}{#1}}
\newcommand{\fstexabt}[1]{\projexabt{#1}{\kw{l}}}
\newcommand{\sndexabt}[1]{\projexabt{#1}{\kw{r}}}
\newcommand{\projexcst}[2]{{#1}\mathbin\cdot{#2}}
\newcommand{\fstexcst}[1]{\projexcst{#1}{\kw{l}}}
\newcommand{\sndexcst}[1]{\projexcst{#1}{\kw{r}}}

\newcommand{\dgenprodcst}[1]{\brackets{#1}}
\newcommand{\dgentuplecst}[1]{\brackets{#1}}
\newcommand{\genprodabt}[3]{\OpABT{\kw{prod}}{\genff{#1}{#2}{#3}}}
\newcommand{\vargenprodcst}[3]{\prod_{{#2}\in{#1}}{#3}}
\newcommand{\genprodcst}[3]{\dgenprodcst{#3}_{#2\in #1}}
\newcommand{\gentupleabt}[3]{\OpABT{\kw{tpl}}{\genff{#1}{#2}{#3}}}
\newcommand{\gentuplecst}[3]{\dgentuplecst{#3}_{{#2}\in{#1}}}
\newcommand{\lgentuplescst}[3]{\gentuplecst{#1}{#2}{\fldexcst{#2}{#3}}}
\newcommand{\genprojabt}[3]{\projexabt{#3}{#2}}
\newcommand{\genprojcst}[3]{\projexcst{#3}{#2}}
\newcommand{\genprodcstgen}[3]{\dgenprodcst{\varexplff{#1}{#2}{#3}}}
\newcommand{\gentuplecstgen}[3]{\dgentuplecst{\varexplff{#1}{#2}{#3}}}

\newcommand{\rcdtycst}[1]{\dgenprodcst{#1}}
\newcommand{\rcdexcst}[1]{\dgentuplecst{#1}}
\newcommand{\fldtycst}[2]{\pairff{#1}{#2}}
\newcommand{\fldexcst}[2]{\pairff{#1}{#2}}
\newcommand{\fldselexcst}[2]{\projexcst{#2}{#1}}


%%% Local Variables: 
%%% mode: latex
%%% TeX-master: "book"
%%% End: 

\input{../pfpl/sum-defns}
\input{../pfpl/icoi-defns}
\newcommand{\LangT}{\textbsf{T}}

\newcommand{\nattyabt}{\cd{nat}}
\newcommand{\nattycst}{\nattyabt}

\newcommand{\zeroabt}{\cd{z}}
\newcommand{\zerocst}{\zeroabt}
\newcommand{\succabt}[1]{\OpABT{\cd{s}}{#1}}
\newcommand{\succcst}[1]{\succabt{#1}}

\newcommand{\predabt}[1]{\OpABT{\cd{p}}{#1}}
\newcommand{\predcst}[1]{\predabt{#1}}

\newcommand{\numeral}[1]{\overline{#1}}

\newcommand{\natrecabt}[6]{\OpABTp{\kw{rec}}{#3;\AbsABT{#4}{\AbsABT{#5}{#6}}}{#2}}
\newcommand{\natreccst}[6]{\kwop{rec}{#2}\,\cdbraces{\zerocst\casebrcst{#3}\casesepcst\succcst{#4}\kwop{with}{#5}\casebrcst{#6}}}

\newcommand{\natiterabt}[5]{\OpABTp{\cd{iter}}{#3;\AbsABT{#4}{#5}}{#2}}
\newcommand{\natitercst}[5]{\kwop{iter}{#2}\,\cdbraces{\zerocst\casebrcst{#3}\casesepcst\succcst{#4}\casebrcst{#5}}}

\newcommand{\natcaseabt}[4]{\OpABTp{\cd{ifz}}{#2;\AbsABT{#3}{#4}}{#1}}
\newcommand{\natcasecst}[4]{\kwop{ifz}{#1}\,\cdbraces{\zerocst\casebrcst{#2}\casesepcst\succcst{#3}\casebrcst{#4}}}

\newcommand{\nullstrabt}{\epsilon}
\newcommand{\consstrabt}[2]{{#1}\cdot{#2}}

%%% Local Variables: 
%%% mode: latex
%%% TeX-master: "book"
%%% End: 


% =========================================================================== %
%                                                                             %
%                          Using this LaTeX Template                          %
%                                                                             %
% - new tasks are on their own section (how Gradescope expects them)          %
% - use '\task' to start a new task                                           %
% - use 'begin{task} ... \end{task}' if you'd like to preface your answer     %
%   with the question itself (i.e., fill in the '...' with the question)      %
% - use '\nextgroup' to advance from, for example, Task 1.4 to Task 2.1       %
% - use '\skipaheadtask' to skip from, for example, Task 2.2 to Task 2.4      %
%                                                                             %
% You also have access to all the definitions from the handout. See defs.tex, %
% syn-defns.tex, and generic-defns.tex.                                       %
%                                                                             %
%               TODO: Fill in your personal information below!                %
%                                                                             %
% =========================================================================== %
\newcommand{\myname}{Andrew Carnegie}
\newcommand{\myandrewid}{andrew}
\newcommand{\hwnumber}{1}
% =========================================================================== %

\newcounter{group}
\setcounter{group}{1}
\newtheorem{theorem}{Task}[group]
% Remove '\newpage' below to preview your doc compactly.
% Remember to put it back before submitting to Gradescope.
\newcommand{\task}{\newpage\begin{theorem}\end{theorem}}
\newcommand{\nextgroup}{\stepcounter{group}}
\newcommand{\skipaheadtask}{\stepcounter{theorem}}
\newcommand{\ms}[1]{\ensuremath{\mathsf{#1}}}
\newcommand{\irl}[1]{\mathtt{#1}}
\newcounter{rule}
\setcounter{rule}{0}
\newcommand{\rn}
  {\addtocounter{rule}{1}(\arabic{rule})}

\newcounter{infercount}
\setcounter{infercount}{1}
\newcommand{\infern}[2]{\inferrule{#1}{#2}(\text{S}_{\arabic{infercount}}\stepcounter{infercount})}
\newcommand*\ts[2]{%
  \,\scalebox{1}[0.5]{$\sststile[ss]{\textstyle#1}{\textstyle#2}$}\,
}

\title{15-312 Assignment \hwnumber}
\author{\myname\\(\myandrewid)}
\date{\today}

\begin{document}
\maketitle

\section{Syntax}

\[
\begin{array}{r l l l l}
\ms{Type} & \tau \,\,\,\,\, ::= \\
	& \irl{nat}                	 			& \irl{nat}											& \text{naturals}\\
	& \unittyabt                	 			& \unittycst										& \text{unit}\\
  & \booltyabt                       & \booltycst                    & \text{boolean}\\
  & \prodtyabt{\tau_1}{\tau_2}       & \prodtycst{\tau_1}{\tau_2}    & \text{product}\\
	&\irl{arr}(\tau_1;\tau_2) 				& \arrtycst{\tau_1}{\tau_2} 									& \text{function}\\
  &\listtyabt{\tau}		& \listtycst{\tau}											& \text{list}\\
	 \\
\ms{Exp}
        & e   \,\,\,\,\, ::= \\
 	& x                                			& x 												& \text{variable}\\
  & \irl{nat}[n]							& \numeral{n}												& \text{number}\\
  & \irl{unit}							& ()												& \text{unit}\\
  & \irl{T}							& \irl{T}												& \text{true}\\
  & \irl{F}	   					& \irl{F}												& \text{false}\\
  & \ifexabt{x}{e_1}{e_2} & \ifexcst{x}{e_1}{e_2}  & \text{if}\\
  & \irl{lam}(x:\tau.e) 						&\lambda \; x : \tau. e 		& \text{abstraction}\\
  & \irl{ap}(f;x) 					& \appcst{f}{x} 										& \text{application}\\
  & \irl{tpl}(x_1;x_2)     	& \pairexcst{x_1}{x_2}                									& \text{tuple}\\
 	& \irl{fst}(x)					& \fstexcst{x}   										& \text{first projection}\\
 	& \irl{snd}(x)					& \sndexcst{x}   										& \text{second projection}\\
 	& \nilexabt					& []   										& \text{nil}\\
 	& \consexabt{x_1}{x_2}					& x_1::x_2   										& \text{cons}\\
 	& \listcaseexabt{l}{e_1}{x}{xs}{e_2}					& \listcaseexcst{l}{e_1}{x}{xs}{e_2}   	& \text{match list}\\
  & \irl{let}(e_1; x : \tau.e_2)			& \irl{let}\; x = e_1 \; \irl{in}\; e_2   	& \text{let}\\
  \\
\ms{Val}
        & v   \,\,\,\,\, ::= \\
 	& \irl{val}[l](n)                                			& n^l 												& \text{numeric value}\\
 	& \irl{val}[l](\irl{T})                               			& \irl{T}^l 								  & \text{true value}\\
 	& \irl{val}[l](\irl{F})                                			& \irl{F}^l								  & \text{false value}\\
 	& \irl{val}[l](\irl{Null})                                  & \irl{Null}^l 								  & \text{null value}\\
 	& \irl{val}[l](\irl{cl}(V; x.e))                & (V, x.e)^l 					& \text{function value}\\
 	& \irl{val}[l_2](l_1)                                			& l_1^{l_2} 								  & \text{loc value}\\
 	& \irl{val}[l](\pairexabt{v_1}{v_2})                             & \pairexcst{v_1}{v_2}^l 								  & \text{pair value}\\
  \\
\ms{Loc}
        & l   \,\,\,\,\, ::= \\
 	& \irl{loc}(l)                                			& l 												& \text{location}\\
\end{array}
\]

\section{Garbage collection semantics}

Model dynamics using judgement of the form:
\[
\fbox{$V,H,R \; \vdash e \Downarrow^s v, H'$}
\]

Where $V : VID \to Val$, $H : Loc \to Val$, and $R : \{Loc\}$. This can be read as: under stack $V$, heap $H$, and roots $R$,
the expression $e$ evaluates to $v$ using maximum heap space $s$, and engenders a new heap $H'$.\\

Roots represents the set of locations required to compute the continuation \emph{excluding} the current expression.
We can think of roots as the heap allocations necessary to compute the context with a hole that will be filled
by the current expression.\\

Below defines the size of reachable values and space for roots:

\begin{align*}
  &reach_{H}(n^l) = \{l\}\\
  &reach_{H}(\irl{T}^l) = \{l\}\\
  &reach_{H}(\irl{F}^l) = \{l\}\\
  &reach_{H}(\irl{Null}^l) = \{l\}\\
  &reach_{H}((V, x.e)^l) = \{l\} \cup (\bigcup\limits_{y \in FV(e) \setminus x} reach_H(V(y))) \\
  &reach_{H}(l_1^{l_2}) = \{l_2\} \cup loc_{H}(H(l_1))\\
  &reach_{H}(\pairexcst{v_1}{v_2}^l) = \{l\} \cup reach_{H}(v_1) \cup reach_{H}(v_2)\\\\
  &loc_{H}(l) = \{l\} \cup reach_{H}(H(l))\\\\
  &space_{H}(R) = |\bigcup\limits_{l \in R} loc_{H}(l)|\\\\
  &locs_{V,H}(e) = \bigcup\limits_{x \in FV(e)} reach_H(V(x))
\end{align*}

\begin{mathpar}

\infern
{ x \in dom(V)
}
{V,H,R \; \vdash x \Downarrow^{space_H(R \cup (reach_H(V(x))))} V(x),H}

\infern
{
  (l \; \text{fresh})\\
  H' = H[l \mapsto n^l]
}{
  V,H,R \; \vdash \numeral{n} \Downarrow^{space_{H'}(R \cup (\{l\})} n^l,H'
}

\infern{
  (l \; \text{fresh})\\
  H' = H[l \mapsto \irl{T}^l]
}{
  V,H,R \; \vdash \irl{T} \Downarrow^{space_{H'}(R \cup \{l\})} \irl{T}^l,H'
}

\infern{
  (l \; \text{fresh})\\
  H' = H[l \mapsto \irl{F}^l]
}{
  V,H,R \; \vdash \irl{F} \Downarrow^{space_{H'}(R \cup \{l\})} \irl{F}^l,H'
}

\infern{
  (l \; \text{fresh}) \\
  H' = H[l \mapsto \irl{Null}^l]
}{
  V,H,R \; \vdash () \Downarrow^{space_{H'}(R \cup \{l\})} \irl{Null}^l,H'
}

\infern{
  V(x) = \irl{T}^l\\
  V,H,R \; \vdash e_1 \Downarrow^{s_1} v_1, H_1
}{
  V,H,R \; \vdash \ifexabt{x}{e_1}{e_2} \Downarrow^{s_1} v_1, H_1
}

\infern{
  V(x) = \irl{F}^l\\
  V,H,R \; \vdash e_2 \Downarrow^{s_2} v_2, H_2
}{
  V,H,R \; \vdash \ifexabt{x}{e_1}{e_2} \Downarrow^{s_2} v_2, H_2
}

% function

\infern{
  (l \; \text{fresh})\\
  H' = H[l \mapsto (V,x.e)^l]
}{
  V,H,R \; \vdash \irl{lam}(x : \tau.e) \Downarrow^{space_{H'}(R \cup \{l\})} (V, x.e)^l,H'
}

\infern{
  V(f) = (V_1, x.e)^{l_1} \\
  V(x) = v_1 \\
  V_1[x \mapsto v_1], H, R \; \vdash e \Downarrow^s v, H'
}{
  V,H,R \; \vdash \appcst{f}{x} \Downarrow^{s} v,H'
}

% tuples

\infern{
  V(x_1) = v_1 \\
  V(x_2) = v_2 \\
  (l \; \text{fresh})\\
  H' = H[l \mapsto \pairexcst{v_1}{v_2}^l]
}{
  V,H,R \; \vdash \pairexcst{x_1}{x_2} \Downarrow^{space_{H'}(R \cup \{l\})} \pairexcst{v_1}{v_2}^l,H'
}

\infern{
  V(x) = \pairexcst{v_1}{v_2}^l\\
}{
  V,H,R \; \vdash \fstexcst{x} \Downarrow^{space_{H'}(R \cup reach(v_1))} v_1,H
}

\infern{
  V(x) = \pairexcst{v_1}{v_2}^l\\
}{
  V,H,R \; \vdash \sndexcst{x} \Downarrow^{space_{H'}(R \cup reach(v_2))} v_2,H'
}

% lists

\infern{
  (l \; \text{fresh})\\
  H' = H[l \mapsto \irl{Null}^l]
}{
  V,H,R \; \vdash \nilexabt \Downarrow^{space_{H'}(R \cup \{l\})} \irl{Null}^l,H'
}

\infern{
  V(x_1) = v_1\\
  V(x_2) = v_2\\
  (l \; \text{fresh})\\
  H' = H[l \mapsto \pairexcst{v_1}{v_2}^l]
}{
  V,H,R \; \vdash \consexcst{x_1}{x_2} \Downarrow^{space_{H'}(R \cup \{l\})} \pairexcst{v_1}{v_2}^l,H'
}

\infern{
  V(z) = \irl{Null}^l\\
  V,H,R \; \vdash e_1 \Downarrow^{s_1} v_1, H_1 \\
}{
  V,H,R \; \vdash \listcaseexcst{z}{e_1}{x}{xs}{e_2} \Downarrow^{s_1} v_1,H_1
}

\infern{
  V(z) = \pairexcst{v_h}{v_t}^l \\
  V[x \mapsto v_h, xs \mapsto v_t],H,R \; \vdash e_2 \Downarrow^{s_2} v_2, H_2 \\
}{
  V,H,R \; \vdash \listcaseexcst{z}{e_1}{x}{xs}{e_2} \Downarrow^{s_2} v_2,H_2
}

\infern{
  V,H,R \cup locs_{V,H}(\irl{lam}(x : \tau.e_2)) \; \vdash e_1 \Downarrow^{s_1} v_1, H' \\
  V[x \mapsto v_1],H',R \; \vdash e_2 \Downarrow^{s_2} v_2, H_2 \\
}{
  V,H,R \; \vdash \irl{let}(e_1; x : \tau.e_2) \Downarrow^{\max{(s_1,s_2)}} v_2,H_2
}
\end{mathpar}

\section{Heap allocation semantics}

The following rules will use judgements of the form:
\[
\fbox{$V,H\; \vdash e \Downarrow^s v, H'$}
\]

Where $V : VID \to Val$ and $H : Loc \to Val$. This can be read as: under stack $V$ and heap $H$
the expression $e$ evaluates to $v$ while allocating $s$ heap cells, and engenders a new heap $H'$.\\

This is different from the above GC semantics, since $s$ counts the number of heap allocations, not the maximum heap size.
Hence this heap allocation semantics is an upperbound on the GC semantics.

We define the following metric for measuring the number of heap allocations:

\begin{align*}
  K^{int} = 1\\
  K^{true} = 1\\
  K^{false} = 1\\
  K^{null} = 1\\
  K^{tuple} = 1\\
  K^{nil} = 1\\
  K^{cons} = 1\\
  K^{\_} = 0
\end{align*}

Where $K^{\_}$ are all other constants.

\begin{mathpar}

\infern
{ x \in dom(V)
}
{V,H \vdash x \Downarrow^{K^{var}} V(x),H}

\infern
{
  (l \; \text{fresh})\\
  H' = H[l \mapsto n^l]
}{
  V,H \vdash \numeral{n} \Downarrow^{K^{int}} n^l,H'
}

\infern{
  (l \; \text{fresh})\\
  H' = H[l \mapsto \irl{T}^l]
}{
  V,H \vdash \irl{T} \Downarrow^{K^{true}} \irl{T}^l,H'
}

\infern{
  (l \; \text{fresh})\\
  H' = H[l \mapsto \irl{F}^l]
}{
  V,H \vdash \irl{F} \Downarrow^{K^{false}} \irl{F}^l,H'
}

\infern{
  (l \; \text{fresh}) \\
  H' = H[l \mapsto \irl{Null}^l]
}{
  V,H \vdash () \Downarrow^{K^{null}} \irl{Null}^l,H'
}

\infern{
  V(x) = \irl{T}^l\\
  V,H \vdash e_1 \Downarrow^{s_1} v_1, H_1
}{
  V,H \vdash \ifexabt{x}{e_1}{e_2} \Downarrow^{s_1} v_1, H_1
}

\infern{
  V(x) = \irl{F}^l\\
  V,H \vdash e_2 \Downarrow^{s_2} v_2, H_2
}{
  V,H \vdash \ifexabt{x}{e_1}{e_2} \Downarrow^{s_2} v_2, H_2
}

% function

\infern{
  (l \; \text{fresh})\\
  H' = H[l \mapsto (V,x.e)^l]
}{
  V,H \vdash \irl{lam}(x : \tau.e) \Downarrow^{K^{lam}} (V, x.e)^l,H'
}

\infern{
  V(f) = (V_1, x.e)^{l_1} \\
  V(x) = v_1 \\
  V[x \mapsto v_1], H \vdash e \Downarrow^s v, H'
}{
  V,H \vdash \appcst{f}{x} \Downarrow^{s} v,H'
}

% tuples

\infern{
  V(x_1) = v_1 \\
  V(x_2) = v_2 \\
  (l \; \text{fresh})\\
  H' = H[l \mapsto \pairexcst{v_1}{v_2}^l]
}{
  V,H \vdash \pairexcst{x_1}{x_2} \Downarrow^{K^{tuple}} \pairexcst{v_1}{v_2}^l,H'
}

\infern{
  V(x) = \pairexcst{v_1}{v_2}^l\\
}{
  V,H \vdash \fstexcst{x} \Downarrow^{K^{fst}} v_1,H
}

\infern{
  V(x) = \pairexcst{v_1}{v_2}^l\\
}{
  V,H \vdash \sndexcst{x} \Downarrow^{K^{snd}} v_2,H'
}

% lists

\infern{
  (l \; \text{fresh})\\
  H' = H[l \mapsto \irl{Null}^l]
}{
  V,H \vdash \nilexabt \Downarrow^{K^{nil}} \irl{Null}^l,H'
}

\infern{
  V(x_1) = v_1\\
  V(x_2) = v_2\\
  (l \; \text{fresh})\\
  H' = H[l \mapsto \pairexcst{v_1}{v_2}^l]
}{
  V,H \vdash \consexcst{x_1}{x_2} \Downarrow^{K^{cons}} \pairexcst{v_1}{v_2}^l,H'
}

\infern{
  V(z) = \irl{Null}^l\\
  V,H \vdash e_1 \Downarrow^{s_1} v_1, H_1 \\
}{
  V,H \vdash \listcaseexcst{z}{e_1}{x}{xs}{e_2} \Downarrow^{s_1} v_1,H_1
}

\infern{
  V(z) = \pairexcst{v_h}{v_t}^l \\
  V[x \mapsto v_h, xs \mapsto v_t],H \vdash e_2 \Downarrow^{s_2} v_2, H_2 \\
}{
  V,H \vdash \listcaseexcst{z}{e_1}{x}{xs}{e_2} \Downarrow^{s_2} v_2,H_2
}

\infern{
  V,H \vdash e_1 \Downarrow^{s_1} v_1, H' \\
  V[x \mapsto v_1],H' \vdash e_2 \Downarrow^{s_2} v_2, H_2 \\
}{
  V,H \vdash \irl{let}(e_1; x : \tau.e_2) \Downarrow^{s_1 + s_2} v_2,H_2
}
\end{mathpar}

\section{Soundness for heap allocation}

We simplify the soundness proof of the type system for the general metric to one with monotonic resource.
(No function types for now)

\begin{theorem}[Soundness]
\label{a} let $H \vDash V : \Gamma$ and $\Sigma; \Gamma \sststile{q'}{q} e : B$
\begin{enumerate}
  \item If $V,H \vdash e \leadsto v$
\end{enumerate}
\end{theorem}



\end{document}
